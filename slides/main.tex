\documentclass{beamer}

\usepackage{graphicx,hyperref,udesc,url}
\usepackage[utf8]{inputenc}
\usepackage[T1]{fontenc}
\usepackage{booktabs}
\usepackage{caption}
\usepackage[portuges]{babel}
\usepackage{ amssymb }

\setbeamertemplate{caption}[numbered]

\usepackage[
    backend=bibtex,
    style=alphabetic,
    citestyle=authoryear
]{biblatex}

\addbibresource{bibliografia.bib}

\newcommand{\ccite}[1]{(\citeauthor{#1}, \citeyear{#1})}

\title[Modelo Slides Udesc]{Modelo de slides UDESC}

\author[Rafael Castro, Renan S. Silva]{
    Rafael Castro, Renan S. Silva\\\medskip
    {\small \url{rafaelcgs10@gmail.com}} \\ 
{\small \url{uber.renan@gmail.com}}}

\institute[UDESC]{
    Departamento de Ci\^encia da Computa\c{c}\~ao \\
    Centro de Ci\^encias e Tecnol\'ogias\\
Universidade do Estado de Santa Catarina}

\begin{document}

\begin{frame}
    \titlepage

\end{frame}

\section{Introdução}
\begin{frame}
    \frametitle{Introdução}

    \begin{figure}[!htb]
        \centering
        \begin{minipage}{0.48\textwidth}
            \begin{itemize}
                %\item Segundo~\ccite{de2011algoritmo} e~\ccite{qiao2010algorithm}, o planejamento operacional de uma empresa de transporte de urbano pode ser dividido conforme a figura~\ref{fig_etapas};
                \item O planejamento operacional de uma empresa de transporte de urbano pode ser dividido conforme a figura~\ref{fig_etapas};
                \item Este trabalho tem como objetivo resolver problema do Escalonamento de Tripulação (CSP);
            \end{itemize}
        \end{minipage}
    %
        \begin{minipage}{.48\textwidth}
        {
            \centering
            \includegraphics[height=0.8\textheight]{../tcc/figuras/etapas.pdf}
            \captionof{figure}{Etapas do planejamento}
            \label{fig_etapas}
        }
        \end{minipage}
    \end{figure}
\end{frame}

\begin{frame}
    \frametitle{Relevância}

    \begin{description}
        \item [Teória] O CSP é um problema $\mathcal{NP}$-Hard, que pode ser reduzido para o problema de cobertura/particionamento de conjuntos;
        \item [Prática]\ccite{zeren2012improved} afirma que os gastos com a tripulação são a segunda maior fonte de gastos das empresas, atrás apenas dos gastos com combustíveis;
    \end{description}
\end{frame}

\begin{frame}
    \frametitle{Definição}

    O CSP consiste determinar jornadas para um conjunto de tripulantes, onde

    \begin{description}
        \item[Tarefa] É uma atividade que deve ser realizada, que possui um tempo de inicio e fim predefinidos;
        \item[Jornada] É um conjunto de tarefas que devem ser executadas por uma mesma tripulação;
        \item[$\blacktriangleright$]Jornadas possuem restrições, carga horaria máxima, etc;
        \item[$\blacktriangleright$]Existe um custo para deslocar-se entre duas tarefas;
        \item[$\blacktriangleright$]Deseja-se minimizar o custo total de cobrir todas as jornadas;
    \end{description}
\end{frame}

\begin{frame}
    \frametitle{Introdução}

\end{frame}

\begin{frame}
    \frametitle{Introdução}

\end{frame}

\section{Introdução}
\begin{frame}
    \frametitle{Introdução}

\end{frame}

\section{Referências}

\begin{frame}
    %\bibliography{bibliografia}
    \printbibliography
\end{frame}

\end{document}
