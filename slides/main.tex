\documentclass{beamer}

\usepackage{graphicx,hyperref,udesc,url}
\usepackage[utf8]{inputenc}
\usepackage[T1]{fontenc}
\usepackage{booktabs}
\usepackage{caption}
\usepackage[portuges]{babel}
\usepackage{ amssymb }

\setbeamertemplate{caption}[numbered]

\usepackage[
    backend=bibtex,
    style=alphabetic,
    citestyle=authoryear
]{biblatex}

\addbibresource{bibliografia.bib}

\newcommand{\ccite}[1]{(\citeauthor{#1}, \citeyear{#1})}

\title[Modelo Slides Udesc]{Modelo de slides UDESC}

\author[Rafael Castro, Renan S. Silva]{
    Rafael Castro, Renan S. Silva\\\medskip
    {\small \url{rafaelcgs10@gmail.com}} \\ 
{\small \url{uber.renan@gmail.com}}}

\institute[UDESC]{
    Departamento de Ci\^encia da Computa\c{c}\~ao \\
    Centro de Ci\^encias e Tecnol\'ogias\\
Universidade do Estado de Santa Catarina}

\begin{document}

\begin{frame}
    \titlepage

\end{frame}

\begin{frame}{Overview}
    \tableofcontents
\end{frame}

\section{Introdução}
\begin{frame}
    \frametitle{Introdução}

    \begin{figure}[!htb]
        \centering
        \begin{minipage}{0.48\textwidth}
            \begin{itemize}
                %\item Segundo~\ccite{de2011algoritmo} e~\ccite{qiao2010algorithm}, o planejamento operacional de uma empresa de transporte de urbano pode ser dividido conforme a figura~\ref{fig_etapas};
                \item O planejamento operacional de uma empresa de transporte de urbano pode ser dividido conforme a figura~\ref{fig_etapas};
                \item Este trabalho tem como objetivo resolver problema do Escalonamento de Tripulação (CSP);
            \end{itemize}
        \end{minipage}
    %
        \begin{minipage}{.48\textwidth}
        {
            \centering
            \includegraphics[height=0.8\textheight]{../tcc/figuras/etapas.pdf}
            \captionof{figure}{Etapas do planejamento}
            \label{fig_etapas}
        }
        \end{minipage}
    \end{figure}
\end{frame}

\begin{frame}
    \frametitle{Relevância}

    \begin{description}
        \item [Teória] O CSP é um problema $\mathcal{NP}$-Hard, que pode ser reduzido para o problema de cobertura/particionamento de conjuntos;
        \item [Prática]\ccite{zeren2012improved} afirma que os gastos com a tripulação são a segunda maior fonte de gastos das empresas, atrás apenas dos gastos com combustíveis;
    \end{description}
\end{frame}

\begin{frame}
    \frametitle{Definição}

    O CSP consiste determinar jornadas para um conjunto de tripulantes, onde

    \begin{description}
        \item[Tarefa] É uma atividade que deve ser realizada, que possui um tempo de inicio e fim predefinidos;
        \item[Jornada] É um conjunto de tarefas que devem ser executadas por uma mesma tripulação;
        \item[$\blacktriangleright$]Jornadas possuem restrições, carga horaria máxima, etc;
        \item[$\blacktriangleright$]Existe um custo para deslocar-se entre duas tarefas;
        \item[$\blacktriangleright$]Deseja-se minimizar o custo total de cobrir todas as jornadas;
    \end{description}
\end{frame}

\section{Formulação}
\begin{frame}
    \frametitle{Formulação}
    Dentre as possíveis modelagens possíveis para o CSP, utilizou-se uma com base no problema de particionamento de conjuntos(SPP);

    \begin{figure}[!htb]
        \centering
        \begin{minipage}{0.48\textwidth}
            $A = \begin{pmatrix}
                1 & 1 & 0 & 0 & 0 & 0 & 0 \\
                0 & 0 & 1 & 1 & 1 & 1 & 1 \\
                1 & 0 & 1 & 0 & 0 & 1 & 0 \\
                0 & 0 & 0 & 1 & 1 & 1 & 0 \\
                0 & 1 & 0 & 0 & 1 & 0 & 1 \\
            \end{pmatrix}$
        \end{minipage}
%
        \begin{minipage}{.48\textwidth}
            \begin{subequations}
                \label{spppp}
                \begin{align}
                \label{spp2}  \text{min} \: \sum_{j \in J} c_j x_j \\
                \label{spp22} \sum_{j \in J} a_{ij} x_j = 1, \forall i \in I \\
                \label{spp24} x_j \in \{0, 1\}, \forall j \in J
            \end{align}
        \end{subequations}
        \end{minipage}
    \end{figure}
\end{frame}

\begin{frame}
    \frametitle{Modelando o CSP com o SPP}
    \begin{figure}[!htb]
        \centering
        \begin{minipage}{0.48\textwidth}
        {
            \centering
            \includegraphics[width=0.8\textwidth]{../tcc/figuras/graph.pdf}
            \captionof{figure}{Possíveis jornadas representadas em um grafo}
            \label{treta}
        }
        \end{minipage}
%
        \begin{minipage}{.48\textwidth}
            \begin{itemize}
                \item Deve-se enumerar todos as possíveis jornadas viáveis;
                \item O número de jornadas cresce exponencialmente em função do número de tarefas;
                \item Se não forem enumeradas todas as jornadas, perde-se a solução ótima;
                \item Enumerar todas as jornadas é inviável;
            \end{itemize}
        \end{minipage}
    \end{figure}
\end{frame}

\section{Geração de colunas}
\begin{frame}
    \frametitle{Geração de colunas}
    O método de geração de colunas é capaz de:

    \begin{itemize}
        \item Lidar com um grande número de variáveis;
        \item Considerar implicitamente todas as jornadas;
        \item Iniciar com um conjunto reduzido de jornadas;
        \item Encontrar iterativamente todas as jornadas necessárias para encontrar a solução ótima;
    \end{itemize}
\end{frame}

\begin{frame}
    \frametitle{Funcionamento}

    \begin{figure}[!htb]
        \centering
        \begin{minipage}{0.55\textwidth}
            \begin{itemize}
                \item A geração de colunas é dividida em dois problemas menores: Problema mestre e subproblema;
                \item O problema mestre é o problema original com um conjunto reduzido de jornadas(colunas);
                \item O subproblema é um problema de programação linear inteira que determina qual jornada
                    deve ser inserida no problema mestre;
            \end{itemize}
        \end{minipage}
%
        \begin{minipage}{.40\textwidth}
{
    \centering
    \includegraphics[width=1\linewidth]{../tcc/figuras/gercolumn.pdf}
    \captionof{figure}{Processo de geração de colunas}
    \label{treta}
}
        \end{minipage}
    \end{figure}

\end{frame}

\section{Conclusões parciais}
\begin{frame}
    \frametitle{Conclusões parciais}
    \begin{itemize}
        \item Kappa;
        \item Keepo;
        \item Kippa;
        \item Klappa;
        \item KappaRoss;
    \end{itemize}
\end{frame}

%\section{Referências}

\begin{frame}
    %\bibliography{bibliografia}
    \printbibliography
\end{frame}

\end{document}
