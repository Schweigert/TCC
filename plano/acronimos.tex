%%Arquivo em UTF8
%LISTA DE ACRONIMOS A SER USADO EM ARTIGOS E TCC da área da computação
%INICIADO POR CHARLES C. MIERS E COMPLEMENTADO POR GLAUBER C. BATISTA

%Defining: \acro{acronym}[short name]{full name}
% Usaging:
% \ac{acronym}     -- writes the full name followed by the acronym in brackets; later calls will write only the acronym
%\acf{acronym}     -- writes the full name followed by the acronym in brackets
%\acs{acronym}     -- writes the short name only
%\acl{acronym}     -- writes the full name only
% Use p at the end of previous commands for plural form (e.g., \acp for the plural form of \ac)
%\acresetall        -- reset usage of all acronyms (i.e., \ac will print full name again)
%\acused            -- mark the acronym as used

\begin{acronym}%listof
\acro{API}[API]{Application Programming Interface}
\acro{ASP}[ASP]{Authentication Service Provider}
\acro{AWS}[AWS]{Amazon Web Services}
\acro{CA}[CA]{Certification Authority}
\acro{DC}[DC]{Data Center}
\acro{DNS}[DNS]{Domain Name System}
\acro{FTP}[FTP]{File Transfer Protocol}
\acro{HTTP}[HTTP]{HyperText Transfer Protocol}
\acro{IdP}[IdP]{Identity Provider}
\acro{LARC}[LARC]{Laborat\'{o}rio de Arquitetura e Redes de Computadores}
\acro{LabP2D}[LabP2D]{Laborat\'{o}rio de Processamento Paralelo e Distribu\'{i}do}
\acro{NTP}[NTP]{Network Time Protocol}
\acro{PKI}[PKI]{Public Key Infrastructure}
\acro{RA}[RA]{Registration Authority}
\acro{SAML}[SAML]{Security Assertion Markup Language}
\acro{SHA}[SHA]{Secure Hash Algorithm}
\acro{SP}[SP]{Service Provider}
\acro{SPF}[SPF] {Single Point of Failure}
\acro{SSL}[SSL]{Secure Sockets Layer}
\acro{SSO}[SSO]{Single Sign-On}
\acro{TI}[IT]{Information Technology}
\acro{TIC}[TIC]{Tecnologia da Informa\c{c}\~{a}o e Comunica\c{c}\~{a}o}
\acro{TTP}[TTP]{Trusted Third-Part}
\acro{UA}[UA]{User-Agent}
\acro{UDESC}[UDESC]{Universidade do Estado de Santa Catarina}
\acro{VLAN}[VLAN]{Virtual Local Area Network}
\acro{XML}[XML]{Extensible Markup Language}

\end{acronym}%listofacronyms}
