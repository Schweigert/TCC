\section{Metodologia}
\label{met}

Para que seja poss\'{i}vel atingir os objetivos, serão utilizados dois métodos: pesquisa referenciada, aplicada durante o Trabalho de Conclusão de Curso I, e pesquisa aplicada, aplicada durante o Trabalho de Conclusão de Curso II.

Na pequisa referenciada serão levantadas arquiteturas da literatura, buscando as mais adequadas ao gerenciamento de recursos no microserviço de gerenciamento de mundos virtuais. Será dividido em 2 situações: levantamento de arquiteturas de microserviços descritas na literatura, levantando a topologia, protocolos e objetivos dessas arquiteturas. Por fim o levantamento de possiveis simulações para efetuar testes durante a pesquisa aplicada (\textit{e.g., colônia de formigas}).

Na pesquisa aplicada, o resultado a ser obtido é a análise e caracterização das arquiteturas de microserviços empregados a jogos MMORPG, visando uma profunda análise sobre os recursos computacionais consumidos. Divide-se em algumas 3 situações: aplicação das arquiteturas descritas e selecionadas, realização dos testes utilizando a simulação descrita na pesquisa referenciada e análise dos dados coletados.

Para que os resultados sejam alcançados, são definidas as seguintes etapas:

\begin{enumerate}
  \item \textbf{Levantamento de fichamento das referências:} Pesquisa de fontes para embasamento teórico do trabalho, com base nos objetivos específicos;

  \item \textbf{Consolidação das referências:} Fundamentação necessária para compreender o objeto de trabalho a tingir o objetivo do Trabalho de Conclusão de Curso I;

  \item \textbf{Identificação e caracterização de arquiteturas descritas na literatura:} Enumeração e caracterização das arquiteturas de microserviços descritas, bem como se elas se empenham em solucionar problemas de gerenciamento espacial;

  \item \textbf{Identificação e caracterização de simulações aplicáveis ao teste:} Eleger e caracterizar uma simulação a ser aplicada nos testes que demandem das arquiteturas;

  \item \textbf{Escrita Trabalho de Conclusão de Curso I};

  \item \textbf{Aplicação das arquiteturas selecionadas na pesquisa referênciada:} Desenvolvimento das arquiteturas descritas para aplicação dos testes;

  \item \textbf{Aplicação da simulação descrita na pesquisa referênciada:} Desenvolvimento da simulação para iteragir com as arquiteturas de microserviços;

  \item \textbf{Planejamento de testes:} Descrição e planejamento de testes sobre as arquiteturas levantadas da literatura.

  \item \textbf{Realização dos testes utilizando a simulação descrita na pesquisa referênciada:} Aplicação em um \textit{cluster} pré-definido para todos os testes com recurso limitado, metrificando os dados obtidos em gráficos para comparação;

  \item \textbf{Análise das arquiteturas testadas:} Analisar as métricas obtidas e descrever resultados obtidos;

  \item \textbf{Escrita Trabalho de Conclusão de Curso II};
\end{enumerate}
