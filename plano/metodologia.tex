\section{Metodologia}
\label{met}

Para que seja possível atingir os objetivos, serão utilizados dois métodos: pesquisa referenciada, desenvolvida durante o Trabalho de Conclusão de Curso I, e pesquisa aplicada, desenvolvida durante o Trabalho de Conclusão de Curso II.

Na pequisa referenciada serão levantadas arquiteturas da literatura, buscando as mais adequadas ao escopo deste trabalho. Será dividido em três situações: levantamento de arquiteturas de microserviços descritas na literatura e caracterização dessas arquiteturas. 
%
Por fim, o levantamento e caracterização de possíveis simulações para efetuar testes durante a pesquisa aplicada.

Na pesquisa aplicada, o resultado a ser obtido é a análise das arquiteturas de microserviços caracterizadas, visando uma análise sobre os recursos computacionais consumidos e identificação de seus gargalos. 
%
Divide-se em três situações: aplicação das arquiteturas descritas e selecionadas, realização dos testes utilizando a simulação descrita na pesquisa referenciada e análise dos dados coletados.

Para que os resultados sejam alcançados, são definidas as seguintes etapas:

\begin{enumerate}
  \item \textbf{Levantamento e fichamento das referências:} Pesquisa de fontes para embasamento teórico do trabalho, com base nos objetivos específicos;

  \item \textbf{Consolidação das referências:} Compreensão e seleção de artefatos literários que permitam atingir o objetivo do Trabalho de Conclusão de Curso I;

  \item \textbf{Identificação e caracterização de arquiteturas descritas na literatura:} Enumeração e caracterização das arquiteturas de microserviços descritas na literatura, bem como os seus objetivos;

  \item \textbf{Especificação das arquiteturas selecionadas:} Especificar o funcionamento das arquiteturas selecionadas.

  \item \textbf{Identificação e caracterização de simulações aplicáveis ao teste:} Eleger e caracterizar a simulação a ser aplicada nos testes;

  \item \textbf{Especificação da simulação elegida:} Especificar os requisitos;

  \item \textbf{Escrita Trabalho de Conclusão de Curso I};

  \item \textbf{Desenvolvimento da simulação:} Desenvolvimento da simulação para interagir com as arquiteturas de microserviços;

  \item \textbf{Desenvolvimento da arquitetura:} Desenvolvimento da arquitetura para executar os testes;

  \item \textbf{Aplicação das arquiteturas selecionadas na pesquisa referênciada:} Aplicação das arquiteturas descritas sobre uma nuvem computacional;

  \item \textbf{Realização dos testes utilizando a simulação descrita na pesquisa referênciada:} Execução de testes da arquitetura desenvolvida sobre a nuvem computacional;

  \item \textbf{Análise das arquiteturas testadas:} Analisar as métricas obtidas dos testes e descrever resultados, identificando possíveis gargalos nas arquiteturas;

  \item \textbf{Otimização para melhorar as métricas obtidas:} Analisar pontos de gargalo nos microserviços analisados e propor soluções viáveis para aumentar o desempenho desses sistemas.

  \item \textbf{Escrita Trabalho de Conclusão de Curso II};
\end{enumerate}
