\section{Introdução e Justificativa}
\label{sec:int}

Os avanços tecnológicos de sistemas distribuídos estão permitindo que pessoas utilizem de serviços com um grande volume de dados para aplicações sensíveis a latência. Essa situação é bem favorável a área de jogos massivos, tendo atraído pesquisadores para testar e validar novas abordagens em serviços com objetivando reduzir a carga desses serviços e reduzir o impacto a latência para o usuário final, resultando em uma melhor experiência aos jogadores da categoria de jogo tratado no presente trabalho\cite{mmo_analytic}.

O mercado de jogos massivos multijogadores de interpretação (\textit{MMORPG}) vem crescendo desde 2012 \cite{new_york_times}, sendo 2016 um dos mais lucrativos até então, segundo o site Statista \cite{statista_2016}. A sua projeção para 2018 é que sejam arrecadados mais de 30 bilhões de dólares americanos com esta categoria de jogos \cite{statista_2018}, um aumento de 20\% a mais sobre o ano de 2016.

\textit{Massively Multiplayer Online} ou \textit{MMO} (como são popularmente conhecidos) são os jogos de interpretação multijogador massivos. A principal característica desse estilo de jogo é a comunicação e representação virtual de um mundo fantasia onde cada jogador pode interagir com objetos virtuais compartilhados ou tomar ações sobre outros de jogadores em tempo real, tendo como principais objetivos a resolução de problemas conforme a sua regra de design, o desenvolvimento do personagem e a interação entre os jogadores \cite{video_game_technologies}.

Serviços MMO são utilizados como negócio viável e lucrativo, a qual a experiência de uso do usuário final é um fator crítico para o sucesso. A maioria dos jogos massivos no mercado atual dispoem de uma arquitetura com diversos servidores, onde a performance destes servidores influenciam na experiência do usuário final. Modelar um sistema performático em torna-se um trabalho essencial para a satisfação do usuário final\cite{1417630}.

Uma métrica popular para mensurar a performance do jogo é o número de conexões. Em geral, se o servidor caos ultrapasse esses limites, problemas de experiência com o usuário poderam ocorrer\cite{1417630}:

\begin{enumerate}
  \item Longo tempo de resposta aos clientes.
  \item Dessincronização com os clientes, promovendo roll back.
  \item Problemas internos ao servidor, como sobrecarga no banco de dados ou regras de negócio.
  \item Queda da conexão com os clientes.
\end{enumerate}

Essas ocorrências podem ser geradas por:

\begin{enumerate}
  \item Regra de negócio limitante.
  \item Baixo poder computacional do servidor.
  \item Grande complexidade computacional para computar a regra de negócios do jogo.
  \item Uma arquitetura que consome muito recurso pela própria infraestrutura.
  \item Uma arquitetura que não foi planejada para receber determinado número de conexões.
\end{enumerate}
