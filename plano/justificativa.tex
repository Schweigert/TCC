\section{Introdução e Justificativa}
\label{sec:int}

Jogos de interpretação multijogador massivos surgiram de uma categoria de jogos de mesa baseados em representação de personagens nomeado como \textit{Role-Playing Game} (RPG), sendo um grande exemplo o jogo \textit{Dungeon and Dragons} \cite{tsr1980dungeons}. A principal característica desse estilo de jogo é a comunicação e representação virtual de um mundo onde cada jogador pode interagir com objetos virtuais compartilhados no mundo ou outros jogadores, tendo como maior objetivo do jogo a resolução de problemas e desenvolvimento do personagem \cite{video_game_technologies}.

O mercado de jogos massivos vem crescendo desde 2012 \cite{new_york_times}, sendo 2016 um dos mais lucrativos até então segundo o site Statista \cite{statista_2016}. A sua projeção para 2018 é que sejam arrecadados mais de 30 bilhões de dólares americanos com essa categoria de jogos \cite{statista_2018}, um aumento de 20% a mais sobre o ano de 2016.

Se faz de extrema importância a pesquisa sobre este gênero para melhorar a qualidade de serviço e otimizar o transporte de dados entre o cliente e o serviço online. Um dos grandes problemas atuais é a escalabilidade e qualidade desses serviços, visto que temos que entregar um serviço escalável conforme a demanda de usuários sem perder tempo de resposta, tornando um serviço crítico e sensível a pequenas alterações e detalhes de sua implementação \cite{cloud_fog}.

\begin{figure}[h]
\caption{Plot gráfico comparando o número de conexões ao decorrer de 11 dias
\cite{system_performance}.}
\centering
\includegraphics[width=1\textwidth]{img/connection_peer_hour.png}
\label{fig:conection_peer_hour}
\end{figure}

A Figura ~\ref{fig:conection_peer_hour} mostra um serviço de MMORPG utilizando 4 servidores distintos separados por um multiplexador. Pode-se perceber que existem picos próximos a 2250 conexões.

A análise realizada por \cite{system_performance} mostra um jogo de pequeno porte. Jogos de porte maior podem conter milhares de jogadores online simultaneamente. Um exemplo é o jogo RuneScape, a qual possui 90 mil jogadores online simultaneamente \cite{runescape_online_users}.

Em geral, serviços para essa categoria de jogo tendem a transitar e manipular uma quantia grande de dados, e por esse motivo torna-se alvo de diversas pesquisas nas áreas de sistemas distribuidos e redes de computadores. Nos últimos anos a pesquisa dessa área estava focada em tentar resolver problemas de escalabilidade e distribuição de carga \cite{load_balance} \cite{kd_tree}, trazendo como resposta positiva novos métodos para escalar os serviços de jogos de forma satisfatória, entretanto de grande complexidade \cite{system_performance}. Nos ultimos anos houve um elevado investimento na pesquisa nas áreas de BigData e Sistemas Distribuídos para suprir a demanda do mercado. Essas pesquisas revelaram novos métodos viáveis de replicação de dados, chamadas remota de procedimentos e sincronização de alto desempenho com o objetivo de tornar viável escrever aplicações de grande porte com menor custo computacional. Essas pesquisas trouxeram vários avanços tecnológicos a qual atualmente já encontram-se desenvolvidos e aplicados em diversas tecnologias disponíveis de forma OpenSource. Grandes exemplos são novos bancos de dados que desprezam o principio ACID e tomam o BASE como novo princípio, linguagens de programação a qual foram pensadas para serem facilmente escaláveis e modularizadas e a utilização de protocolos de alto desempenho tentando otimizar a transição de dados pelo protocolo TCP.

Essa pesquisa trará resposta a perguntas como:

\begin{itemize}
  \item Tais tecnologias desenvolvidas para suprir a alta demanda de aplicações que não são sensíveis a latência são eficientes para serviços de jogos massivos?
  \item Qual o grau de escalabilidade que um serviço usando essas técnicas pode obter?
  \item Esses serviços são viáveis para projetos reais?
  \item Quais os pontos fracos que obtemos com a utilização desses serviços?
  \item Ouve comprometimento da qualidade de serviço ao usuário final? E ouve algum ganho aos desenvolvedores?
  \item É possível obter métricas para comparar com experimentos anteriores e futuros do mesmo segmento?
  \item É possível ter métricas de comparação com um serviço cliente servidor convencional?
  \item É possível publicar os resultados de forma OpenSource para comunidade utilizar?
  \item Essa arquitetura proverá um método fácil de acoplar a motores gráficos já existentes no mercado?
\end{itemize}
