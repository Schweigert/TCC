\documentclass[11pt]{article}

%\usepackage{luatextra}
%\defaultfontfeatures{Ligatures=TeX}
%\usepackage{fontspec}

\usepackage{float}

\usepackage[portuges]{babel}
%\usepackage{fontspec}
\usepackage[T1]{fontenc}
%\usepackage[latin1]{inputenc}
\usepackage[utf8]{inputenc}
\usepackage[table]{xcolor}

\usepackage{microtype}

\usepackage[font=small,labelfont=bf,tableposition=top]{caption}
\usepackage[margin=1in,headheight=35pt,headsep=0.1in]{geometry}
%\usepackage[alf]{abntcite}

\usepackage{setspace}
\usepackage{amsmath}
\usepackage{amssymb}
\usepackage{amsfonts}
\usepackage{epsfig}
\usepackage[pdftex]{hyperref}
\usepackage{multirow}
\usepackage{fancyhdr}
\usepackage[nolist]{acronym}

%\usepackage{bibtex}

\listofabbreviations{Lista de Abreviaturas}
\begin{acronym}[]
	\acro{amqp}[AMQP]{{\it Advanced Message Queuing Protocol}}
	\acro{api}[API]{{\it Application Programming Interface}}
    \acro{aws}[AWS]{{\it Amazon Web Services}}
	\acro{cli}[CLI]{{\it Command Line Interface}}
	\acro{ddos}[DDoS]{{\it Distributed Denial of Service}}
	\acro{iaas}[IaaS]{{\it Infrastructure as a Service}}
    \acro{ids}[IDS]{{\it Intrusion Detection System}}
	\acro{kvm}[KVM]{{\it Kernel-based Virtual Machine}}
    \acro{ldap}[LDAP]{{\it Lightweight Directory Access Protocol}}
	\acro{nist}[NIST]{{\it National Institute of Standards and Technology}}
	\acro{paas}[PaaS]{{\it Platform as a Service}}
    \acro{rpc}[RPC]{{\it Remote Procedure Call}}
	\acro{saas}[SaaS]{{\it Software as a Service}}
	\acro{snmp}[SNMP]{{\it Simple Network Management Protocol}}
	\acro{qos}[QoS]{{\it Quality of Service}}
	\acro{sdn}[SDN]{{\it Software Defined Network}}
	\acro{vlan}[VLAN]{{\it Virtual Local Area Network}}
	\acro{vm}[VM]{{\it Virtual Machine}}
	\acro{vpn}[VPN]{{\it Virtual Private Network}}

	\acro{POV}[POF]{{\it Point of View}}

	\acro{FPS}[FPS]{{\it First-person shooter}}
	\acro{TPS}[TPS]{{\it Third-person shooter}}
	\acro{RTS}[RTS]{{\it Real-time strategy}}
	\acro{MMO}[MMO]{{\it Massively multiplayer online}}
	\acro{RPG}[RPG]{{\it Role-playing game}}
	\acro{MMORPG}[MMORPG]{{\it Massively multiplayer online role-playing game}}
	\acro{MOBA}[MOBA]{{\it Multiplayer online battle arena}}
	\acro{MMOFPS}[MMOFPS]{{\it Massively multiplayer online first-person shooter}}

	\acrodefplural{vpn}[VPNs]{{\it Virtual Private Networks}}
	\acrodefplural{vlan}[VLANs]{{\it Virtual Local Area Networks}}
	\acrodefplural{vm}[VMs]{{\it Virtual Machines}}
\end{acronym}

% Defining: \acro{acronym}[short name]{full name}
% Usaging:
% \ac{acronym}     -- writes the full name followed by the acronym in brackets; later calls will write only the acronym
% \acf{acronym}     -- writes the full name followed by the acronym in brackets
% \acs{acronym}     -- writes the short name only
% \acl{acronym}     -- writes the full name only
% Use p at the end of previous commands for plural form (e.g., \acp for the plural form of \ac)
% \acresetall        -- reset usage of all acronyms (i.e., \ac will print full name again)
% \acused                -- mark the acronym as used

%\oddsidemargin -0.7cm
%\evensidemargin -0.7cm
\topmargin -2.0cm
%\headheight 0  cm
\headsep 1.5cm
%\hoffset -1.0cm
%\footskip 40pt
%\textheight = 235mm \textwidth 185mm
\oddsidemargin 0.4cm
\evensidemargin 0.4cm
\textheight = 235mm \textwidth 165mm


\pagestyle{plain}

\usepackage{multicol}
\addtolength\columnsep{2pt}

%\newcommand{\apud}[4]{\citeauthor{#1} \mkbibparens{\citeyear{#1},\space{#2} apud \cite{#3},\space{#4}}}

\begin{document}

\pagestyle{fancy}
%\lhead{\includegraphics[width=0.3\columnwidth]{figuras/logo_dcc.png}}
\lhead{
  \includegraphics[scale=0.75]{figuras/logo_dcc.pdf}
}
\chead{
  \scriptsize{
    UNIVERSIDADE DO ESTADO DE SANTA CATARINA -- UDESC\\
    CENTRO DE CIÊNCIAS TECNOLÓGICAS -- CCT\\
    DEPARTAMENTO DE CIÊNCIA DA COMPUTAÇÃO -- DCC
  }
}
%\rhead{\includegraphics[width=0.3\columnwidth]{figuras/logo_udescjlle.png}}
\rhead{
  \includegraphics[scale=0.03]{figuras/logo_udescjlle.pdf}
}

\title{
Plano de Trabalho de Conclusão de Curso\\
Análise de arquiteturas de microsserviços empregados a jogos MMORPG voltada a otimização do uso de recursos de gerenciamento de mundos virtuais
}

\author{
Marlon Henry Schweigert -- \texttt{marlon.henry@magrathealabs.com}\\
Charles Christian Miers -- \texttt{charles.miers@udesc.br} {\it (orientador)}\\
%$<$Nome do Coorientador -- \texttt{email@coorientador} {\it (coorientador)}$>$ (se for o caso)\\
~\\
Turma 2018/1 -- Joinville/SC
}

\date{12 de Março de 2018}

\maketitle


%\singlespacing  %espaçamento simples
\onehalfspacing  %espaçamento de 1,5
%\doublespacing  %espaçamento duplo


\begin{abstract}
\noindent
  A crescente popularização de jogos massivos demanda por novas abordagens tecnológicas a fim de suprir as necessidades dos usuários com menor custo de recursos computacionais.
  %
  Projetar essas arquiteturas, do ponto de vista da rede, é algo pertinente e impactante para o sucesso desses jogos.
  %
  O objetivo deste trabalho é propor uma análise voltada a identificar abordagens para otimização dos recursos computacionais consumidos pelas arquiteturas identificadas.
  %
  Esse objetivo será atingido após realizar uma pesquisa referenciada, seguida de uma análise das principais arquiteturas e, preferencialmente, a execução de simulações usando uma nuvem computacional para auxiliar na identificação de gargalos de recursos. % e provendo soluções viáveis a esses problemas.
  %
  Os resultados obtidos auxiliarão provedores de serviços \ac{MMORPG} a reduzir gastos de manutenção e melhorar a qualidade de tais serviços.

\textbf{Palavras-chave:} \textit{arquitetura de microsserviços, desenvolvimento de jogos, rede de jogos, jogos massivos, otimização de recursos, nuvens computacionais}
\end{abstract}

\section{Introdução e Justificativa}
\label{sec:int}

Os avanços tecnológicos de sistemas distribuídos estão permitindo que pessoas utilizem de serviços com um grande volume de dados para aplicações sensíveis a latência. Essa situação é bem favorável a área de jogos massivos, tendo atraído pesquisadores para testar e validar novas abordagens em serviços com objetivando reduzir a carga desses serviços e reduzir o impacto a latência para o usuário final, resultando em uma melhor experiência aos jogadores da categoria de jogo tratado no presente trabalho\cite{mmo_analytic}.

O mercado de jogos massivos multijogadores de interpretação (\textit{MMORPG}) vem crescendo desde 2012 \cite{new_york_times}, sendo 2016 um dos mais lucrativos até então, segundo o site Statista \cite{statista_2016}. A sua projeção para 2018 é que sejam arrecadados mais de 30 bilhões de dólares americanos com esta categoria de jogos \cite{statista_2018}, um aumento de 20\% a mais sobre o ano de 2016.

\textit{Massively Multiplayer Online} ou \textit{MMO} (como são popularmente conhecidos) são os jogos de interpretação multijogador massivos. A principal característica desse estilo de jogo é a comunicação e representação virtual de um mundo fantasia onde cada jogador pode interagir com objetos virtuais compartilhados ou tomar ações sobre outros de jogadores em tempo real, tendo como principais objetivos a resolução de problemas conforme a sua regra de design, o desenvolvimento do personagem e a interação entre os jogadores \cite{video_game_technologies}.

Serviços MMO são utilizados como negócio viável e lucrativo, a qual a experiência de uso do usuário final é um fator crítico para o sucesso. A maioria dos jogos massivos no mercado atual dispoem de uma arquitetura com diversos servidores, onde a performance destes servidores influenciam na experiência do usuário final. Modelar um sistema performático em torna-se um trabalho essencial para a satisfação do usuário final\cite{1417630}.

Uma métrica popular para mensurar a performance do jogo é o número de conexões. Em geral, se o servidor caos ultrapasse esses limites, diversas ocorrências de experiência com o usuário poderam ocorrer\cite{1417630}. As ocorrências mais comuns são:

\begin{itemize}
  \item \textbf{Longo tempo de resposta aos clientes}: esta ocorrência trás uma péssima qualidade ao usuário ou até mesmo impossibilitando o uso do serviço.
  \item \textbf{Dessincronização com os clientes}: esta ocorrência promove rollback na aplicação, proporcionando desconforto ou má fluidez do jogo.
  \item \textbf{Problemas internos ao servidor}: esta ocorrência pode estar ligada a diversos outros erros como sobrecarga no banco de dados, gerenciamento lento do espaço ou inconsistências dentro do jogo perante a regra de negócios.
  \item \textbf{Queda da conexão com os clientes}.
\end{itemize}

Existem algumas causas comuns para essas falhas:

\begin{enumerate}
  \item \textbf{Baixo poder computacional do servidor}: poder computacional baixo para a qualidade de experiência do usuário final desejada.
  \item \textbf{Grande complexidade computacional}: o serviço usa algoritmos ruins ou até mesmo a regra de negócios demanda de algoritmos complexos.
  \item \textbf{Limitado pela própria arquitetura}: está limitado aos microserviços que servem o macroserviço.
  \item \textbf{Uma arquitetura que não foi planejada para receber determinado número de conexões}.
\end{enumerate}

A literatura atual propoe algumas alternativas de arquiteturas para serviços de jogos MMORPG de microserviços, \cite{stephenclarkewillson2017} \cite{albion_online_unite}. Existem algumas variações híbridas \textit{peer-to-peer}, a qual não é aceita até então por jogos comerciais por ser demasiadamente complexo a manutenção de estado de jogo.

A proposta de otimização das análises realizadas sobre as arquiteturas de microserviços para jogos massivos focada ao gerenciamento de mundos virtuais proposta pela literatura trás impacto direto a melhoria da qualidade de experiência ao usuário final, por sua vez, proporcionando soluções com melhor garantia de sucesso sobre outras arquiteturas problemáticas.

\section{Objetivos}
\label{obj}

Análisar e caracterizar arquiteturas de microsserviços empregados a jogos \ac{MMORPG} voltada a otimização do uso de recursos (\textit{\textit{e.g.,} memória e processamento}~\cite{1417630}) de gerenciamento de mundos virtuais.
%
Identificar e analisar soluções tangíveis (\textit{e.g.,} protocolos, compressão, paralelismo, \textit{etc.}~\cite{1417630}) para otimizar os recursos utilizados pelas arquiteturas analisadas.

Os objetivos específicos são:
\begin{itemize}
    \item Identificar e definir arquiteturas empregadas na categoria de jogos do presente trabalho.
    \item Identificar e definir os protocolos utilizados nessas arquiteturas.
    \item Identificar e definir os microsserviços dessas arquiteturas.
    \item Identificar e analisar ferramentas de análise de recursos para definir métricas as arquiteturas identificadas e caracterizadas.
    \item Especificar requisitos para a arquitetura estudada.
    \item Aplicar as arquiteturas descritas na literatura em uma nuvem de computadores \textit{OpenStack}.
    \item Analisar o comportamento das arquiteturas aplicadas, levantando questões de desempenho e recursos utilizados.
    \item Propor alternativas de otimização para os problemas encontrados nas devidas arquiteturas.
\end{itemize}

\section{Metodologia}
\label{met}

Para que seja possı́vel atingir os objetivos, serão utilizados dois métodos: pesquisa referenciada, aplicada durante o Trabalho de Conclusão de Curso I, e pesquisa aplicada, aplicada durante o Trabalho de Conclusão de Curso II.

Na pequisa referenciada serão levantadas arquiteturas da literatura, buscando as mais adequadas ao gerenciamento de recursos no microserviso de gerenciamento espacial do jogo, visando a análise de banda.

\section{Cronograma proposto}
\label{cro}

\vspace{0.5cm}
{\tiny
\noindent \begin{tabular}{|c||c|c|c|c|c|c|c|c|c|c|c|c|}
  \hline
  \multirow{2}{*}{\textbf{\small{Etapas}}} & \multicolumn{12}{|c||}{\textbf{\small{2018}}} \\
  \cline{2-13}
   & \textbf{J} & \textbf{F} & \textbf{M} & \textbf{A} & \textbf{M} & \textbf{J} & \textbf{J} & \textbf{A} & \textbf{S} & \textbf{O} & \textbf{N} & \textbf{D} \\
  \hline \hline
  \textbf{\small{1}} & X & X & X & X & X & X & X & X & X & X & X & X \\ \hline
\end{tabular}
}

\section{Linha de Pesquisa}

Este trabalho será desenvolvido junto ao Grupo de Redes e Aplicações Distribuídas (GRADIS) e ao Laboratório de Processamento Paralelo e Distribuído (LabP2D). Esta pesquisa abrange as áreas de Redes de Computadores, Sistemas Distribuídos, Segurança em Redes de Computadores, Processamento Paralelo e Engenharia de Software.

\section{Forma de Acompanhamento/Orientação}

O acompanhamento será realizado principalmente através de reuniões semanais ou quinzenais, com duração máxima de 1 (uma) hora. 
%
Eventualmente as reuniões poderam ser trocadas por vídeo-conferência, troca de menssagens de correio eletrônico ou telefone. 
%
O controle das tarefas a fazer serão feitas baseadas em uma ata gerada a cada reunião. 
%
Metas semanais ou quinzenais serão atribuídas para melhor acompanhamento.


\pagebreak

\bibliographystyle{abnt-alf}
\bibliography{tccplanoudesc}

\vskip 1.5cm

\begin{minipage} {0.49\linewidth}
  \centering
  \rule{7.2cm}{0.1mm}

  \textbf{\textit{Charles Christian Miers}}
\end{minipage}
\begin{minipage} {0.49\linewidth}
  \centering
  \rule{7.2cm}{0.1mm}

  \textbf{\textit{Marlon Henry Schweigert}}
\end{minipage}

\bigskip
\bigskip
\bigskip

\begin{minipage} {1\linewidth}
  \centering
  \rule{7.2cm}{0.1mm}

  \textbf{\textit{Rafael Rodrigues Obelheiro}}

  \textit{(Coordenador do GRADIS)}
\end{minipage}


\end{document}
