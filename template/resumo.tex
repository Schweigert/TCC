A \ac{TA} � uma �rea do conhecimento utilizada para identificar os recursos para proporcionar ou ampliar habilidades de pessoas com defici�ncia, incapacidades ou com mobilidade reduzida. 
%
Neste trabalho � realizada uma pesquisa referenciada sobre pessoas com defici�ncias, mais especificamente pessoas que possuem \ac{PC}, seguindo para uma pesquisa aplicada na qual � definida a especifica��o e o desenvolvimento de um software de Comunica��o Alternativa Ampliada. 
%
No conjunto de portadores de \ac{PC} o trabalho trata em especial das pessoas que possuem habilidades locomotoras limitadas em conjunto com dificuldades na fala. 
%
A especifica��o de solu��o alternativa apresentada neste trabalho, possibilita que essas pessoas se comuniquem com os seus terapeutas com objetivo de estimular sua cogni��o. 
%
O trabalho est� organizado em tr�s etapas: o contexto, que diz respeito as pessoas com defici�ncia; as iniciativas de \ac{TA} e as suas classifica��es, a especifica��o do problema que � abordada no trabalho; e a especifica��o da proposta deste trabalho.
%
Contudo, �s iniciativas em \ac{TA} tendem a possuir alguns dos seus aspectos regionilizados em fun��o das demandas locais.
%
Neste sentido, esse trabalho atuou junto a \ac{ADEJ} para identificar esse contexto regionalizado e o Grupo Assistiva da \ac{UDESC}.  
