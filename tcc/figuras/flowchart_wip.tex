\documentclass{article}

\usepackage[english, brazilian]{babel}
\usepackage[T1]{fontenc}
\usepackage[latin1]{inputenc}
\usepackage{tikz}
\usetikzlibrary{arrows}
\usetikzlibrary{calc}
\usetikzlibrary{fit}
\usetikzlibrary{positioning}
\usetikzlibrary{shapes.geometric}
\usetikzlibrary{shapes.misc}
\usetikzlibrary{shapes}


%%%<
\usepackage{verbatim}
\usepackage[active,tightpage]{preview}
\PreviewEnvironment{tikzpicture}
\setlength\PreviewBorder{5pt}%
%%%>

\begin{comment}
:Title: Simple flow chart
:Tags: Diagrams

With PGF/TikZ you can draw flow charts with relative ease. This flow chart from [1]_
outlines an algorithm for identifying the parameters of an autonomous underwater vehicle model.

Note that relative node
placement has been used to avoid placing nodes explicitly. This feature was
introduced in PGF/TikZ >= 1.09.

.. [1] Bossley, K.; Brown, M. & Harris, C. Neurofuzzy identification of an autonomous underwater vehicle `International Journal of Systems Science`, 1999, 30, 901-913


\end{comment}


\begin{document}
\pagestyle{empty}


% Define block styles
\tikzstyle{decision} = [diamond, draw,
    text width=4.5em, text badly centered, node distance=3cm, inner sep=0pt]
\tikzstyle{block} = [rectangle, draw,
    text width=5em, text centered, rounded corners, minimum height=4em]
\tikzstyle{line} = [draw, -latex']
\tikzstyle{cloud} = [draw, ellipse, node distance=3cm,
    minimum height=2em]

\begin{tikzpicture}[node distance = 2cm, auto]
    % Place nodes
    \node [block, minimum width=40mm, thick, fill=gray!25, dotted] (init) at (0,0) {Colunas iniciais};

    \node [block, minimum width=40mm, below of=init    , text width = 37mm] (add)      {Adicionar Jornadas};
    \node [block, minimum width=40mm, below of=add     , text width = 37mm] (solvespp) {Resolver o SPP};
    \node [block, minimum width=40mm, below of=solvespp, text width = 37mm] (calc)     {Calcular pre\c{c}os duais};

    \node [block   , minimum width=40mm, right=5mm of init    , text width = 37mm, thick, fill=gray!25, dotted] (heursolve) at (6,0) {Resolver o Subproblema de forma heur\'istica};
    \node [decision, below=5mm of heursolve, aspect = 2       , text width = 37mm, thick, fill=gray!25, dotted] (redcost)            {Jornada de custo reduzido < 0?};

    \node [block   , minimum width=40mm, below=5mm of redcost , text width = 37mm] (exactsolve)         {Resolver o Subproblema de forma exata};
    \node [decision, below=5mm of exactsolve, aspect = 2      , text width = 37mm] (redcost2)           {Jornada de custo reduzido < 0?};
    \node [block   , minimum width=40mm, below=6mm of redcost2, text width = 37mm] (end)                {Retornar solu\c{c}\~ao \'otima do problema mestre};

    \node [text width=1cm, below right = 2mm and -5mm of redcost ] (n1) {N\~ao};
    \node [text width=1cm, below right = 2mm and -5mm of redcost2] (n2) {N\~ao};

    \node [text width=1cm, above left  = -5mm and 6mm of redcost ] (s1) {Sim};
    \node [text width=1cm, above left  = -5mm and 6mm of redcost2] (s2) {Sim};

    \node [draw, dashed, fit=(heursolve) (redcost2) (s1)] (b1) {};
    \node [draw, dashed, fit=(add) (calc)               ] (b2) {};

    % Draw edges
    %\path [line] (init)       -> (add);
    \path [line] (add)        -> (solvespp);
    \path [line] (solvespp)   -> (calc);

    \path [line] (heursolve)  -> (redcost);
    \path [line] (redcost)    -> (exactsolve);
    \path [line] (exactsolve) -> (redcost2);
    %\path [line] (redcost2)   -> (end);

    \path [line] (redcost)    -> +(-3,0);
    \path [line] (redcost2)   -> +(-3,0);

    \draw[line ,-implies,double, double distance=2mm] (init)     -- (add);
    \draw[line ,-implies,double, double distance=2mm] (redcost2) -- (end);

    %\draw[line ,-implies,double, double distance=2mm] [yshift = 5mm](b1) -- (b2);
    %\draw[line ,-implies,double, double distance=2mm] (b2) -- (b1);

    %\draw[line ,-implies,double, double distance=2mm] (b1) -- ++(0,1) node (b2){};
    %\draw[line ,-implies,double, double distance=2mm] (b1) -- ++(0,1) node (b2){};
    %\draw[line ,-implies,double, double distance=2mm] (b2) -- (b1) coordinate[pos=2];
    %\draw[line ,-implies,double, double distance=2mm, yshift =-5mm] (b2) -- (b1);

    %\path [line] (b1) -- (b2) coordinate(1,1);
    %\path [line, xshift=1cm] (b2) -- (b1);
    \draw[transform canvas={yshift=-2cm}, -implies, double, double distance=6mm] (b1) -> (b2) node[midway,fill=white, yshift= 2.6mm, xshift= 2.1mm]  {\emph{Nova jornada}};
    \draw[transform canvas={yshift= 2cm}, -implies, double, double distance=6mm] (b2) -> (b1) node[midway,fill=white, yshift=-2.6mm, xshift=-1.5mm] {\emph{Pre\c{c}os duais}};
\end{tikzpicture}


\end{document}
