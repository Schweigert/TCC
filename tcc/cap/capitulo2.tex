\chapter{Fundamentação Teórica}
\label{cap2}

\section{Jogos Eletrônicos}

O primeiro sistema de entretenimento interativo foi construido em 1947, utilizando como base de exibição um tubo de raios catódicos, criado por Thomas Goldsmith Jr. e Estle Ray Mann.
%
Essa criação foi patenteada em janeiro de 1948, datando então o inicio dos jogos eletrônicos~\cite{Adams2014Jan, patents1947Jan}.



O jogo eletrônico, ou entretenimento interativo, é uma atividade intelectual que integra um sistema de regras, na qual utiliza tal sistema a fim de definir seus objetivos ou pontuação por meio de um computador a fim de dispertar alguma emoção ao jogador~\cite{video_game_technologies}.
%
Os jogos eletrônicos são aplicações convencionais, que executam sobre algum sistema operacional.
%
O sistema operacional, hardware ou base de execução da aplicação gráfica define a sua plataforma, \textit{e. g.,} Linux, Windows, PS4, XBox, Web.



Inicialmente os jogos eram implementados de forma simples por conta do hardware disponível durante os anos 80.
%
As implementações de jogos para videogames eram arquitetadas diretamente para algum hardware proprietário, sem sistema operacional, por muitas vezes sem utilizar comunicação por rede ou memória de disco~\cite{adams_1208533}.
%
Por sua vez, os jogos para computadores eram básicos e não utilizavam rede pela dificultade de manutenção e custo desses serviços, inviabilizando jogos dessa categoria~\cite{adams_1208533}.
%
Na década de 80, o videogame Atari foi uma plataforma popular. A sua especificação era~cite{atari_age}:

\begin{itemize}
  \item \ac{CPU} com 1.19 \ac{MHz}
  \item Processador de audio e vídeo dedicado \ac{tia}, permitindo a alteração de 40 x 192 pixels a cada frame usando a tecnologia \ac{NTSC} e 2 canais de som monofônico com 4 bits de intonação e 1 bit de volume.
  \item 128 bytes de memória \ac{RAM}, podendo ser expandido com o cartucho.
  \item Os cartuchos podem ter, no máximo, 4 kB de capacidade.
\end{itemize}

Com a popularidade da comunicação entre computadores a partir do ano 1990, jogos (\textit{e. g.,} Habitat\footnote{Habitat: \url{http://www.mobygames.com/game/c64/habitat/credits}}, Tibia\footnote{Tibia: \url{http://www.tibia.com/}, Runescape\footnote{Runescape: \url{https:\\www.runescape.com}}}) começam a utilizar, como requisito básico do jogo, a comunicação com a rede.
%
Tais jogos popularizaram os jogos do gênero MMORPG, deixando de ser aplicações locais, para ser clientes de um serviço arquitetado na Internet~\cite{adams_1208533, Adams2014Jan}.



\subsection{Árvore de gêneros de jogos}
\label{sec:arvore_de_generos_de_jogos}




\begin{itemize}
  \item \textbf{Aventura}: Esse gênero de jogos aborda a exploração e resolução de problemas lógicos baseados em algum roteiro. O primeiro jogo dessa categoria é Zork\footnote{Zork: \url{https://github.com/devshane/zork}}, lançado nos anos 1980.
  \item \textbf{Ação}: Esse gênero de jogos utiliza a coordenação motora do jogador para instigar desafios a serem completados. Um título famoso é o jogo Space Invaders\footnote{Space Invaders: \url{https://github.com/dwmkerr/spaceinvaders}}.
  \item \textbf{Ação e Aventura}: Esse gênero de jogos mescla as categorias de Aventura e Ação. Seu objetivo é desvendar problemas lógicos com relação ao ambiente enquando precisa de destreza para movimentação pelo ambiente. Um exemplo é a série de jogos Legend of Zelda\footnote{Legend of Zelda: \url{https://www.zelda.com/}}.
  \item \textbf{Jogos de interpretação ou \ac{RPG}}: Esse gênero aborda a interpretação de personagens em um universo. Ela também é uma categoria que surgiu dos jogos de \ac{RPG} de mesa junto aos jogos de Ação e Aventura. The Elder Scrolls V\footnote{The Elder Scrools V: \url{http://store.steampowered.com/agecheck/app/489830/}} é um exemplo de jogo desse gênero.
  \item \textbf{Jogos Massivos ou \ac{MMO}}: São jogos que prezam a interação entre vários jogadores. Um grande jogo dessa macro categoria é o jogo Second Life\footnote{Second Life: \url{https://secondlife.com}}.
    \begin{itemize}
      \item \textbf{\ac{MMORPG}}: Esse gênero preza pelas características dos jogos RPG, porém o jogador é imerso em um mundo compartilhado com outros jogadores que estão interagindo com o ambiente ou outros jogadores simultâneamente. O jogo Black Desert Online\footnote{Black Desert Online: \url{https://blackdesert.playredfox.com/black_desert}} é um dos principais lançamentos de 2017 da categoria.
    \end{itemize}
\end{itemize}



\subsection{Jogos Massivos}

Jogos \ac{MMORPG} são utilizados como negócio viável e lucrativo, sendo que experiência de jogabilidade na qual o usuário final será submitido é um fator crítico para o sucesso.
%
O mercado de jogos \ac{MMORPG} vem crescendo desde 2012~\cite{new_york_times}, sendo no ano de 2016 um dos mais lucrativos~\cite{statista_2016}.
%
A sua projeção para 2018 é que sejam arrecadados mais de 30 bilhões de dólares americanos com esta categoria de jogos~\cite{statista_2018}, um aumento de 20\% a mais sobre o ano de 2016.



\ac{MMORPG} são jogos de interpretação de papéis massivos.
%
A principal característica desse estilo de jogo é a comunicação e representação virtual de um mundo fantasia no qual cada jogador pode interagir com objetos virtuais compartilhados ou tomar ações sobre outros jogadores em tempo real, tendo como principais objetivos a resolução de problemas conforme a sua regra de \textit{design}, o desenvolvimento do personagem e a interação entre os jogadores\cite{video_game_technologies}.
%

Um jogo \ac{MMORPG} é arquitetado em duas partes~\cite{mmo_analytic}:
\begin{itemize}
  \item \textbf{Serviço}: É o macrosserviço que implementa as regras de negócio e requisitos do jogo.
  O serviço disponibiliza uma interface com ações possíveis ao cliente sobre algum protocolo de rede.
  \item \textbf{Cliente}: Cliente é a aplicação que realizará as requisições com a interface do macrosserviço, exibindo o estado de jogo de forma imersiva ao jogador.
\end{itemize}


Um jogo \ac{MMORPG} é arquitetado em duas partes~\cite{mmo_analytic}:
\begin{itemize}
  \item \textbf{Serviço}: É o macrosserviço que implementa as regras de negócio e requisitos do jogo.
  O serviço disponibiliza uma interface com ações possíveis ao cliente sobre algum protocolo de rede.
  \item \textbf{Cliente}: Cliente é a aplicação que realizará as requisições com a interface do macrosserviço, exibindo o estado de jogo de forma imersiva ao jogador.
\end{itemize}

A maioria dos jogos \ac{MMORPG} disponíveis no mercado estão implementados sobre uma arquitetura que executa sobre diversos servidores\cite{stephenclarkewillson2017}, nos quais o desempenho destes servidores influencia tanto na experiência de jogabilidade do usuário final, quanto no custo de manutenção destes serviços~\cite{1417630}.
%
Em especial, o presente trabalho tratará com maiores detalhes as arquiteturas utilizadas no serviço dessa categoria de jogos.

\section{Trabalhos Relacionados}
\label{sec:similares}
