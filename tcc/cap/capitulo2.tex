\chapter{Fundamentação Teórica}
\label{cap2}

\section{Jogos Eletrônicos}



Jogo eletrônico é uma forma de entretenimento digital que seguem regras de negócio a fim de proporcionar alguma emoção no jogador ao completar algum objetivo dentro deste jogo.
%
Tais jogos podem ser classificados como coletivos, individuais ou competitivos~\cite{video_game_technologies}.



Como conceito, um jogo eletrônico é um jogo controlado por computador, onde o termo \textit{entretenimento interativo} é a referência formal para qualquer plataforma a qual execute um jogo eletrônico.
%
O termo jogo significa gracejo, brincadeira, divertimento. O jogo eletrônico é uma atividade intelectual que integra um sistema de regras, na qual utiliza esse sistema de regras a fim de definir seus objetivos ou pontuação por meio de um computador~\cite{video_game_technologies}.



Uma \textit{plataforma} é um computador a qual executa as regras de negócio e exibem o jogo de forma visual ao jogador. Um \textit{video game} é um computador na qual a televisão é o dispositivo de \textit{feedback} primário~\cite{video_game_technologies}.
%
Um \textit{jogo de computador} (\textit{PC Game} ou \textit{Computer Game}) são jogos específicos para computadores pessoais, já um \textit{jogo de console} executa em um video game~\cite{video_game_technologies}.



Os jogos eletrônicos podem ser classificados conforme o seu \textit{gênero}.
%
Eles são classificados dentro de cada gênero conforme um estilo comum ou um conjunto de características (\textit{e.g.,} perspectiva, estilo de jogo, interação, objetivo, etc)~\cite{video_game_technologies}.



Um grande problema classificação por gênero é a arbitrariedade e consistência.
%
É comum um jogo eletrônico ter características de dois grandes grupos.
%
Nesses casos ele pode dividir seu nome dentre os dois gêneros ou gerar um novo gênero~\cite{video_game_technologies} dentro da árvore de categorização de jogos (Seção ~\ref{sec:arvore_de_categoria_de_jogos}).



\subsection{Árvore de categoria de jogos}
\label{sec:arvore_de_categoria_de_jogos}



A árvore de categoria de jogos descrita abaixo mostra somente ramos superficiais para categorias que não serão abordadas no atual documento.

\begin{itemize}
  \item Aventura
  \item Ação
  \item Ação e Aventura
  \item Plataforma
  \item Luta
  \item Tiro
    \begin{itemize}
      \item Tiro em primeira pessoa ou \ac{FPS}
      \item Tiro em terceira pessoa ou \ac{TPS}
    \end{itemize}
  \item Estratégia em tempo real ou \ac{RTS}
  \item Estratégia em turnos
  \item Jogos de interpretação ou \ac{RPG}
  \item Jogos Massivos ou \ac{MMO}
    \begin{itemize}
      \item \ac{MMORPG}
      \item \ac{MOBA}
      \item \ac{MMOFPS}
      \item muitos outros
    \end{itemize}
\end{itemize}

\section{Trabalhos Relacionados}
\label{sec:similares}
