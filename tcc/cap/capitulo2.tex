\chapter{Fundamentação Teórica}
\label{cap2}

\section{Jogos Eletrônicos}

O primeiro sistema de entretenimento interativo foi construido em 1947, utilizando como base de exibição um tubo de raios catódicos, criado por Thomas Goldsmith Jr. e Estle Ray Mann~\cite{Adams2014Jan}.
%
Essa criação foi patenteada em janeiro de 1948~\cite{patents1947Jan}, datando então o inicio dos jogos eletrônicos.



O termo jogo significa gracejo, brincadeira, divertimento~\cite{Adams2014Jan}.
%
O jogo eletrônico é uma atividade intelectual que integra um sistema de regras, na qual utiliza tal sistema a fim de definir seus objetivos ou pontuação por meio de um computador a fim de dispertar alguma emoção ao jogador~\cite{video_game_technologies}.
%
Como conceito formal, um jogo eletrônico é um jogo controlado por computador~\cite{Adams2014Jan}.
%
O termo \textit{entretenimento interativo} é utilizado para se referenciar a qualquer jogo eletrônico que execute sobre alguma plataforma~\cite{video_game_technologies}.
%
Uma \textit{plataforma} é um computador a qual executa as regras de negócio e exibem o jogo de forma visual ao jogador~\cite{video_game_technologies} por meio de algum dispositivo de retorno.



Por essa definição, um videogame é uma plataforma na qual a televisão é o dispositivo de retorno primário, utilizando algum hardware proprietário para executar o jogo eletrônico~\cite{video_game_technologies}.
%
Além de jogos para videogames, também existem jogos para computadores pessoais.
%
Um jogo de computador é um jogo específico para computadores pessoais, a qual executará sobre um sistema operacional de propósito geral.
%
Dessa forma, computadores pessoais também são considerados plataformas, porém categorizados pelo seu sistema operacional~\cite{video_game_technologies}.
%
Existem outras diversas plataformas, \textit{e.g.,} web, realidade virtual, dispositivos móveis em geral, etc


Além de sua categoria por plataforma, os jogos eletrônicos podem ser classificados conforme o seu \textit{gênero}.



\subsection{Árvore de gêneros de jogos}
\label{sec:arvore_de_generos_de_jogos}



O gênero de um jogo é classificado conforme um estilo comum ou um conjunto de características (\textit{e.g.,} perspectiva, estilo de jogo, interação, objetivo, etc)~\cite{video_game_technologies}.
%
Pode-se agrupar em uma árvore de gêneros porém essa árvore não é a arbitrária e consistênte.
%
É comum um jogo eletrônico ter características de dois grandes grupos distintos.
%
Nesses casos, os títulos podem estar classificados dentro de multiplos gêneros ou gerar um novo ramo ~\cite{video_game_technologies} dentro da árvore de categorização de jogos.



A árvore de categoria de jogos descreve grupo de jogos que são semelhantes por características específicas a cada jogo~\cite{video_game_technologies, Adams2014Jan}.
%
Os gêneros não são definidos pelo conteúdo do jogo, mas pelo desafio principal que será proporcionado ao jogador~\cite{Adams2014Jan}.
%
A árvore de categoria de jogos descrita abaixo mostra de forma macro as principais categorias de jogos e onde encontra-se a categoria de jogos do atual trabalho~\cite{video_game_technologies}.



\begin{itemize}
  \item \textbf{Aventura}: Esse gênero de jogos aborda a exploração e resolução de problemas lógicos baseados em algum roteiro. O primeiro jogo dessa categoria é Zork\footnote{Zork: \url{https://github.com/devshane/zork}}, lançado nos anos 1980.
  \item \textbf{Ação}: Esse gênero de jogos utiliza a coordenação motora do jogador para instigar desafios a serem completados. Um título famoso é o jogo Space Invaders\footnote{Space Invaders: \url{https://github.com/dwmkerr/spaceinvaders}}.
  \item \textbf{Ação e Aventura}: Esse gênero de jogos mescla as categorias de Aventura e Ação. Seu objetivo é desvendar problemas lógicos com relação ao ambiente enquando precisa de destreza para movimentação pelo ambiente. Um exemplo é a série de jogos Legend of Zelda\footnote{Legend of Zelda: \url{https://www.zelda.com/}}.
  \item \textbf{Jogos de interpretação ou \ac{RPG}}: Esse gênero aborda a interpretação de personagens em um universo. Ela também é uma categoria que surgiu dos jogos de \ac{RPG} de mesa junto aos jogos de Ação e Aventura. The Elder Scrolls V\footnote{The Elder Scrools V: \url{http://store.steampowered.com/agecheck/app/489830/}} é um exemplo de jogo desse gênero.
  \item \textbf{Jogos Massivos ou \ac{MMO}}: São jogos que prezam a interação entre vários jogadores. Um grande jogo dessa macro categoria é o jogo Second Life\footnote{Second Life: \url{https://secondlife.com}}.
    \begin{itemize}
      \item \textbf{\ac{MMORPG}}: Esse gênero preza pelas características dos jogos RPG, porém o jogador é imerso em um mundo compartilhado com outros jogadores que estão interagindo com o ambiente ou outros jogadores simultâneamente. O jogo Black Desert Online\footnote{Black Desert Online: \url{https://blackdesert.playredfox.com/black_desert}} é um dos principais lançamentos de 2017 da categoria.
    \end{itemize}
\end{itemize}



\subsection{Jogos Massivos}

Jogos \ac{MMORPG} são utilizados como negócio viável e lucrativo, sendo que experiência de jogabilidade na qual o usuário final será submitido é um fator crítico para o sucesso.
%
O mercado de jogos \ac{MMORPG} vem crescendo desde 2012~\cite{new_york_times}, sendo no ano de 2016 um dos mais lucrativos~\cite{statista_2016}.
%
A sua projeção para 2018 é que sejam arrecadados mais de 30 bilhões de dólares americanos com esta categoria de jogos~\cite{statista_2018}, um aumento de 20\% a mais sobre o ano de 2016.



\ac{MMORPG} são jogos de interpretação de papéis massivos.
%
A principal característica desse estilo de jogo é a comunicação e representação virtual de um mundo fantasia no qual cada jogador pode interagir com objetos virtuais compartilhados ou tomar ações sobre outros jogadores em tempo real, tendo como principais objetivos a resolução de problemas conforme a sua regra de \textit{design}, o desenvolvimento do personagem e a interação entre os jogadores\cite{video_game_technologies}.
%

Um jogo \ac{MMORPG} é arquitetado em duas partes~\cite{mmo_analytic}:
\begin{itemize}
  \item \textbf{Serviço}: É o macrosserviço que implementa as regras de negócio e requisitos do jogo.
  O serviço disponibiliza uma interface com ações possíveis ao cliente sobre algum protocolo de rede.
  \item \textbf{Cliente}: Cliente é a aplicação que realizará as requisições com a interface do macrosserviço, exibindo o estado de jogo de forma imersiva ao jogador.
\end{itemize}


Um jogo \ac{MMORPG} é arquitetado em duas partes~\cite{mmo_analytic}:
\begin{itemize}
  \item \textbf{Serviço}: É o macrosserviço que implementa as regras de negócio e requisitos do jogo.
  O serviço disponibiliza uma interface com ações possíveis ao cliente sobre algum protocolo de rede.
  \item \textbf{Cliente}: Cliente é a aplicação que realizará as requisições com a interface do macrosserviço, exibindo o estado de jogo de forma imersiva ao jogador.
\end{itemize}

A maioria dos jogos \ac{MMORPG} disponíveis no mercado estão implementados sobre uma arquitetura que executa sobre diversos servidores\cite{stephenclarkewillson2017}, nos quais o desempenho destes servidores influencia tanto na experiência de jogabilidade do usuário final, quanto no custo de manutenção destes serviços~\cite{1417630}.
%
Em especial, o presente trabalho tratará com maiores detalhes as arquiteturas utilizadas no serviço dessa categoria de jogos.

\section{Trabalhos Relacionados}
\label{sec:similares}
