\chapter{Fundamentação Teórica}
\label{cap2}

O conceito de arquitetura de microsserviços para jogos \ac{MMORPG} (Seção \ref{sec:arquitetura_de_microsserviços}), embora com a popularidade dos títulos da categoria de jogo abordado pelo presente documento, normalmente possui as suas especificações desconhecidas para grande público.
%
Tais arquiteturas tem uma evolução tecnológica significativa compando jogos de 1990 e jogos de 2010.
%
Nesse sentido, torna-se necessário entender as arquiteturas utilizadas e propostas por engenheiros de software de tais jogos ou pela literatura atual visando conhecer tais conceitos e funcionamento básico, do ponto de vista de sistemas distribuídos.



Jogos \ac{MMORPG} criados durante 2010 e 2018 utilizam arquiteturas de microsserviços (Seção \ref{sec:microsserviços_de_mmorpg}) para abstrair regras de negócio em uma aplicação distribuída.
%
Dessa forma, tendem a suportar um número maior de conexões simultâneas comparado a arquiteturas cliente-servidor com uma boa qualidade de jogabilidade aos usuários finais, caso sejam bem projetados.
%
O processo de análise e detalhamento destas arquiteturas (Seção \ref{sec:arquiteturas}) é imporante para compreender o comportamento do serviço sobre uma determinada carga de conexões.



Após realizar o detalhamento de tais arquiteturas, torna-se necessário apresentar trabalhos com objetivos similares (Seção \ref{sec:similares}),
%
exibindo exemplos de métodos, métricas e ferramentas utilizadas.



\section{Contextualização de microsserviços}
\label{sec:arquitetura_de_microsserviços}



Arquiteturas de microsserviços é uma tendência de grandes empresas para resolver problemas de escalabilidade, flexibilidade, desempenho e gerenciamento em aplicações web.
%
Empresas como Linkedin, Netflix e Amazon utilizam essas arquiteturas para prover serviços a uma quantia massiva de clientes~\cite{8169955}.



A partir de 2013 houve um aumento na procura por arquiteturas de microsserviços~\cite{google_trends:2018}.
%
Esse aumento na busca é dado pelo interesse de desenvolver serviços que suportem um grande número de conexões, facilite o seu desenvolvimento e a sua manutenção.
%
O aumento na procura deste novo paradigma pode ser visualizado na figura \ref{fig:trends} pelo número de pequisas pelo termo \textit{microservices} no buscador Google\footnote{Google: \url{https://www.google.com}.}.



\begin{figure}[htb!]
  \caption{Gráfico de tendências de pesquisa do Google pelo termo \textit{microservices}}
  \label{fig:trends}
  \includegraphics[height=3cm]{img/microsserviços_trends.png}
  \centering

  Fonte:~\cite{google_trends:2018}
\end{figure}


Dentre este tópico de busca, está relacionado alguns tópicos como Programação de Interfaces de Aplicação, Virtualização (Docker), Computação em Nuvem e Arquiteturas Orientadas a Serviço~\ref{fig:trends}.
%
Esses também estrão presentes no presente trabalho, visto que existe uma correlação desses tópicos e a literatura.

Para entender os problemas a qual arquiteturas de microserviços comprometem-se a resolver, se faz necessário entender a contextualização de problemas anteriores a esse modelo de arquitetura para serviços.

\subsection{Arquiteturas de Microsserviços}

O processo de desenvolvimento de software ágil tem como propósito produzir um processo de construção flexível a necessidade do cliente,
%
onde cada entrega está sujeita a ajustes~\cite{joseph_cooper, 8169955}.
%
As arquiteturas baseadas em microsserviços é um novo paradigma a qual propõe que os sistemas devam estar fragmentados para garantir melhor disponibilidade.
%
Além disso, contribuem com a manutenção, escalabilidade, modificações e um melhor uso dos recursos computacionais~\cite{newman2015building, 8169955}.
%
Essa arquitetura facilita o processo de desenvolvimento ágil, a qual detem um grande espaço sobre empresas de desenvolvimento de jogos.

Arquiteturas monolíticas utilizadas anteriormente aposta em uma forma ingênua de desenvolvimento, focando na forma rápida e prática de desenvolvimento de software,
%
utilizando alguma infraestrutura de rede simplificada~\cite{joseph_cooper}.
%
Em contra partida, esse método de desenvolvimento não é escalável para times maiores e aplicações que tenham um grande volume de acesso.
%
A resolução de conflitos no desenvolvimento dessa arquitetura é custosa e o sistema estará preso ao número máximo de conexões.
%
Por esse motivo, a utilização de arquitetura de microsserviços começa a aumentar conforme a sua popularidade vista na figura \ref{fig:trends}.

\subsection{MMORPG}
% conceito
% classificação

\section{Microsserviços em MMORPG}
\label{sec:microsserviços_de_mmorpg}

\section{Arquiteturas}
\label{sec:arquiteturas}

% Classificação
% Protocolos, otimizações

\section{Trabalhos Relacionados}
\label{sec:similares}
