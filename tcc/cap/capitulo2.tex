\chapter{Fundamentação Teórica}
\label{cap2}

\section{Jogos Eletrônicos}

O primeiro sistema de entretenimento interativo foi construido em 1947, utilizando como base de exibição um tubo de raios catódicos, criado por Thomas Goldsmith Jr. e Estle Ray Mann~\cite{Adams2014Jan}.
%
Essa criação foi patenteada em janeiro de 1948~\cite{patents1947Jan}, datando então o inicio dos jogos eletrônicos.



O termo jogo significa gracejo, brincadeira, divertimento~\cite{Adams2014Jan}.
%
O jogo eletrônico é uma atividade intelectual que integra um sistema de regras, na qual utiliza tal sistema a fim de definir seus objetivos ou pontuação por meio de um computador~\cite{video_game_technologies}.
%
Como conceito formal, um jogo eletrônico é um jogo controlado por computador~\cite{Adams2014Jan}.
%
O termo \textit{entretenimento interativo} é utilizado para se referenciar a qualquer jogo eletrônico que execute sobre alguma plataforma~\cite{video_game_technologies}.
%
Uma \textit{plataforma} é um computador a qual executa as regras de negócio e exibem o jogo de forma visual ao jogador~\cite{video_game_technologies} por meio de algum dispositivo de retorno.



Por essa definição, um videogame é uma plataforma na qual a televisão é o dispositivo de retorno primário, utilizando algum hardware proprietário para executar o jogo eletrônico~\cite{video_game_technologies}.
%
Além de jogos para videogames, também existem jogos para computadores pessoais.
%
Um jogo de computador é um jogo específico para computadores pessoais, a qual executará sobre um sistema operacional de propósito geral.
%
Dessa forma, computadores pessoais também são considerados plataformas, porém categorizados pelo seu sistema operacional~\cite{video_game_technologies}.
%
Existem outras diversas plataformas, \textit{e.g.,} web, realidade virtual, dispositivos móveis em geral, etc


Além de sua categoria por plataforma, os jogos eletrônicos podem ser classificados conforme o seu \textit{gênero}.
%
O gênero de um jogo é classificado conforme um estilo comum ou um conjunto de características (\textit{e.g.,} perspectiva, estilo de jogo, interação, objetivo, etc)~\cite{video_game_technologies}.



\subsection{Árvore de categoria de jogos}
\label{sec:arvore_de_categoria_de_jogos}



Um grande problema classificação por gênero é a arbitrariedade e consistência.
%
É comum um jogo eletrônico ter características de dois grandes grupos.
%
Nesses casos ele pode dividir seu nome dentre os dois gêneros ou gerar um novo gênero~\cite{video_game_technologies} dentro da árvore de categorização de jogos (Seção ~\ref{sec:arvore_de_categoria_de_jogos}).



A árvore de categoria de jogos descreve grupo de jogos que são semelhantes por características específicas a cada jogo~\cite{video_game_technologies, Adams2014Jan}.
%
Os gêneros não são definidos pelo conteúdo do jogo, mas pelo desafio que será proporcionado ao jogador~\cite{Adams2014Jan}.
%
Tal árvore é importante para melhor classificar eventuais títulos presentes no mercado para uma visão macro de seu sistema computacional~\cite{video_game_technologies}.
%
É válido ressaltar que mesmo jogos parecidos podem obter abordagens diferentes em seu desenvolvimento, seja por variações de hardware, software, tempo e recurso de investimento para seu desenvolvimento~\cite{Adams2014Jan}.
%
É válido ressaltar que um título também não precisa estar em uma única categoria dessa árvore~\cite{video_game_technologies}.
%
A árvore de categoria de jogos descrita abaixo mostra de forma macro as principais categorias de jogos e onde encontra-se a categoria de jogos do atual trabalho~\cite{video_game_technologies}.

\begin{itemize}
  \item \textbf{Aventura}: Essa categoria de jogos aborda a exploração e resolução de problemas lógicos básicos baseados em algum roteiro. O primeiro jogo dessa categoria é Zork\footnote{Zork: \url{https://github.com/devshane/zork}}, lançado nos anos 1980.
  \item \textbf{Ação}: Essa categoria de jogos utiliza a coordenação motora do jogador para instigar desafios a serem completados. Um título famoso é o jogo Space Invaders\footnote{Space Invaders: \url{https://github.com/dwmkerr/spaceinvaders}}.
  \item \textbf{Ação e Aventura}: Essa categoria de jogos mescla as categorias de Aventura e Ação. Seu objetivo é desvendar problemas lógicos com relação ao ambiente enquando precisa de destreza para movimentação pelo ambiente. Um exemplo é a série de jogos Legend of Zelda\footnote{Legend of Zelda: \url{https://www.zelda.com/}}.
  \item \textbf{Estratégia em tempo real ou \ac{RTS}}: Utiliza o gerenciamento de recursos obtidos de forma estratégia do jogo para gerar dificuldade. Assim como jogos de luta, a sua popularidade é maior entre jogos onde permite a competição entre jogadores online StarCraft\footnote{StartCraft: \url{https://starcraft2.com/en-us/}}.
  \item \textbf{Estratégia em turnos}: Essa categoria de jogos utiliza estratégia de movimentos unitários e atômicos para ganhar do oponente. Xadrez é um exemplo de jogo não computacional. Final Fantasy\footnote{Final Fantasy: \url{https://www.finalfantasy.com/}} é um exemplo de jogo neste segmento.
  \item \textbf{Jogos de interpretação ou \ac{RPG}}: The Elder Scrolls V\footnote{The Elder Scrools V: \url{http://store.steampowered.com/agecheck/app/489830/}}
  \item \textbf{Jogos Massivos ou \ac{MMO}}
    \begin{itemize}
      \item \textbf{\ac{MMORPG}}: Black Desert Online\footnote{Black Desert Online: \url{https://blackdesert.playredfox.com/black_desert}}
      \item \textbf{\ac{MOBA}}: DOTA 2\footnote{DOTA 2: \url{http://br.dota2.com/}}
      \item \textbf{\ac{MMOFPS}}:Battlefield Battlelog\footnote{Battlefield Battlelog: \url{http://battlelog.battlefield.com/}}
    \end{itemize}
\end{itemize}



Existem


\section{Trabalhos Relacionados}
\label{sec:similares}
