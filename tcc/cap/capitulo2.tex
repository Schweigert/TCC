\chapter{Fundamentação Teórica}
\label{cap2}


O termo \textit{jogos eletrônicos} é amplamente difundido, entretanto as especificações, características e histórico deste termo não são de conhecimento popular.
%
A Seção~\ref{sec:jogos_eletronicos} trata a definição de jogo eletrônico, um levante histórico e o impacto da evolução do hardware no desenvolvimento dos jogos.
%
Este termo é tomado como introdução para o conceito de gênero de jogo, abordado na Subseção~\ref{sec:arvore_generos}, a qual referencia os principais gêneros, características e tecnologias (do ponto de vista de rede de computadores) que são comuns em cada gênero.
%
Esta introdução acerca dos gêneros e suas tecnologias de comunicação busca trazer a importância do desempenho das arquiteturas dos jogos \ac{mmorpg} e proporção da comunidade impactada caso hajam falhas de funcionamento em tais arquiteturas.



Após definir a categoria de jogo abordado, a Seção~\ref{sec:mmorpg} apresenta uma introdução ao impacto de mercado desse gênero, uma definição simplista do gênero e a divisão das camadas de aplicação que permeiam uma arquitetura para um jogo \ac{mmorpg}.
%
Antes de abordar sobre as camadas da infraestrutura de uma arquitetura de jogo \ac{mmorpg} e seus problemas recorrentes relativos a rede (Seção~\ref{sec:problemas}), faz-se obrigatório o entendimento sobre a jogabilidade deste gênero (Seção~\ref{sec:jogabilidade}).


Os conceitos de cliente (Seção~\ref{sec:cliente}) e serviço (Seção~\ref{sec:microsservicos}) são abordados a fim de introduzir conceitos básicos de arquiteturas para jogos \ac{mmorpg}.
%
O objetivo destas seções é referenciar diversas tecnologias e técnicas utilizadas nesses sistemas a fim de permitir o desenvolvimento de uma arquitetura de microsserviços específica a jogos \ac{mmorpg}.
%
Por fim, torna-se obrigatório a apresentação de trabalhos relacionados (Seção~\ref{sec:similares}) a arquitetura de jogos \ac{mmorpg} desenvolvidos de forma distribuída e/ou sobre uma arquitetura de microsserviços.
%
Esta seção em específico aborda exemplos de métodos e métricas utilizadas para mensurar o desempenho de tais arquiteturas, realizando por fim uma análise destes trabalhos (Subseção~\ref{sec:similares_analise}).


\section{Jogos Eletrônicos}
\label{sec:jogos_eletronicos}


O primeiro sistema de entretenimento interativo foi construído em 1947, utilizando como base de exibição um tubo de raios catódicos.
%
Essa criação foi patenteada em janeiro de 1948, datando então o início dos jogos eletrônicos~\cite{Adams2014Jan, patents1947Jan}.



O jogo eletrônico, ou entretenimento interativo, é uma atividade intelectual que integra um sistema de regras, na qual utiliza tal sistema a fim de definir seus objetivos ou pontuação por meio de um computador, com o objetivo de despertar alguma emoção ao jogador~\cite{video_game_technologies}.
%
Os jogos eletrônicos são aplicações convencionais, que executam sobre algum sistema operacional ou hardware apropriado a este fim.
%
O sistema operacional, hardware ou base de execução da aplicação gráfica define a sua plataforma (\textit{e.g.,} GNU/Linux, MS-Windows, Sony PS4, MS-XBox, web, etc.)~\cite{adams_1208533}.



Inicialmente, os jogos eram implementados de forma simples por conta da limitação de hardware das plataformas da década de 80.
%
As implementações de jogos para \textit{videogames} eram projetadas diretamente para algum hardware proprietário, sem sistema operacional, por muitas vezes sem utilizar comunicação por rede ou armazenamento em memória secundária~\cite{rollings2003andrew}.
%
Além de diversas plataformas não terem acesso a rede, os serviços para jogos eram inviabilizados pelo custo de manutenção e pela ausência de demanda na qual teriam os requisitos mínimos para jogar~\cite{adams_1208533}.
%
Na década de 80, o \textit{videogame} Atari foi uma plataforma popular, vendendo 30.000 unidades em seu lançamento contra apenas 2.000 unidades do seu concorrente Intellivision~\cite{atari_age}.



A crescente de recursos computacionais disponíveis em computadores pessoais e \textit{videogames} após os anos 90, permitiu que desenvolvedores criassem novos estilos de jogos que utilizavam de hardware mais específico~\cite{adams_1208533}.
%
Dentre esses recursos, iniciou-se o uso da rede de computadores para proporcionar a interação entre jogadores em equipamentos distintos~\cite{statisita_consumo_rede}.
%
Jogos como EA Habitat\footnote{EA Habitat: \url{http://www.mobygames.com/game/c64/habitat/credits}}, CipSoft Tibia\footnote{CipSoft Tibia: \url{http://www.tibia.com/}} e Jajex Runescape\footnote{Jajex Runescape: \url{https://www.runescape.com}} começaram a utilizar, como requisito obrigatório do jogo, a conexão com a Internet para interagir em um mundo compartilhado com outros jogadores.
%
Tais jogos popularizaram um novo gênero, trazendo inovação tecnológica como complemento a sua jogabilidade, propondo novos desafios aos jogadores ao jogar com centenas ou milhares de jogadores simultâneos~\cite{guinness_runescape, 1417630}, criando o gênero de jogos \ac{mmo}.

Muitos outros jogos do gênero foram criados devido ao aumento da quantidade de público alvo e viabilidade de novas tecnologias, na qual fomentaram a evolução dos jogos \ac{mmorpg}.
%
Nesse sentido, as redes de computadores realizaram papeis de impulsionadores para várias categorias de jogos, que antes não eram possíveis por conta da limitação de comunicação entre computadores.
%
Sendo assim, torna-se necessário ter uma visão geral das principais categorias de jogos eletrônicos com relação as tecnologias empregadas do ponto de vista de redes de computadores.



\subsection{Árvore de gêneros de jogos eletrônicos}
\label{sec:arvore_generos}


A classificação por gênero é uma ferramenta tradicional para auxiliar a fácil identificação de características de alguma literatura, arte e outras mídias.
%
Dentro de jogos eletrônicos, o gênero permite que jogadores busquem jogos com características conforme o seu interesse comum~\cite{Clarke2015}.
%
Uma árvore de gêneros para jogos eletrônicos pode ser visualizada na Figura~\ref{fig:generos}.


\begin{figure}[htb!]
\caption{Árvore de gêneros de jogos eletrônicos simplificada.}
\label{fig:generos}
\includegraphics[width=\textwidth]{img/cap2/generos.png}
\centering

Adaptado de:~\cite{adams_1208533}
\end{figure}



Um gênero de jogo eletrônico é uma categoria específica para agrupar estilos de jogabilidade parecidos.
%
Porém, um gênero não define de forma explícita o conteúdo expresso em algum jogo eletrônico, mas sim um desafio comum presente no jogo analisado~\cite{adams_1208533, video_game_technologies}.
%
Na Figura~\ref{fig:generos} foram apresentados os principais gêneros, na qual podem ser brevemente descritos como:


\begin{itemize}
  \item Estratégia: São focados em uma jogabilidade que exija habilidades de raciocínio e/ou gerenciamento de recurso do jogo. Neste gênero, o jogador tem uma boa visualização do mundo, controlando indiretamente as suas tropas disponíveis~\cite{rollings2003andrew}. É comum encontrar jogos que disponibilizam algum modo de competição entre jogadores usando \ac{lan}, \ac{wan} ou \ac{p2p}~\cite{adams_1208533}.
    \begin{itemize}
      \item \ac{rts}: Utiliza as características de um jogo de estratégia, porém esse subgênero indica que as ações dos jogadores são concorrentes. É comum encontrar modos de jogo competitivo utilizando \ac{lan} neste gênero~\cite{adams_1208533}.
    \end{itemize}
  \item \ac{mmo}: Preza pela interação com outros jogadores em um mundo compartilhado~\cite{adams_1208533}. SecondLife\footnote{SecondLife: \url{https://www.secondlife.com/}} é um jogo focado na interação social, com artifícios de comércio e relacionamentos em um mundo fictício criado pela comunidade~\cite{tecmundo_secondlife}. Em grande parte, esses jogos utilizam tecnologia \ac{wan} e \ac{cs}.
    \begin{itemize}
      \item \ac{moba}: Coloca um número fixo de jogadores separados em dois times, no qual o time com melhor estratégia de posicionamento e gerenciamento de recursos em equipe ganha a partida. Jogos \ac{moba} perdem algumas características breves do gênero \ac{rpg}, deixando de lado a interpretação e contextualização de um mundo, fixando-se somente em um combate estratégico e momentâneo (distribuído em partidas atômicas) entre as equipes, carregando consigo somente as características de comércio e comunidade dos jogos \ac{mmo}~\cite{adams_1208533}. Tal subgênero é popularmente conhecido pelos títulos Blizzard Dota 2\footnote{Blizzard Dota 2: \url{http://br.dota2.com/}} e Riot League of Legends\footnote{Riot League of Legends: \url{https://br.leagueoflegends.com/pt/}}. O jogo League of Legends obteve 100 milhões de usuários ativos em 2016 sendo o jogo mais jogado do mundo neste ano~\cite{lol_statista}, tendo torneios nacionais e internacionais~\cite{lol_sportv}. É popular nesse subgênero utilizar tecnologias como \ac{lan}, \ac{p2p} e \ac{wan}.
      \item \ac{mmorpg}: Herda características dos gêneros ação e aventura, \ac{rpg}, e \ac{mmo} diretamente. Nesse gênero é permitido interações em um mundo compartilhado aos jogadores, na qual a interação entre outros jogadores (herdado dos jogos \ac{mmo}), com o mundo (herdado dos jogos de ação e aventura) e com objetivos guiados por \acp{npc} (herdados de jogos \ac{rpg}) se faz como desafio e objetivo do jogo~\cite{adams_1208533}. Um título popular para esse gênero é o jogo Blizzard Word of WarCraft, o qual tem o título de maior comunidade de jogo em um único serviço do mundo\footnote{Blizzard Word of WarCraft: \url{https://worldofwarcraft.com/pt-br/}}. A grande parte dos jogos \ac{mmorpg} utilizam tecnologia \ac{wan} e \ac{cs}.
    \end{itemize}
  \item Aventura: Caracterizado por desafios envolvendo ações com diversos \acp{npc} ou com o ambiente a fim de solucionar desafios~\cite{adams_1208533}. A grande parte desses jogos utilizam arquiteturas \ac{wan}, \ac{p2p} ou \ac{lan}.
    \begin{itemize}
      \item Ação e Aventura: Herda características da categoria de Aventura. O jogador é imerso em um mundo para interagir com o ambiente e com \acp{npc}, além de se preocupar com a movimentação no cenário~\cite{adams_1208533}. Um título popularmente conhecido desse gênero é a série de jogos nomeada Nintendo The Legend of Zelda\footnote{Nintendo The Legend of Zelda: \url{https://www.zelda.com/}}. É comum nesses jogos encontrar tecnologia \ac{lan} ou \ac{p2p} para modo de jogo cooperativo.
    \end{itemize}
  \item Simulação: Caracterizados por abordar temas da realidade. São comuns jogos de construção e gerenciamento, animais de estimação, vida social e simulação de veículos~\cite{adams_1208533}. A grande parte desses jogos não permite a interação entre os demais jogadores. É popular encontrar serviços como \textit{ranking}, loja e janela de notícias utilizando \ac{cs}.
    \begin{itemize}
      \item Esportes: Trata somente da simulação de esportes, nos quais os times podem ser controlados tanto por uma inteligência artificial quanto por jogadores \textit{online}~\cite{adams_1208533}. O jogo FIFA\footnote{FIFA: \url{https://www.easports.com/br/fifa}} é um título popular nesse segmento. É comum encontrar tecnologias \ac{p2p} e \ac{lan}.
    \end{itemize}
  \item Ação: Preza pela habilidade de coordenação motora e reflexos do jogador, para tomar uma ação a fim de superar seus desafios no cenário alcançando algum objetivo~\cite{adams_1208533}. É comum encontrar tecnologias \ac{lan}, \ac{p2p}, \ac{wan} e \ac{cs}.
    \begin{itemize}
      \item Jogos de Tiro: Usa um número finito de armas para executar ações a distância. O posicionamento, movimentação estratégia e mira são fatores de desafio ao jogador nesse gênero~\cite{adams_1208533}. É comum encontrar tecnologias \ac{lan}, \ac{p2p} ou \ac{wan}.
        \begin{itemize}
          \item \ac{fps}: Utiliza o método de gravação conhecido como \ac{pov}. Nesse método, o modo de exibição do mundo é dado como a visão de um personagem do jogo, na qual o jogador tem visão pelo próprio personagem~\cite{video_game_technologies, adams_1208533}. É comum encontrar tecnologias \ac{lan}, \ac{p2p} ou \ac{wan}.
          \item \ac{tps}: Diferente dos jogos \ac{fps}, os jogos \ac{tps} utilizam câmeras soltas no cenário no qual o jogador é visível na cena exibida~\cite{video_game_technologies, adams_1208533}. É comum encontrar tecnologias \ac{lan}, \ac{p2p} ou \ac{wan}.
        \end{itemize}
    \end{itemize}
  \item Jogos sérios: Tem como objetivo transmitir um conteúdo educacional~\cite{video_game_technologies}. O jogo Sherlock Dengue 8~\cite{sherlock_dengue} é um título desenvolvido com o objetivo de conscientizar os problemas e a prevenção da Dengue no Brasil. É comum encontrar tecnologias \ac{lan}, \ac{p2p}, \ac{wan} e \ac{cs}.
\end{itemize}

A árvore de gêneros guia tanto os usuários finais para classificar jogos que lhe agradem, quanto desenvolvedores a fim de seguir tendências de mercado pelas características do gênero.
%
Dessa forma, encontra-se um padrão nos jogos, o qual inclusive orienta o desenvolvimento das arquiteturas de tais jogos~\cite{video_game_technologies}.

Dentre vários gêneros, alguns utilizam popularmente algumas tecnologias de rede.
%
A Tabela~\ref{tab:comunicacao_genero} indica a correlação de tecnologias de rede comuns nos gêneros, além de trazer a correlação de número de jogadores por gênero de jogo em seus serviços.
%
Essa correlação é importante para identificar as características referentes a jogabilidade com multijogadores~\cite{video_game_technologies}.


\begin{table}[htb!]
\centering
\caption{Tipos de comunicação e quantidade de jogadores impactados por ocorrências conforme o seu gênero.}
\label{tab:comunicacao_genero}
\begin{tabular}{l|l|l|l|l|l}
\hline \hline
                & \ac{lan}   & \ac{wan}   & \ac{p2p}    & \ac{cs}  &  Jogadores Impactados por falhas                     \\ \hline \hline
ESTRATÉGIA      & \checkmark & \checkmark & \checkmark &              &   até 10~\cite{eoe3}                              \\ \hline
RTS             & \checkmark & \checkmark &            & \checkmark   &   até 10~\cite{starcraft2}                        \\ \hline
MOBA            & \checkmark & \checkmark & \checkmark & \checkmark   &   até 10~\cite{lol_how_work_games}                \\ \hline
MMO             &            & \checkmark &            & \checkmark   &   mais que 1000~\cite{runescape_online_users}     \\ \hline
MMORPG          &            & \checkmark &            & \checkmark   &   mais que 1000~\cite{runescape_online_users}     \\ \hline
AVENTURA        & \checkmark & \checkmark & \checkmark & \checkmark   &   até 100~\cite{minecraft}                        \\ \hline
AÇÃO            & \checkmark & \checkmark & \checkmark & \checkmark   &   até 10~\cite{cuphead}                           \\ \hline
AÇÃO E AVENTURA & \checkmark & \checkmark & \checkmark & \checkmark   &   até 10~\cite{cuphead}                           \\ \hline
SIMULAÇÃO       & \checkmark & \checkmark & \checkmark & \checkmark   &   até 10~\cite{eurotruck2}                        \\ \hline
ESPORTES        & \checkmark & \checkmark & \checkmark & \checkmark   &   até 10~\cite{fifa2018}                          \\ \hline
FPS             & \checkmark & \checkmark & \checkmark & \checkmark   &   até 100~\cite{battlefield3}                     \\ \hline
TPS             & \checkmark & \checkmark & \checkmark & \checkmark   &   até 100~\cite{battlefield3}                     \\ \hline
JOGOS SÉRIOS    & \checkmark & \checkmark & \checkmark & \checkmark   &   até 10~\cite{sherlock_dengue}                   \\ \hline \hline
\end{tabular}

Fonte: O próprio autor.
\end{table}


Dentre todos os jogos, o gênero \ac{mmorpg} é o mais impactado pela quantidade de jogadores\cite{mmo_analytic}, visível na Tabela~\ref{tab:comunicacao_genero}.
%
Nesse sentido, as arquiteturas do serviço e cliente se tornam um ponto crítico no desenvolvimento a fim de suportar a carga necessária ao desenho do jogo.
%
Por esse motivo, a escolha por abordar o gênero \ac{mmorpg} se torna interessante do ponto de vista computacional, visto que o impacto de falhas em tais serviços afeta um vasto número de jogadores, comparado com outros gêneros.



\section{\ac{mmorpg}}
\label{sec:mmorpg}



Jogos \ac{mmorpg} são utilizados como negócio viável e lucrativo, sendo que a experiência de jogabilidade na qual o usuário final será submetido é um fator crítico para o sucesso dos títulos deste gênero.
%
O mercado de jogos \ac{mmorpg} vem crescendo desde 2012~\cite{new_york_times}, sendo no ano de 2017 um dos mais lucrativos~\cite{statista_2018_mmo}.
%
A projeção deste mercado para 2018 era de mais de 30 bilhões de dólares americanos em circulação sobre esta categoria de jogos~\cite{statista_2018}, porém este valor foi ultrapassado com 30,7 bilhões de dólares~\cite{statista_2018_mmo}.



\ac{mmorpg} são jogos de interpretação de papéis massivos, originados ao unir  os gêneros \ac{rpg} e \ac{mmo}.
%
A principal característica desse estilo de jogo é a comunicação e representação virtual de um mundo fantasia, no qual cada jogador pode interagir com objetos virtuais compartilhados ou tomar ações sobre outros jogadores em tempo real, tendo como principais objetivos a resolução de problemas conforme a sua regra de negócio, o desenvolvimento do personagem e a interação entre os jogadores~\cite{video_game_technologies}.



Um jogo \ac{mmorpg} é arquitetado em três partes~\cite{mmo_analytic}:
\begin{itemize}
  \item \textbf{Cliente}: Aplicação que realiza as requisições com a interface do serviço, exibindo o estado de jogo de forma imersiva ao jogador. Este tema é melhor abordado na Subseção \ref{sec:cliente}.
  \item \textbf{Servidor}: O computador, ou o conjunto de computadores, que recebe as requisições do cliente a fim de serem processadas pelo Serviço.
  \item \textbf{Serviço}: Implementa as regras de negócio e requisitos do jogo. O serviço disponibiliza uma interface com ações possíveis ao cliente sobre algum protocolo de rede. Este tema é abordado na Subseção~\ref{sec:microsservicos}.
\end{itemize}

Jogos \ac{mmorpg} trabalham como serviço.
%
Por este motivo, o modelo de negócios de tais jogos, do ponto de vista de redes de computadores, é suscetível a perda financeira em casos de negação de serviço~\cite{1417630}.
%
A maioria dos jogos \ac{mmorpg} disponíveis no mercado estão implementados sobre uma arquitetura que executa sobre diversos servidores~\cite{stephenclarkewillson2017}, nos quais o desempenho destes servidores influencia tanto na experiência de jogabilidade do usuário final, quanto no custo de manutenção destes serviços~\cite{1417630}.
%
Por sua vez, o Cliente é implementado em algum ambiente convencional a jogos, como motores gráficos, bibliotecas gráficas ou sobre alguma outra plataforma, como web.
%
Nesse sentido, torna-se necessário descrever as características de jogabilidade de jogos \ac{mmorpg}, de forma genérica, a fim de melhor compreender o funcionamento da arquitetura de um cliente e de um serviço para jogos \ac{mmorpg}.



\section{Jogabilidade de jogos MMORPG}
\label{sec:jogabilidade}



É comum serviços \ac{mmorpg} terem regras de negócio parecidas, visto que pertencem ao mesmo gênero.
%
Explicar algumas regras de negócio recorrentes neste gênero facilita a compreensão básica de forma genérica de um jogo \ac{mmorpg} e auxilia a compreensão do modelo computacional implementado nestes serviços.
%
Deste modo, torna-se necessário definir algumas funcionalidades básicas que estão dentro do contexto de jogabilidade de jogos \ac{mmorpg} para melhor compreensão de sua arquitetura em seções futuras.


O sistema de autenticação é o sistema que geralmente inicia o cliente de algum jogo \ac{mmorpg}~\cite{albion_online_unite, matthiasrudy2011}.
%
Este sistema é implementado via protocolo \ac{http} ou \ac{rpc}, a fim de disponibilizar um código para validar todas as futuras ações da seção do usuário.
%
Este sistema pode ser visualizado de forma macro na Figura~\ref{fig:autenticacao}.

\begin{figure}[htb!]
\caption{Sistema de autenticação para jogos \ac{mmorpg}.}
\label{fig:autenticacao}
\includegraphics[height=5cm]{img/cap2/autenticacao.png}
\centering

Fonte: Adaptado de ~\cite{LeckyThompson2008Nov}
\end{figure}


O sistema de autenticação visualizado na Figura~\ref{fig:autenticacao} é popularmente conhecido, porém não o único.
%
O jogo Realm of the Mad God\footnote{Realm of the Mad God: \url{https://www.realmofthemadgod.com/}} é um exemplo na qual não exige autenticação do usuário, entretanto ele não armazena em seu banco de dados o progresso do jogador ou permite açõe sociais caso o jogador não efetue a autenticação.


Um método de autenticação popular é definido pela RFC7519~\cite{rfc7519}, com a tecnologia \ac{jwt}.
%
O código de validação repassado é auditorado em qualquer serviço pertencente ao jogo, visto que ele foi assinado pelo sistema de autenticação do serviço~\cite{Ikem2018May}.


Após a autenticação, é comum existir um sistema para seleção de personagem, caso o jogo seja desenhado com este objetivo.
%
Efetuada a seleção ou criação de um personagem, o jogador será imerso no mundo compartilhado do jogo com os demais jogadores, na qual poderá controlar somente o seu personagem~\cite{matthiasrudy2011}.
%
Nos jogos \ac{mmorpg} é comum a restrição da visão do personagem (Figura~\ref{fig:proximidade}), ora pelas características de jogabilidade do gênero \ac{mmorpg} ora por motivos de desempenho e otimização.
%
Como o jogador não precisa obter dados de regiões que não estão em sua área de interesse, visto que o mesmo só poderá controlar o personagem selecionado, dessa forma não há necessidade da transmissão de informações dos objetos que estão fora do contexto do personagem selecionado~\cite{albion_online_unite}.

\begin{figure}[htb!]
\caption{Área de interesse com base na proximidade de um jogador.}
\label{fig:proximidade}
\includegraphics[width=\textwidth]{img/cap2/proximidade.png}
\centering

Fonte:~\cite{albion_online_unite}
\end{figure}


Um exemplo de área de interesse pode ser visualizado na Figura~\ref{fig:proximidade}, na qual o personagem selecionado (destacado em vermelho) tem uma área de interesse de um raio pré definido (destacado em azul)~\cite{albion_online_unite}, sendo que o jogador não tem informações dos demais objetos e jogadores fora de sua área de interesse a nível de rede (destacado em amarelo).
%
Esta característica impede trapaças, visto que o cliente tem informações que só estão contidas em sua área de interesse, consecutivamente reduzindo a carga de rede necessária para o funcionamento do cliente~\cite{albion_online_unite}.


Após o jogador estar com o controle do personagem no ambiente, é comum em tais jogos que ele possa realizar determinadas ações de interação com o mundo e/ou social.
%
Nesse sentido, algumas ações comuns dentro do ambiente de um jogo \ac{mmorpg}~\cite{mmorpg_culture} são:

\begin{itemize}
  \item Enviar e receber mensagem no \textit{chat};
  \item Mover-se pelo ambiente;
  \item Interagir com outros jogadores, \acp{npc} ou objetos fixos do ambiente;
  \item Obter itens do ambiente; e
  \item Interagir com outros sistemas conforme algum objetivo do jogo em específico;
\end{itemize}



O envio e recepção de mensagens do \textit{chat} é dado com o contexto do posicionamento do personagem~\cite{albion_online_unite}, visível na Figura~\ref{fig:chat}.
%
Somente outros personagens dentro de um raio podem receber alguma mensagem emitida pelo jogador $P_n$.

\begin{figure}[htb!]
\caption{\textit{Chat} baseado em contexto de posicionamento, utilizando Distância Euclidiana.}
\label{fig:chat}
\includegraphics[height=6cm]{img/cap2/chat.png}
\centering

Fonte: Adaptado de ~\cite{albion_online_unite}
\end{figure}

Essa distância (Figura~\ref{fig:chat}) pode ser calculada utilizando o método de distância euclidiana~\cite{Deza2009Aug}, na qual a distância entre dois personagens pode ser calculada pela equação $d(p, q) = \sqrt{\sum_{i=1}^{n}(q_i - p_i)^2}$.
%
Para diminuir a complexidade das comparações, a fim de decidir quais personagens $P_n$ devem receber a mensagem, é comum utilizar técnicas de divisão de área utilizando algoritmos como \textit{Quadtree} ou \textit{Octree}~\cite{Lengyel2011Jun}, subdividindo os quadrantes de uma região do ambiente do jogo a fim de facilitar a consulta de quais personagens estão em determinada área deste ambiente.


Nesse sentido, a Figura~\ref{fig:chat} mostra a interseção entre o raio de quatro personagens.
%
Nesse exemplo, mostra-se visível que as mensagens de $P_1$ devem ser visíveis a $P_2$ e $P_3$, mas não a $P_4$, caso seja utilizado a distância euclidiana como regra de distância.

Existem outros sistemas de mensageria que podem ser utilizados, como a divisão de regiões, geralmente quadradas, na qual o jogador emissor de alguma mensagem encaminhará a todos os jogadores dentro desta região.
%
Tal sistema tende a consumir menos recurso, porém pode ter impacto direto sobre a regra de negócio do jogo~\cite{albion_online_unite}.



O sistema de movimento pelo ambiente do jogo possibilita que cada jogador movimente seu personagem pelo ambiente a fim de explorá-lo.
%
Dessa maneira, este é um sistema crítico para um jogo \ac{mmorpg}, visto que o posicionamento de objetos é utilizado para diversas consultas de proximidade, além de necessitar uma frequência de atualização constante do posicionamento dos jogadores a fim de manter a integridade de funcionamento do serviço de ambiente do jogo~\cite{albion_online_unite}.
%
Um exemplo de movimentação no ambiente pode ser visualizado na Figura~\ref{fig:walk}.

\begin{figure}[htb!]
\caption{Personagens e os seus pontos de destino.}
\label{fig:walk}
\includegraphics[height=4cm]{img/cap2/walk.png}
\centering

Fonte: O próprio autor.
\end{figure}


Com a movimentação descrita na Figura~\ref{fig:walk}, o personagem torna-se livre a explorar o ambiente seguindo as regras de negócio do jogo, permitindo a interação com o ambiente, objetos posicionados no cenário ou outros jogadores.
%
Dessa forma, a interação com estes elementos também é afetada pela área de interesse e a sua movimentação, na qual o personagem terá um raio limitante para cada tipo de interação no ambiente.
%
Um exemplo de ambiente com personagens ($P_1$ e $P_2$), objetos ($O_1$) e \acp{npc} (NPC somente) pode ser visualizado na Figura~\ref{fig:obj_e_npc1}~\cite{albion_online_unite}.

\begin{figure}[htb!]
\caption{Personagens, objetos e \acp{npc} no ambiente.}
\label{fig:obj_e_npc1}
\includegraphics[height=5cm]{img/cap2/obj_e_npc1.png}
\centering

Fonte: O próprio autor.
\end{figure}

Na Figura~\ref{fig:obj_e_npc1} fica visível que a área de interesse pode ser um fator limitante conforme a operação com jogadores, objetos e \acp{npc}~\cite{albion_online_unite}.
%
Dessa forma, o desempenho necessário para a execução dessas tarefas está tanto atrelado ao conjunto de algoritmos utilizado, quanto aos recursos computacionais gastos para tais operações.


Essas operações precisam de desempenho para não causar frustração ao jogador final~\cite{7008965}.
%
Nesse sentido, torna-se necessário conhecer os problemas computacionais recorrentes com relação aos serviços de jogos \ac{mmorpg}.



\section{Problemas em jogos \ac{mmorpg}}
\label{sec:problemas}

Uma métrica popular para mensurar o desempenho de um serviço \ac{mmorpg} é o número de conexões~\cite{1417630} simultâneas suportadas.
%
Em geral, caso o serviço ultrapasse o limite para o qual foi projetado, diversas falhas de conexão, problemas de lentidão ou dessincronização com o cliente podem ocorrer.
%
Neste contexto, as ocorrências comuns são~\cite{1417630}:

\begin{itemize}
  \item \textbf{Longo tempo de resposta aos clientes}: implica em uma qualidade insatisfatória de jogabilidade ao usuário ou até mesmo impossibilitando o uso do serviço.
  \item \textbf{Dessincronização com os clientes}: realiza reversão das ações realizadas pelo jogador na aplicação. Reversão é definida pela situação na qual uma requisição é solicitada ao servidor, um pré-processamento aparente é executado no cliente, porém esta requisição é negada, sendo necessário desfazer o pré-processamento aparente realizado ao cliente.
  \item \textbf{Problemas internos ao serviço}:  podem estar relacionados a diversos outros erros internos de implementação ou a capacidade de recurso computacional (\textit{e.g.,} sobrecarga no banco de dados, gerenciamento lento do espaço ou inconsistências dentro do jogo perante a regra de negócios).
  \item \textbf{Falha de conexão entre o cliente e o serviço}: causa a negação de serviço ao usuário final.
\end{itemize}

Existem algumas causas comuns para as ocorrências descritas~\cite{1417630}:

\begin{itemize}
  \item \textbf{Baixo poder computacional do servidor}: poder computacional abaixo do necessário para a qualidade de experiência de jogabilidade do usuário final desejada.
  \item \textbf{Complexidade de algoritmos}: o serviço usa algoritmos de alta complexidade ou regras de negócios que demandam por um algoritmo complexo.
  \item \textbf{Limitado pela própria arquitetura}: está limitado diretamente pelo número de conexões, não suportando a carga recebida.
  \item \textbf{Limitado pela rede}: a quantidade de requisições não é suportada pelo meio físico na qual a arquitetura está implantada.
\end{itemize}

Tais ocorrências estão diretamente correlacionadas a carga na qual tais serviços estão submetidos.
%
Esta carga pode ser amenizada utilizando técnicas de provisionamento de recursos e balanceamento de carga~\cite{1417630}, entretanto o serviço pode estar limitado pela sua arquitetura.

A área de desenvolvimento web compartilha várias ocorrências comuns geradas por sobrecarga do serviço~\cite{7830692}.
%
Em desenvolvimento web é comum utilizar a abordagem de microsserviços para resolver o problema de sobrecarga, modularizando o  funcionamento em módulos menores.
%
Da mesma forma, faz sentido modularizar um serviço \ac{mmorpg} em microsserviços para suportar cargas maiores e diminuir o custo de manutenção~\cite{7515686}.



Do ponto de vista da arquitetura de computadores, as operações existentes em um jogo \ac{mmorpg} seguem um padrão de interação com o mundo, criar, excluir ou manipular objetos deste mundo.
%
Para suprir o desenvolvimento de tais sistemas, se faz necessário compreender os padrões de desenvolvimento de tais arquiteturas, os quais implementam as operações básicas de interação com o mundo.



\subsection{Arquitetura de Clientes MMORPG}
\label{sec:cliente}



A arquitetura de um cliente \ac{mmorpg} é um aspecto fundamental, mas não único, para o sucesso de um jogo deste gênero.
%
O seu funcionamento é totalmente visível ao usuário final e tem o principal objetivo de exibir o estado do mundo de forma gráfica ao usuário~\cite{albion_online_unite}.
%
Um exemplo de cliente \ac{mmorpg} é o jogo Sandbox-Interactive Albion\footnote{Sandbox-Interactive Albion: \url{https://albiononline.com/en/home}}, que pode ser visualizado na Figura~\ref{fig:cliente_albion}.



\begin{figure}[htb!]
\caption{Exemplo de Cliente MMORPG (Sandbox-Interactive Albion).}
\label{fig:cliente_albion}
\includegraphics[width=\textwidth]{img/cap2/cliente_albion.png}
\centering

Fonte:~\cite{albion_online_unite}
\end{figure}

A Figura~\ref{fig:cliente_albion} representa na prática, como será exibido ao usuário final o ambiente do jogo.
%
O modelo teórico apresentado no cliente leva como base o campo de visão do jogador, no qual pode ser visualizado na Figura~\ref{fig:proximidade}.


Do ponto de vista de computação gráfica, um cenário 3D pode ser visualizado como uma árvore.
%
Utilizar árvores para descrever um cenário ajuda tanto no formado de armazenamento em disco para leitura facilitada (\textit{e.g.}, \ac{json}, \ac{xml}, etc.), operações de inserção e exclusão, organização do projeto e redução da complexidade utilizando transformações lineares em sistemas gráficos como \textit{OpenGL} e \textit{DirectX}~\cite{Lengyel2011Jun}.
%
Esse modelo é amplamente utilizado em motores gráficos, na qual pode ser encontrado em motores gráficos populares como \textit{Godot}, \textit{Unity3D} e \textit{Unreal 3}.
%
A Figura~\ref{fig:scene_tree} ilustra um exemplo exibindo a árvore de um cenário na \ac{ide} do motor gráfico \textit{Godot}.



\begin{figure}[htb!]
\caption{\textit{Scene tree view} no motor gráfico Godot.}
\label{fig:scene_tree}
\includegraphics[height=8cm]{img/cap2/scene_tree.png}
\centering

Fonte: O próprio autor.
\end{figure}



Dentro dessa árvore, cada nodo tem uma funcionalidade específica.
%
Essas funcionalidades variam a cada motor gráfico, podendo existir nodos focados à parte física, renderização ou controles~\cite{godot_docs}.
%
Em um jogo \ac{mmorpg}, o seu serviço será responsável por enviar atualizações dos parâmetros aos nodos frequentemente e o cliente será responsável por realizar chamadas remotas a fim de descrever as ações a qual o jogador aplicou sobre seu personagem~\cite{photon_engine}.


Do ponto de vista da rede de computadores, a arquitetura de um cliente de jogo \ac{mmorpg} deve suportar consultas e chamadas de métodos remotos em um serviço~\cite{albion_online_unite}.
%
Um cliente, para um jogo \ac{mmorpg}, pode seguir o estilo de arquitetura \ac{rest}, porém não obrigatoriamente sobre o protocolo \ac{http}, mas sim usando algum protocolo \ac{rpc} sobre o protocolo \ac{tcp} ou \ac{udp}~\cite{albion_online_unite, stephenclarkewillson2017}.
%
Essa comunicação é realizada pelo módulo \textit{Network}, presente em um cliente \ac{mmorpg}.


O módulo de \textit{Network} implementado em um cliente de jogo \ac{mmorpg} é responsável por realizar as requisições \ac{rpc} conforme as ações realizadas pelo jogador.
%
Este também é responsável por aplicar os parâmetros na \textit{Scene Tree} ou chamar métodos remotos no cliente por ordens do serviço.
%
Além disso, o módulo \textit{Network} possui uma \textit{Thread} dedicada ao gerenciamento de entrada e saída em relação ao serviço~\cite{albion_online_unite}.
%
Os principais módulos podem ser observados na Figura~\ref{fig:gateway}.


\begin{figure}[htb!]
\caption{Modelo de um cliente genérico.}
\label{fig:gateway}
\includegraphics[width=\textwidth]{img/cap2/cliente.png}
\centering

Adaptado de:~\cite{507915, faber}
\end{figure}


A Figura~\ref{fig:gateway} refere-se a uma visão macro de um cliente \ac{mmorpg}, na qual é possível ver o ator \textit{jogador} o qual pode executar ações sobre seu personagem por meio do motor gráfico.
%
Por sua vez, existe uma entrada de dados a mais comparado a esquemas de jogos \textit{offline}.
%
Neste caso, o módulo \textit{Rede} será igualmente uma entrada de dados, na qual poderá manipular a cena do motor gráfico~\cite{faber}.
%
Para facilitar o desenvolvimento, a aplicação de cliente é dividida em diversos módulos, entretanto são relevantes ao atual trabalho~\cite{albion_online_unite}:



\begin{itemize}
  \item \textbf{Motor Gráfico:} É o conjunto que aplicará regras sobre os objetos na \textit{Scene Tree}, receberá entradas do usuário e exibirá a \textit{Scene Tree} de forma imersiva. Unity3D\footnote{Unity3D: \url{https://www.unity3d.com}} e GodotEngine\footnote{GodotEngine: \url{https://www.godotengine.org}} são exemplos de \textit{engines}.
  \item \textbf{Rede:} É o módulo responsável pela comunicação entre o serviço e o cliente, a fim de requisitar chamadas de métodos ou obter informações do servidor para sincronizar os estados de jogo.
\end{itemize}



Utilizando esses dois módulos é possível sincronizar o estado de jogo e exibi-los ao jogador.
%
Entretanto, faz-se necessário compreender o funcionamento do serviço a fim de escolher um protocolo padrão para essa sincronização.



\subsection{Arquitetura de Microsserviços}
\label{sec:microsservicos}



Entende-se por microsserviço, aplicações que executam operações menores de um serviço, da melhor forma possível~\cite{stephenclarkewillson2017, Newman2015Feb}.
%
O objetivo de uma arquitetura de microsserviços é funcionar separadamente de forma autônoma, contendo baixo acoplamento~\cite{Newman2015Feb}.
%
Seu funcionamento deve ser desenhado para permitir alinhamentos de alta coesão e baixo acoplamento entre os demais microsserviços existentes em um serviço~\cite{8169955}.



Arquiteturas de microsserviços iniciam uma nova linha de desenvolvimento de aplicações preparadas para executar sobre nuvens computacionais, promovendo maior flexibilidade, escalabilidade, gerenciamento e desempenho, sendo a principal escolha de arquitetura de grandes empresas como Amazon, Netflix e LinkedIn~\cite{7830692,7515686}.
%
Um microsserviço é definido pelas seguintes características~\cite{8169955}:



\begin{itemize}
  \item Deve possibilitar a implementação como uma peça individual do serviço.
  \item Deve funcionar individualmente.
  \item Cada microsserviço deve ter uma interface. Essa interface deve ser o suficiente para utilizar o microsserviço.
  \item A interface deve estar disponível na rede para chamada de processamento remoto ou consulta de dados.
  \item O microsserviço pode ser utilizado por qualquer linguagem de programação e/ou plataforma.
  \item O microsserviço deve executar com as dependências mínimas.
  \item Ao agregar vários microsserviços, o serviço resultante poderá prover funcionalidades complexas.
\end{itemize}



O microsserviço deverá ser uma entidade separada.
%
A entidade deve ser implantada sobre um sistema isolado (\textit{e.g.,} Docker~\footnote{Docker:~\url{https://www.docker.com/}}, \acp{vm}, \textit{etc.}).
%
Toda a comunicação entre os microsserviços de um serviço será executada sobre a rede, a fim de reforçar a separação entre cada serviço.
%
As chamadas pela rede com o cliente ou entre os microsserviços será executada através de uma \ac{api}, permitindo a liberdade de tecnologia em que cada microsserviço será implementado~\cite{Newman2015Feb}.
%
Isso permite que o sistema suporte tecnologias distintas que melhor resolvam os problemas relacionados ao contexto deste microsserviço.
%
Isso pode ser visualizado na Figura~\ref{fig:microsservicos_tecnologias}.



\begin{figure}[htb!]
\caption{Microsserviços podem ter diferentes tecnologias.}
\label{fig:microsservicos_tecnologias}
\includegraphics[height=5cm]{img/cap2/microsservicos_tecnologias.png}
\centering

Adaptado de:~\cite{Newman2015Feb}
\end{figure}


Uma arquitetura de microsserviços é escalável, como visível na Figura~\ref{fig:microsservicos_escalabilidade}.
%
A arquitetura permite o aumento do número de microsserviços sob demanda para suprir a necessidade de escalabilidade.
%
Este modelo computacional obtém maior desempenho, principalmente se executar sobre plataformas de computação elástica, na qual o orquestrador do serviço pode aumentar o número de instâncias conforme a necessidade de requisições~\cite{Nadareishvili2016Aug}.



\begin{figure}[htb!]
\caption{Microsserviços são escaláveis.}
\label{fig:microsservicos_escalabilidade}
\includegraphics[width=\textwidth]{img/cap2/microsservicos_escalabilidade.png}
\centering

Adaptado de:~\cite{Newman2015Feb}
\end{figure}



Microsserviços desenvolvidos para web utilizam arquitetura \ac{rest} baseado sobre o protocolo \ac{http}.
%
É uma boa prática utilizar o corpo com conteúdo da requisição e resposta no formato \ac{json} nas chamadas a uma \ac{api} de microsserviço web~\cite{Nadareishvili2016Aug}.
%
Entretanto, não é uma prática comum para um serviço \ac{mmorpg} utilizar o protocolo \ac{http} pela sua elevada carga administrativa na requisição~\cite{1417630}.
%
Por esse motivo, torna-se relevante compreender a composição de uma arquitetura com microsserviços para \ac{mmorpg}, visto que possuem um foco objetivo devido ao gênero do jogo.



\subsection{Microsserviços para jogos \ac{mmorpg}}
\label{sec:arquiteturas}



A fim de otimizar o tempo de resposta das arquiteturas de microsserviços de jogos \ac{mmorpg}, é incomum a utilização de protocolos \ac{http} em tais arquiteturas.
%
Por esse motivo, a seção atual mostrará o funcionamento básico do protocolo \ac{rpc} e a sua utilização para atualização dos parâmetros na \textit{Scene Tree} e o modelo \ac{rest}~\cite{albion_online_unite}.



Em engenharia de software, é comum a utilização de arquiteturas \ac{mvc} a fim de organizar o código fonte e prover agilidade de desenvolvimento~\cite{Chadwick2012Oct, LeckyThompson2008Nov}.
%
A separação de um serviço \ac{mmorpg} pode ser dada seguindo este padrão de projeto, dividindo-se em três camadas~\cite{5718337}:


\begin{enumerate}
  \item \textit{Model}: Representa qualquer dado presente no jogo (\textit{e.g.,} itens, personagens, \acp{npc}, objetivos, etc.).
  \item \textit{View}: Representa o modo a qual estes dados são exibidos, do ponto de vista de redes (\textit{e.g.,} O mapa será exibido somente na área de interesse do jogador, no formato \ac{json}).
  \item \textit{Controller}: Representa as operações sobre modelos que são requiridos pelos jogadores (\textit{e.g.,} andar, pegar item, interagir com \acp{npc}, etc.).
\end{enumerate}



Dentro de um \textit{Controller} é implementado operações a qual o cliente pode requirir através de chamadas \ac{rpc} a fim de manipular ou obter o estado de \textit{Models} da aplicação.
%
Esses métodos padrões seguem o protocolo \ac{crud}, contendo quatro métodos principais para complementar as consultas sobre os \textit{Models}~\cite{Chadwick2012Oct, LeckyThompson2008Nov}:



\begin{enumerate}
  \item \textit{Create}: Representa a criação de um novo objeto no banco.
  \item \textit{Update}: Representa a atualização de um objeto no banco.
  \item \textit{Delete}: Representa a exclusão de um objeto no banco.
  \item \textit{Read}: Representa a consulta sobre este objeto no banco.
\end{enumerate}



Para o padrão \ac{crud}, faz-se necessário que os métodos \textit{Read}, \textit{Update} e \textit{Delete} repassem o parâmetro de identificação do objeto a ser consultado~\cite{LeckyThompson2008Nov}.
%
Outros argumentos necessários nos métodos \textit{Create} e \textit{Update} são os atributos do objeto, além do seu retorno ser um valor booleano representando se a operação foi bem sucedida~\cite{Chadwick2012Oct, LeckyThompson2008Nov}.
%
Entretanto, outros métodos mais apropriados ao funcionamento de um \textit{Controller} podem existir.
%
Como exemplo, pode-se visualizar uma interface \ac{crud} na Figura~\ref{fig:crud}.



\begin{figure}[htb!]
\caption{Cliente pode realizar requisições \ac{crud} ao serviço.}
\label{fig:crud}
\includegraphics[height=2.5cm]{img/cap2/crud.png}
\centering

Fonte: Adaptado de~\cite{albion_online_unite}.
\end{figure}



Essas operações são executadas utilizando protocolo \ac{rpc}~\cite{albion_online_unite}.
%
As requisições de métodos remotos são realizadas entre dois processos distintos, a fim de gerar uma computação distribuída~\cite{rpc}.
%
Entretanto, a base do protocolo não é legível como em servidores web que utilizam \ac{json} para transmissão de estrutura de dados, mas sim uma codificação binária nomeado \ac{xdr}~\cite{xdr}.


Uma técnica comum em jogos é a compressão de pacotes utilizando mapeamento \textit{hash} de bytes~\cite{LeckyThompson2008Nov}.
%
Tanto o cliente quanto o serviço precisam ter a mesma estrutura de dados.
%
Dessa forma, é possível trocar o nome das funções solicitadas em \ac{rpc} por poucos bytes para transitar na rede.
%
Já para operações \ac{crud}, pode-se utilizar tanto requisições sobre o protocolo \ac{http} ou sobre um protocolo otimizado sobre \ac{tcp} dependendo da necessidade de desempenho, como exemplo o protocolo \ac{rpc} ou protocolos de fluxo de dados proprietários~\cite{LeckyThompson2008Nov}.

Dado um serviço que não é implementado sobre uma arquitetura de microsserviços, a utilização de dois protocolos irá complicar o gerenciamento de \textit{threads} para responder de duas formas diferentes, entretanto o seu esquema de estados ficará simplificado.
%
Esta simplificação pode ser visualizada na figura~\ref{fig:network_crud_rpc}.


\begin{figure}[htb!]
\caption{Diagrama de requisições entre serviço e cliente com operações \ac{crud} e \ac{rpc} em uma arquitetura monolítica.}
\label{fig:network_crud_rpc}
\includegraphics[height=6.5cm]{img/cap2/network_rpc_crud.png}
\centering

Fonte: Adaptado de~\cite{LeckyThompson2008Nov}
\end{figure}

Entretanto, diferente da Figura~\ref{fig:network_crud_rpc}, a fim de simplificar a complexidade da implementação e gerenciamento de concorrência no serviço, pode-se implementar utilizando o paradigma de microsserviços.
%
Como relatado na Seção~\ref{sec:microsservicos}, uma arquitetura de microsserviços permite múltiplas tecnologias, pois a comunicação entre todos os elementos de um microsserviço será pela rede.
%
Por esse motivo, é possível utilizar um serviço web para realizar operações \ac{crud} e um serviço dedicado para realizar operações \ac{rpc}.
%
Essa arquitetura pode ser melhor visualizada na Figura~\ref{fig:network_crud_rpc_micro}.

\begin{figure}[htb!]
\caption{Diagrama de requisições entre serviço e cliente com operações \ac{crud} e \ac{rpc} em uma arquitetura de microsserviços.}
\label{fig:network_crud_rpc_micro}
\includegraphics[height=6.5cm]{img/cap2/network_rpc_crud_micro.png}
\centering

Fonte: O próprio autor.
\end{figure}



Como resultado da integração entre Cliente, Serviço e Motor Gráfico, o resultado final obtido é descrito pelo diagrama presente na Figura~\ref{fig:integracao_unity_albion}.
%
Tal integração está presente no jogo Sandbox-Interactive Albion\footnote{Sandbox-Interactive Albion: \url{https://albiononline.com/en/home}}~\cite{albion_online_unite}.
%
Percebe-se, neste contexto, que o jogo deverá funcionar sem o motor gráfico, do ponto de vista de redes e estrutura de dados~\cite{albion_online_unite}.


\begin{figure}[htb!]
\caption{Diagrama de integração entre Cliente e Serviço, considerando a \textit{engine} Unity3D.}
\label{fig:integracao_unity_albion}
\includegraphics[width=\textwidth]{img/cap2/integracao_unity_albion.png}
\centering

Fonte:Adaptado de ~\cite{albion_online_unite}
\end{figure}

A Figura~\ref{fig:integracao_unity_albion} ilustra a separação da camada de renderização de objetos da árvore da cena, a camada de integração com o cliente e serviço (descrito anteriormente como módulo \textit{Network}) e o serviço.
%
Nesse sentido, a alteração entre clientes e serviços facilita um sistema de teste de carga e busca de erros automatizado, facilitando a manutenção e desenvolvimento incremental do serviço~\cite{albion_online_unite}.

Utilizando estes padrões de projeto, algumas arquiteturas tornam-se populares no desenvolvimento de jogos \ac{mmorpg}.
%
Para este fim, torna-se de interesse a este trabalho realizar um levantamento de algumas arquiteturas comuns em jogos massivos.



\section{Arquiteturas \ac{mmorpg} identificadas}

Dentre as arquiteturas identificadas, as qualificadas dentro do paradigma de arquiteturas de microsserviços são as arquiteturas Rudy (Subseção~\ref{rudy}), Salz (Subseção~\ref{salz}) e Willson (Subseção~\ref{willson}).

Em geral, as arquiteturas de forma genérica contém microsserviços \textit{web} para \textit{E-Commerce}, operações \ac{crud} através da web e distribuição de atualizações.
%
Os microsserviços de gerenciamento de jogo, por sua vez, respondem através do protocolo \ac{tcp}, podendo ser sobre \ac{rpc} ou protocolos proprietários.
%
A gerência de jogo é a principal mudança entre as arquiteturas, na qual contém abordagens de paralelismo diferentes:

\begin{itemize}
 \item A arquitetura Rudy (Subseção~\ref{rudy}) utiliza uma abordagem com um número menor de microsserviços, focado em manter serviços para consulta de dados de forma eficiente entre os serviços \textit{web} e o serviços de gerenciamento de jogo.
 \item A arquitetura Salz (Subseção~\ref{salz}) aborda um modelo de paralelismo mais complexo comparado a arquitetura Rudy, utilizando diversos microsserviços para funções específicas das funcionalidades do jogo.
%
Dessa forma, o ambiente do jogo torna-se escalável ao número de jogadores, porém a latência tende a aumentar.
\item A arquitetura Willson (Subseção~\ref{willson}) utiliza um modelo intermediário, evitando a divisão em múltiplos serviços para o gerente de jogo e utilizando um modelo de paralelismo próximo a arquitetura Salz.
\end{itemize}


\subsection{Arquitetura elaborada por Rudy}
\label{rudy}


A arquitetura Rudy~\cite{matthiasrudy2011} tem como objetivo criar múltiplos mundos isolados, na qual cada microsserviço será responsável por um ambiente a qual não compartilha dados com os demais ambientes.
%
Esta é uma característica importante para esta arquitetura, visto que o processamento de ações pelo serviço de jogo não precisa lidar com múltiplos processos~\cite{matthiasrudy2011}.
%
Esta arquitetura pode ser visualizada na Figura~\ref{full_rudy}.

\begin{figure}[htb!]
  \caption{Arquitetura Rudy completa.}
  \label{full_rudy}
  \includegraphics[width=\textwidth]{arquiteturas/full_rudy.png}
  \centering

  Adaptado de:~\cite{matthiasrudy2011}.
\end{figure}

No total, seis microsserviços distintos constam na arquitetura Rudy (Figura~\ref{full_rudy}) para o seu funcionamento, não sendo necessário o microsserviço de pagamento (utilizado somente para regra de negócios).
%
Os microsserviços que compõem a arquitetura são definidos pelas suas seguintes responsabilidades~\cite{matthiasrudy2011}:

\begin{enumerate}
  \item \textbf{Serviço de web estático}: Armazena documentos estáticos para o serviço web (\textit{e.g., }imagens, executáveis do jogo, páginas web fixas, etc). Responde sobre o protocolo \ac{http}.
  \item \textbf{Serviço de web dinâmico}: Sistema web para cadastro de contas, guias, informações sobre atualizações, compras e demais demanda de páginas dinâmicas. Responde sobre o protocolo \ac{http}.
  \item \textbf{Balanço de carga web}: Realiza a distribuição de carga sobre o \textit{Serviço web estático} e o \textit{Serviço web dinâmico}. Responde sobre o protocolo \ac{http}.
  \item \textbf{Serviço de Jogo}: Gerencia um mundo inteiro, em um único serviço. Esta abordagem segrega os jogadores em diversos canais, contendo um número máximo de conexões por canal. Cada canal opera sobre uma instância deste microsserviço. Este serviço opera sobre o protocolo \ac{rpc}.
  \item \textbf{Serviço de Autenticação}: Gerencia a autenticação das conexões ao \textit{Serviço de jogo}. Este serviço opera sobre o protocolo \ac{rpc}.
  \item \textbf{Gerenciador de Consultas}: Realiza consultas em memória e disco, utilizando vários bancos de dados diferentes, simulando o uso de banco de dados distribuídos, algo complexo de ser implementado utilizando banco de dados SQL (\textit{e.g.,} PostgreSQL\footnote{PostgreSQL: \url{https://www.PostgreSQL.org}}, MySQL\footnote{MySQL: \url{https://www.mysql.com/}}, etc). Este serviço opera sobre o protocolo \ac{http}.
\end{enumerate}


No contexto do atual trabalho, o serviço de pagamento será ignorado, visto que ele não serve para o funcionamento básico do serviço.
%
Dentre todos os microsserviços, o usuário só tem acesso ao serviço de balanço de carga pelo protocolo \ac{http} e o serviço de jogo sobre o protocolo \ac{rpc}, tendo os demais serviços protegidos por um \textit{firewall}~\cite{matthiasrudy2011}.
%
A proteção do \textit{firewall} é aplicada no serviço de pagamento e no ponto de acesso ao serviço, podendo ser visualizada na Figura~\ref{full_rudy_fw}.


\begin{figure}[htb!]
  \caption{Arquitetura Rudy completa com \textit{firewall}.}
  \label{full_rudy_fw}
  \includegraphics[width=\textwidth]{arquiteturas/full_rudy_fw.png}
  \centering

  Adaptado de:~\cite{matthiasrudy2011}.
\end{figure}




O funcionamento interno do serviço de jogo trabalha em rodadas, visando não penalizar usuários com baixa transferência de dados entre o cliente e o serviço.
%
Cada cliente produz requisições para serem consumidas pelo ciclo de processamento do gerente de jogo.
%
Entretanto, o serviço irá consumir de forma igualitária uma requisição de um jogador diferente, como uma fila~\cite{albion_online_unite, matthiasrudy2011}.
%
O modelo de processamento do gerente de ambiente pode ser visualizado na Figura~\ref{fig:thread_pool}.


\begin{figure}[htb!]
  \caption{Modelo de processos \textit{Thread Pool}.}
  \label{fig:thread_pool}
  \includegraphics[height=6.5cm]{arquiteturas/thread_pool.png}
  \centering

  Adaptado de:~\cite{matthiasrudy2011, Ringler2014Dec}.
\end{figure}

O modelo de processamento do gerente de mundo, visualizado na Figura~\ref{fig:thread_pool} trabalha com um a padrão \textit{Thread Pool}~\cite{Ringler2014Dec, matthiasrudy2011}, executando a chamada de método remoto de cada jogador em uma fila, o qual prioriza executar as chamadas sem repetir a mesma conexão.
%
Dessa forma, cada jogador pode executar somente um método concorrente, sem competir com os demais. Todas as requisições são enfileiradas no \textit{buffer} de rede do serviço.
%
Caso o cliente entre em sua vez de processamento, e nenhuma chamada remota esteja na fila, ele é pulado.



Um serviço que demanda atenção na arquitetura Rudy é o Gerenciador de Consultas, um serviço web que implementa uma camada sobre diversos bancos de dados a fim de prover variedade entre vários bancos de forma facilitada por requisições web, utilizando operações \ac{crud}.
%
Implementar esta camada garante uma padronização de acesso ao banco de dados, porém adiciona um possível gargalo a arquitetura~\cite{matthiasrudy2011}.
%
Os pontos positivos de utilizar esta camada de consultas na arquitetura são:
\begin{enumerate}
  \item Não permite acesso direto ao banco de dados do serviço web e do serviço de jogo.
  \item Permite maior manejo a migrações em tabelas e troca de tecnologias.
  \item Define uma sintaxe estrita para consulta, via \ac{crud}.
  \item Permite acesso do banco a diversos serviços, sem gerenciar o banco.
  \item Permite contar número de requisições e tempo das requisições.
\end{enumerate}



Contudo, adicionar uma camada sobre os bancos de dados para gerenciamento tem pontos negativos~\cite{matthiasrudy2011}.

\begin{enumerate}
  \item Aumenta a complexidade de implementação, teste, administração e ponto de falha.
  \item Adiciona limites como número de conexões, número de requisições, etc.
\end{enumerate}

Entretanto, esta arquitetura não permite escalar um único ambiente para um número de jogadores simultâneos maior ao designado pelo hardware que hospeda o serviço.
%
Por este motivo, as arquiteturas Salz e Willson tomam abordagens para subdividir o ambiente do jogo em mais serviços, podendo assim escalar um único ambiente para mais jogadores simultâneos.


\subsection{Arquitetura elaborada por Salz}
\label{salz}

A arquitetura elaborada por Salz~\cite{albion_online_unite}, a qual pode ser visualizada na Figura~\ref{full_salz} contém sete microsserviços especializados para seu funcionamento.
%
Para o funcionamento adequado, o cliente necessita manter três conexões abertas com o servidor, sendo a conexão de jogo a com maior demanda de banda e a qual necessita o menor tempo de latência~\cite{albion_online_unite}.

\begin{figure}[htb!]
  \caption{Arquitetura Salz.}
  \label{full_salz}
  \includegraphics[width=\textwidth]{arquiteturas/full_salz.png}
  \centering

  Adaptado de:~\cite{albion_online_unite}.
\end{figure}



A Figura~\ref{full_salz} exibe a arquitetura Salz de forma completa, na qual encontram-se quatro bancos de dados distintos, sendo eles o banco de \textit{Pagamento}, \textit{Negociação}, \textit{Autenticação} e \textit{Jogo}, sendo os bancos de Autenticação e Jogo com tecnologias \ac{nosql}. Os demais bancos utilizam tecnologia \ac{sql}.
%
Referente aos microsserviços que compõem a arquitetura, tem-se os seguintes elementos~\cite{salz_albion}:



\begin{enumerate}
  \item \textbf{Serviço de Comunicação}: Gerencia a troca de mensagens entre os jogadores. Opera sobre o protocolo \ac{rpc}.
  \item \textbf{Serviço de Autenticação}: Recepciona e gerencia as conexões dos clientes entre os demais microsserviços. Opera sobre o protocolo \ac{http}.
  \item \textbf{Serviço de Jogo}: Gerencia o estado do mundo, referente a um \textit{chunk} do ambiente. Opera sobre o protocolo \ac{rpc}.
  \item \textbf{Serviço Global}: Gerencia operações globais (\textit{e.g,} interações entre grupos, procedimentos recorrentes globais, \textit{etc.}). Opera sobre \ac{http}.
  \item \textbf{Serviço de Pagamento}: Efetua operações bancárias com serviços externos de pagamento e gerencia o estado de pagamento das contas. Opera sobre o protocolo \ac{http}.
  \item \textbf{Serviço de Negociação}: Opera como um serviço de leilão para itens do jogo. Opera sobre o protocolo \ac{http}.
\end{enumerate}



Os microsserviços que compõem a arquitetura Salz utilizam em grande parte serviços \textit{web}, utilizando somente o protocolo \ac{rpc} para o serviço de jogo, autenticação e comunicação.
%
No serviço de jogo, tanto o cliente quando o serviço podem invocar métodos remotos através do protocolo \ac{rpc}, tendo como padrão métodos com retorno nulo~\cite{salz_albion, photon_serialization}.
%
Esta abordagem garante que o usuário não fique esperando pelo retorno do serviço para continuar a lógica do jogo~\cite{faber}.
%
Para este funcionamento, o modelo de processamento paralelo deste serviço possui mais regras, a qual não são abordadas pela arquitetura Rudy (Figura~\ref{full_rudy}).
%
O modelo de paralelismo do serviço de jogo pode ser visualizado na Figura~\ref{salz_thread_model}.


\begin{figure}[htb!]
  \caption{Modelo de paralelismo do serviço de jogo na arquitetura Salz.}
  \label{salz_thread_model}
  \includegraphics[height=6.5cm]{arquiteturas/salz_thread_model.png}
  \centering

  Adaptado de:~\cite{salz_albion, willson}.
\end{figure}



O microsserviço de jogo executa a lógica do jogo, visível na Figura~\ref{salz_thread_model}, para uma única área em um único processo.
%
Esta decisão é dada por conta da interação entre objetos ser complicada de executar em paralelo, visto o gerenciamento do custo de gerenciamento de semáforos necessários, caso execute estas ações em paralelo~\cite{salz_albion}.



Outra facilidade de implementar um modelo com um único processo para as interações entre objetos é facilitar o não manejo de objetos a qual não estão no campo de visão de todos os jogadores do serviço, realizando estas operações somente quando necessário~\cite{albion_online_unite, salz_albion}.


As entradas e saídas de mensagens(\textit{e.g.,} rede, Banco de dados, \textit{etc.}) e busca de caminho é executado por processos de trabalho separado do principal, utilizando a técnica de \textit{Thread Pool}~\cite{salz_albion, albion_online_unite, Ringler2014Dec}.
%
Todos os demais microsserviços operam sobre múltiplos processos, a partir do processo de conexão \ac{tcp} ao serviço~\cite{salz_albion}.
%
Diferente da arquitetura Rudy(Seção ~\ref{rudy}) a qual prioriza a resolução de requisições, evitando a executar métodos consecutivos da mesma conexão, na arquitetura Salz (Figura~\ref{full_salz}) as chamadas de processo remoto são executadas em ordem \ac{fifo}~\cite{salz_albion}.
%
Desta forma, quanto melhor a conexão com o serviço, mais requisições um cliente poderá executar.
%
Nesse sentido, uma conexão que transfere mais requisições terá mais vantagem sobre uma conexão a qual não transfere tantas requisições, executando mais operações por segundo.



Na arquitetura Salz, são necessárias três conexões \ac{tcp} com o servidor, de forma contínua~\cite{albion_online_unite}.
%
O custo desta operação é alto, e por isso nem sempre é viável utilizar esta abordagem, principalmente para jogos nos quais a demanda de banda seja menor.
%
Para evitar esta decisão, da mesma forma é notória a abordagem utilizada pela arquitetura Willson.



\subsection{Arquitetura elaborada por Willson}
\label{willson}


A arquitetura elaborada por Clarke-Willson leva como principal característica a preocupação da disponibilização de atualizações aos clientes~\cite{willson}.
%
Essas atualizações são distribuídas a todos os microsserviços utilizando sistemas de versionamento de código fonte~\cite{stephenclarkewillson2017, willson}.
%
Por este motivo, a sua arquitetura inclusive compreende microsserviços de armazenamento de arquivos sobre protocolo \ac{http} para atualização dos clientes~\cite{stephenclarkewillson2017}.
%
Uma outra característica do serviço de jogo é a utilização de uma única conexão \ac{tcp} por cliente.
%
Essa arquitetura pode ser visualizada na Figura~\ref{full_willson}.


\begin{figure}[htb!]
  \caption{Arquitetura Willson.}
  \label{full_willson}
  \includegraphics[width=\textwidth]{arquiteturas/full_willson.png}
  \centering

  Fonte:~\cite{stephenclarkewillson2017}.
\end{figure}


A arquitetura Willson, exibido na Figura~\ref{full_willson}, mostra um gradual entre a arquitetura Rudy (Figura~\ref{full_rudy}) e a arquitetura Salz (Figura~\ref{full_salz}), propondo uma arquitetura híbrida, na qual divide funcionalidades em outros microsserviços, porém ainda mantem diversas funcionalidade junto ao serviço de jogo evitando consumo de rede~\cite{albion_online_unite, willson}.
%
Essa abordagem garante menor latência de resposta, porém terá maior consumo de recursos da mesma máquina hospedeira do serviço~\cite{willson}.
%
Os microsserviços que compõem a arquitetura Willson são~\cite{willson, stephenclarkewillson2017}:

\begin{enumerate}
  \item \textbf{Serviço web}: Exibe informações do jogo, oferece operações \ac{crud} para cadastro de usuários e o sistema de pagamento. Opera sobre o protocolo \ac{http}.
  \item \textbf{Serviço de Atualização}: Integrado junto ao processo de desenvolvimento, fornecendo versionamento do cliente por um sistema web. Este microsserviço utiliza serviços internos ao desenvolvimento da aplicação. Opera sobre o protocolo \ac{http}.
  \item \textbf{Serviço de Autenticação}: Gerencia a autenticação dos jogadores através de um serviço web. Também gerencia ações sociais (\textit{e.g.} sistema de amigos, grupos, nações, comércio, \textit{etc}). Opera sobre o protocolo \ac{http}.
  \item \textbf{Serviço de Balanceamento de carga}: Gerencia a carga entre os serviços web da arquitetura. Opera sobre o protocolo \ac{http}.
  \item \textbf{Serviço de Jogo}: Processa a lógica de jogo. Cada instância deste microsserviço gerencia um \textit{chunk} do ambiente do jogo. Opera sobre o protocolo \ac{rpc}.
  \item \textbf{Serviços Globais}: Processa rotinas globais do jogo, a qual não dependem do posicionamento do jogador. Opera sobre o serviço \ac{rpc}.
\end{enumerate}

O serviço de jogo da arquitetura Willson é comparada ao serviço similar na arquitetura Salz, definido no Subcapítulo~\ref{salz}, entretanto utiliza a técnica de \textit{thread pool} com valor de processos da fila de processadores fixo conforme o \textit{hardware} hospedeiro do serviço~\cite{stephenclarkewillson2017}.
%
Para modelos de paralelismo, pode-se utilizar os seguintes números de processos paralelos~\cite{willson}:


\begin{enumerate}
  \item O dobro de número de núcleos: exige maior carga do serviço por troca de contexto entre os processos.
  \item O número exato de núcleos mais n: O serviço perderá tempo de processamento, trocando de contexto para outro processo, porém de forma mais sutil ao dobro do número de núcleos de processamento.
  \item O número de núcleos: exigindo um número menor de carga de contexto, dividindo somente com o sistema operacional.
  \item O número de núcleos menos um: liberando um núcleo para o sistema operacional.
\end{enumerate}

O número padrão de processos paralelos para processamento de requisições é o número de núcleos menos um, visando evitar a troca de contexto com o sistema operacional~\cite{willson}.
%
Essa troca de contexto trás variação na latência da resposta do serviço, a qual não tem controle pelo próprio serviço.
%
Nesse sentido, a melhor escolha é o número de núcleos menos um, escolhido após um teste de \textit{stress}, definido no Subcapítulo~\ref{willson}.



Entretanto, não foi encontrado análises públicas sobre as arquiteturas Rudy, Salz e Willson, com relação ao uso de recursos dessas arquiteturas.
%
Nesse sentido, o atual trabalho define o seu problema sobre o consumo de recursos em buscar uma análise sobre o comportamento das arquiteturas referenciadas.


\section{Definição do Problema}



A escolha de uma arquitetura para serviços \ac{mmorpg} que não suportem uma elevada carga de trabalho podem ser um eventual problema na escalabilidade do negócio de empresas que forneçam tais serviços.
%
Em alguns serviços, a divisão da carga é dada pela área de interesse, dividindo o ambiente em pedaços a fim de diminuir a carga no serviço.
%
Porém, locais populares no ambiente do jogo ainda são vulneráveis a disponibilidade de serviço~\cite{1417630}.



Nesse sentido, arquiteturas de microsserviços surgiram com o objetivo de aumentar a disponibilidade e viabilizar produtos de demanda massiva.
%
Tais arquiteturas dividem o seu funcionamento em módulos menores a fim de proporcionar maior demanda utilizando uma abordagem distribuída.
%
Assim, o custo de atender uma alta demanda de conexões pode consumir uma quantia de recursos computacionais ou uma qualidade insatisfatória de serviço, inviabilizando a sua utilização no mercado.



Um exemplo de implementação de arquitetura de um jogo \ac{mmorpg} pode ser visualizado na Figura~\ref{fig:generica}, no qual cada cliente deve conectar em um microsserviço de gerenciamento de mundo para receber as atualizações da região e submeter seus comandos.
%
Nessa arquitetura, é necessário levar em conta o funcionamento, método de compartilhamento dos dados e abordagem para processamento do ambiente do jogo a fim de descrever a sua escalabilidade e qualidade de serviço.


\begin{figure}[htb!]
\caption{Arquitetura de um jogo \ac{mmorpg} genérico.}
\label{fig:generica}
\includegraphics[height=9.0cm]{img/cap3/generica.png}
\centering

Fonte: O próprio autor
\end{figure}

A arquitetura genérica descrita na Figura~\ref{fig:generica} também precisa corresponder as demandas necessárias de regra de negócio do próprio jogo.
%
Neste caso, só um tipo de microsserviço é visível nesta rede, porém poderiam existir microsserviços responsáveis pela parte social, comercial e estatísticas, aumentando o grau de complexidade da arquitetura.
%
Em relação aos recursos utilizados por uma arquitetura, este trabalho foca na análise das arquiteturas de microsserviços descritas na literatura a fim de analisar o seu comportamento e desempenho.
%
Porém, não foi encontrado nenhuma análise das arquiteturas de microsserviços propostas por Rudy, Salz e Willson, tornando a escolha de um modelo de arquitetura uma tarefa complexa a qual demande testes de tentativa e erro ou análises particulares para viabilizar o funcionamento destes sistemas.
%
Dessa forma, este trabalho tem como principal objetivo analisar as arquiteturas selecionadas com foco em uso dos recursos computacionais, realizando comparações e encontrando características das arquiteturas.

\section{Considerações parciais}


Este capítulo conceituou jogos eletrônicos, gênero de jogo e especificou as características de um jogo \ac{mmorpg}.
%
Após apresentar sobre o gênero de jogo abordado, detalha-se a sua jogabilidade, problemas relevantes a este gênero do ponto de vista de rede de computadores e por fim sobre técnicas e abordagens populares acerca do desenvolvimento destes serviços, em específico sobre o paradigma de arquitetura de microsserviços.

O desenvolvimento de arquiteturas de microsserviços para jogos \ac{mmorpg} é uma tarefa multidisciplinar a qual pequenas variações de protocolo, algoritmos ou microsserviços dentre os modelos impacta diretamente no consumo de recurso destas arquiteturas.
%
Nesse sentido, o atual capítulo trouxe inicialmente um levantamento teórico sobre as principais características deste gênero de jogo, características das principais camadas de cliente, servidor e serviço e por sua vez mostrou abordagens viáveis de desenvolvimento para ambas as camadas.
%
Por fim trouxe uma introdução ao paradigma de arquiteturas de microsserviços, abordando em seguida a sua aplicação para jogos \ac{mmorpg}.

Após a introdução técnica necessária para entendimento das arquiteturas, foi realizado um levantamento de arquiteturas de microsserviços da literatura, descrevendo sua funcionalidade, topologia e por fim protocolos utilizados.
%
Entretanto, não foi encontrado trabalhos correlacionados as arquiteturas encontradas, dessa forma encontrando um dos pontos de apoio científico deste trabalho.

A literatura aborda previsibilidade de carga, análise de disponibilidade e uso de recurso de tais arquiteturas (a fim de guiar escolhas na etapa de elaboração da regra de negócio do produto e viabilizar o comércio do mesmo), relatados na Seção~\ref{sec:similares}.
%
Torna-se de interesse para projetistas de desenvolvimento de jogos \ac{mmorpg} analisar o impacto e uso de recursos computacionais ao implantar uma arquitetura de microsserviços \ac{mmorpg} visando melhorar a disponibilidade de tais jogos.
%
Nesse sentido, se faz necessário realizar um levantamento bibliográfico a cerca dos temas de jogos \ac{mmorpg} e arquiteturas de microsserviços com a finalidade de comprovar a utilidade da proposta do atual trabalho.
