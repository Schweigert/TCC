\chapter{Fundamentação Teórica}
\label{cap2}

\section{Jogos Eletrônicos}



Jogo eletrônico é uma forma de entretenimento digital que seguem regras de negócio a fim de proporcionar alguma emoção no jogador ao completar algum objetivo dentro deste jogo.
%
Tais jogos podem ser classificados como coletivos, individuais ou competitivos~\cite{video_game_technologies}.



Como conceito, um jogo eletrônico é um jogo controlado por computador, onde o termo \textit{entretenimento interativo} é a referência formal para qualquer plataforma a qual execute um jogo eletrônico.
%
O termo jogo significa gracejo, brincadeira, divertimento. O jogo eletrônico é uma atividade intelectual que integra um sistema de regras, na qual utiliza esse sistema de regras a fim de definir seus objetivos ou pontuação por meio de um computador~\cite{video_game_technologies}.



Uma \textit{plataforma} é um computador a qual executa as regras de negócio e exibem o jogo de forma visual ao jogador. Um \textit{video game} é um computador na qual a televisão é o dispositivo de \textit{feedback} primário~\cite{video_game_technologies}.
%
Um \textit{jogo de computador} (\textit{PC Game} ou \textit{Computer Game}) são jogos específicos para computadores pessoais, já um \textit{jogo de console} executa em um video game~\cite{video_game_technologies}.



Os jogos eletrônicos podem ser classificados conforme o seu \textit{gênero}.
%
Eles são classificados dentro de cada gênero conforme um estilo comum ou um conjunto de características (\textit{e.g.,} perspectiva, estilo de jogo, interação, objetivo, etc)~\cite{video_game_technologies}.



Um grande problema classificação por gênero é a arbitrariedade e consistência.
%
É comum um jogo eletrônico ter características de dois grandes grupos.
%
Nesses casos ele pode dividir seu nome dentre os dois gêneros ou gerar um novo gênero~\cite{video_game_technologies} dentro da árvore de categorização de jogos (Seção ~\ref{sec:arvore_de_categoria_de_jogos}).



\subsection{Árvore de categoria de jogos}
\label{sec:arvore_de_categoria_de_jogos}


A árvore de categoria de jogos descreve grupo de jogos que são semelhantes por características específicas a cada jogo~\cite{video_game_technologies}.
%
Tal árvore é importante para melhor classificar eventuais títulos do mercado para uma visão macro do sistema computacional~\cite{video_game_technologies}.
%
É válido ressaltar que mesmo jogos parecidos podem obter abordagens diferentes em seu desenvolvimento, seja por variações de hardware, software, tempo e recurso de investimento para seu desenvolvimento~\cite{video_game_technologies}.
%
É válido ressaltar que um título também não precisa estar em uma única categoria dessa árvore~\cite{video_game_technologies}.
%
A árvore de categoria de jogos descrita abaixo mostra de forma macro as principais categorias de jogos e onde encontra-se a categoria de jogos do atual trabalho~\cite{video_game_technologies}.

\begin{itemize}
  \item \textbf{Aventura}: Essa categoria de jogos aborda a exploração e resolução de problemas lógicos básicos, como carregar objetos ou ativar alavancas em alguma ordem para abrir uma determinada região. O primeiro jogo dessa categoria é Zork\footnote{Zork: \url{https://github.com/devshane/zork}}, lançado nos anos 1980.
  \item \textbf{Ação}: Essa categoria de jogos aborda a habilidade de movimentação dos controles do jogo e velocidade de resposta do jogador. Um título famoso é o jogo Space Invaders\footnote{Space Invaders: \url{https://github.com/dwmkerr/spaceinvaders}}.
  \item \textbf{Ação e Aventura}: Essa categoria de jogos mescla as duas categorias acima. Seu objetivo é desvendar problemas lógicos enquando movimenta-se em algum cenário com as habilidades do jogo. Um exemplo é a série de jogos Legend of Zelda\footnote{Legend of Zelda: \url{https://www.zelda.com/}}.
  \item \textbf{Plataforma}: Essa categoria de jogos utiliza a habilidade de movimentação de um personagem virtual para passar por obstáculos em estágios. New Super Mario Bros\footnote{New Super Mario Bros: \url{http://www2.nintendo.com.au/catalogue/new-super-mario-bros}} é um exemplo dessa categoria.
  \item \textbf{Luta}: Essa categoria utiliza a dextreza de velocidade reflexos do jogador a e conhecimento dos personagens a fim de ganhar de seu oponente. Em geral a sua fama é dada pelos jogos competitivos entre jogadores. Street-Fighter\footnote{Street-Fighter: \url{https://github.com/HammadSiddiqui/Street-Fighter}} é um jogo OpenSource dessa categoria.
  \item \textbf{Tiro}: Simula um confronto forçando o posicionamento e pensamento estratégico do jogador para passar dos desafios do enrredo do jogo.
    \begin{itemize}
      \item \textbf{Tiro em primeira pessoa ou \ac{FPS}}: Utiliza uma visão em primeira pessoa, popularmente conhecida como \ac{POV}~\cite{pov_panix}. POV trata-se de uma tecnica de filmagem a qual coloca a visão como a visão geral de uma cena. Um título popular neste segmento é Battlefield 4\footnote{Battlefield: \url{https://www.battlefield.com/pt-br}}.
      \item \textbf{Tiro em terceira pessoa ou \ac{TPS}}: Essa subcategoria utiliza um ponto de visão externo, como uma terceira pessoa na cena ou posições fixas no cenário para a câmera. Max Payne 3\footnote{Max Payne 3: \url{http://store.steampowered.com/app/204100/Max_Payne_3/}} é um exemplo dessa subcategoria.
    \end{itemize}
  \item \textbf{Estratégia em tempo real ou \ac{RTS}}: Utiliza o gerenciamento de recursos obtidos de forma estratégia do jogo para gerar dificuldade. Assim como jogos de luta, a sua popularidade é maior entre jogos onde permite a competição entre jogadores online StarCraft\footnote{StartCraft: \url{https://starcraft2.com/en-us/}}.
  \item \textbf{Estratégia em turnos}: Essa categoria de jogos utiliza estratégia de movimentos unitários e atômicos para ganhar do oponente. Xadrez é um exemplo de jogo não computacional. Final Fantasy\footnote{Final Fantasy: \url{https://www.finalfantasy.com/}} é um exemplo de jogo neste segmento.
  \item \textbf{Jogos de interpretação ou \ac{RPG}}: The Elder Scrolls V\footnote{The Elder Scrools V: \url{http://store.steampowered.com/agecheck/app/489830/}}
  \item \textbf{Jogos Massivos ou \ac{MMO}}
    \begin{itemize}
      \item \textbf{\ac{MMORPG}}: Black Desert Online\footnote{Black Desert Online: \url{https://blackdesert.playredfox.com/black_desert}}
      \item \textbf{\ac{MOBA}}: DOTA 2\footnote{DOTA 2: \url{http://br.dota2.com/}}
      \item \textbf{\ac{MMOFPS}}:Battlefield Battlelog\footnote{Battlefield Battlelog: \url{http://battlelog.battlefield.com/}}
    \end{itemize}
\end{itemize}

\section{Trabalhos Relacionados}
\label{sec:similares}
