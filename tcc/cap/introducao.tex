\chapter{Introdução}
\label{introducao}


Jogos \acf{mmorpg} são utilizados como negócio viável e lucrativo, sendo que, a experiência de jogabilidade na qual o usuário final será submetido é um fator crítico para o sucesso destes jogos.
%
Este gênero, focado na interpretação de papéis de forma massiva em um ambiente compartilhado, tem como sua principal característica a comunicação e representação virtual de personagens em um mundo fantasia compartilhado.
%
Neste mundo compartilhado cada jogador pode interagir com objetos ou tomar ações sobre outros jogadores em tempo real, tendo como principal objetivo a resolução de problemas dentro de uma comunidade virtual, na qual é resolvido específico a cada projeto~\cite{video_game_technologies}.



A maioria dos jogos \ac{mmorpg} disponíveis no mercado estão implementados sobre uma arquitetura que executa o serviço sobre diversos servidores~\cite{stephenclarkewillson2017}, nos quais o desempenho dos servidores e serviços influenciam tanto na experiência de jogabilidade do usuário final, quanto no custo de manutenção destes serviços~\cite{1417630}.
%
Modelar um sistema de alto desempenho torna-se um trabalho essencial para a satisfação do usuário final neste cenário~\cite{1417630}.
%
As ocorrências geradas por um sistema de baixo desempenho podem acarretar em frustração do usuário com o serviço e/ou aumento dos gastos com recurso computacional para manter tais serviços.
%
Uma ocorrência é qualquer tipo de mal funcionamento em uma aplicação, não necessariamente aparente ao usuário final~\cite{1417630}.
%
Evitar ou eliminar as ocorrências durante o projeto de desenvolvimento das arquiteturas do serviço é um processo crítico para o sucesso desses jogos.


Os avanços tecnológicos de sistemas distribuídos estão permitindo que as pessoas utilizem serviços com volumes massivos de dados para aplicações sensíveis a latência.
%
Essa situação é propícia à área de jogos massivos, e tem formentado pesquisas deste ramo~\cite{mmo_analytic,1417630,6267019,6063041}.
%
O principal objetivo destas pesquisas é reduzir a carga e o impacto da latência para o usuário final nesses serviços, aumentando a quantidade de jogadores simultâneos em um único serviço.
%
Reduzir a carga e impacto da latência em serviços para jogos massivos resulta em uma melhor experiência de jogabilidade aos usuários finais, sendo este um dos fatores críticos para o sucesso destes serviços~\cite{1417630}.


Entende-se por arquitetura de microsserviços uma arquitetura com diversas aplicações menores na qual utilizam troca de mensagens pela rede  para implementar uma regra de negócio complexa, não exibindo necessariamente se este serviço é implementado por um ou mais microsserviços.
%
Em específico, este paradigma de desenvolvimento de arquiteturas herda características da filosofia UNIX, descrito pelo matemático Malcolm Doug McIlroy em 2003 pela seguinte passagem:

\begin{quotation}
    \textit{Write programs that do one thing and do it well. Write programs to work together. Write programs to handle text streams, because that is a universal interface.}\\
    \cite{Raymond2003Oct}
\end{quotation}

Alguns estudos discorrem sobre a escalabilidade dos serviços baseados em microsserviços, sendo que tais estudos usam o desenvolvimento \textit{web} como sua principal aplicação~\cite{photon_engine, mmorpg_culture, DiFrancesco2017Apr}.
%
Entretanto, a estrutura de um serviço \ac{mmorpg} baseado em microsserviços é mais abrangente, necessitando de maior desempenho comparado a serviços \textit{web}~\cite{photon_engine, mmorpg_culture}.
%
Dessa forma, torna-se interessante a análise de serviços \ac{mmorpg} baseados em microsserviços.


Deste modo o problema identificado é a falta de uma análise sobre o consumo de recursos para a manutenção de arquiteturas de microsserviços específicos a jogos \ac{mmorpg}, utilizando como base as arquiteturas encontradas na literatura.
%
Adicionalmente a este problema, será necessário o desenvolvimento de tais arquiteturas para submissão a experimentos e posterior análise.

Este trabalho tem como objetivo analisar o consumo de recursos das principais arquiteturas disponíveis na literatura.
%
Nesse sentido, os objetivos específicos do atual trabalho são:

\begin{itemize}
    \item Identificar arquiteturas empregadas na categoria de jogos \ac{mmorpg}.
    \item Identificar os protocolos utilizados nessas arquiteturas.
    \item Identificar os microsserviços dessas arquiteturas.
    \item Identificar e avaliar ferramentas de análise de métricas para armazenar valores dos testes.
    \item Analisar o comportamento das arquiteturas aplicadas, levantando questões de desempenho e recursos utilizados.
    \item Propor alternativas de otimização para os problemas relacionados a consumo de recursos encontrados nas devidas arquiteturas identificadas.
\end{itemize}


Para dar suporte a proposta deste trabalho, uma revisão da literatura é apresentada a fim de esclarecer os conceitos principais referentes a temática de serviços \ac{mmorpg}.
%
Ao analisar os trabalhos relacionados identificados, houve uma dificuldade para encontrar os tópicos \ac{mmorpg} e sistemas distribuídos baseado em microsserviços.
%
Os resultados obtidos deste trabalho podem contribuir com futuros trabalhos científicos em diversas áreas de desenvolvimento e engenharia de software, computação distribuída ou consumo de recursos, específicos ou não a área de desenvolvimento de jogos.



A análise é inicialmente realizada através de uma pesquisa referenciada sobre arquiteturas de microsserviços e arquitetura de serviços \ac{mmorpg}.
%
Depois, um estudo sobre a intersecção de ambos os temas.
%
Também são identificados trabalhos relacionados na presente literatura que foquem tanto em arquiteturas de  microsserviços genéricas quanto em arquiteturas de serviços \ac{mmorpg}.
%
Em seguida é definida a proposta, junto a testes e cenários a fim de coletar informações e realizar a análise das arquiteturas de microsserviços específicas a jogos \ac{mmorpg}. 



Este trabalho de conclusão de curso possui natureza aplicada pois será necessário a implementação das arquiteturas descritas na literatura, utilizando as mesmas regras de negócio, viabilizando uma análise igualitária entre as arquiteturas propostas.
%
A análise será efetuada de maneira qualitativa, pois será realizado um estudo a partir de valores gerados pelos experimentos, considerando as características individuais de cada arquitetura identificada.

Este trabalho está organizado em seis capítulos, que dão suporte a análise das arquiteturas específicas a jogos \ac{mmorpg} proposta.
%
No Capítulo~\ref{cap2} são apresentados os conceitos necessários para o entendimento desse trabalho, com a finalidade de apresentar o funcionamento básico de um cliente e serviço \ac{mmorpg}, arquitetura de um serviço baseado em microsserviços e por fim em específico algumas arquiteturas de microsserviços \ac{mmorpg} encontradas na literatura.
%
O Capítulo~\ref{cap21} aborda trabalhos relacionados encontrados na literatura, tendo como objetivo principal destacar e comparar suas características, a fim de prover fundamentos para a análise dos serviços descritos na referenciação teórica.
%
A proposta é definida no Capítulo~\ref{cap3}, a qual define o plano de captura de valores, a interpretação de tais valores e cenário a qual as arquiteturas propostas serão implantados.

O Capítulo~\ref{cap5} descreve a implementação e implantação das arquiteturas de microsserviços.
%
A partir da execução dos microsserviços implantados em uma nuvem computacional, pode-se obter dados sobre o consumo de recurso.
%
O Capítulo~\ref{cap6} junta diversos dados capturados e realiza uma análise sobre o consumo de recursos computacionais, justificando e caracterizando comportamentos das arquiteturas desenvolvidas.


