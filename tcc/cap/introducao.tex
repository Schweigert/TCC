\chapter{Introdução}
\label{cap1}

Tanto questões de segurança como de análise de desempenho de uma nuvem computacional dependem da identificação de comportamentos e operações que ocorrem durante o seu uso.
%
Percebe-se que há uma necessidade de identificar operações, muitas das quais são operações  que ocorrem no nível da camada de rede e demandam análise e compreensão do que está trafegando e sua finalidade.
%
Neste sentido, a caracterização de tráfego auxilia através do emprego de técnicas e métodos que possibilitam a coleta e identificação de forma sistematizada.
%
Apesar de não haver uma definição formal ou método para caracterização de tráfego, este termo refere-se à minuciosa análise do tráfego de uma rede, buscando entender seu comportamento e também as consequências que derivam deste comportamento.

A caracterização de tráfego em nuvens computacionais ainda é incipiente em alguns aspectos enquanto em outros recai sobre questões já amplamente estudadas.
%
Quando se trata de caracterização de tráfego de aplicações tradicionais (\textit{e.g.,} servidor web) já existem trabalhos bem detalhados \cite{arlitt:1996:web,braun:1995:web,gill:2007:youtube}.
%
Entretanto, quanto ao tráfego de controle torna-se escasso em função das especificidades de cada nuvem computacional.

Uma nuvem computacional privada opera na infraestrutura própria de uma organização, portanto, toda manutenção e cuidados com segurança são de responsabilidade dela. 
%
Em questão de uso, nuvens privadas são criadas para atender os propósitos da organização, e são acessíveis apenas por indivíduos habilitados pela organização, o que possibilita um grau maior de controle e segurança sobre os dados ali contidos \cite{Jadeja:2012:clouddeliverymodels}.

Dentre os \textit{softwares} de código aberto existentes para gerenciamento e virtualização de nuvens computacionais \ac{iaas}, que permitem a criação de nuvens privadas, destaca-se o OpenStack\footnote{\url{https://www.openstack.org}} pela sua popularidade.
%
Em relação à parte física da instalação de uma nuvem com OpenStack, a arquitetura possui três domínios com políticas de seguranças distintas: Domínio Público, Domínio de Controle e Domínio de Convidados \cite{openstack:newton}. 
%
O Domínio de Controle é uma das partes que define o desempenho de nuvens computacionais OpenStack, justamente por trafegar dados de funcionalidades essenciais ao funcionamento da nuvem.
%
Consequentemente, o estudo do tráfego neste domínio envolve identificar possíveis fatores que limitem seu funcionamento e também identificar limitações intrínsecas do software. 

Este trabalho pretende analisar e caracterizar o tráfego da rede neste domínio, levantando questões de desempenho presentes no \textbf{Domínio de Controle} com o auxílio de um sistema de monitoramento.
%
Mais especificamente, este trabalho pretende analisar o desempenho na rede de controle do OpenStack sob diferentes aspectos: comportamento dos serviços durante instanciação de \ac{vm}; classificar o tráfego recebido pelos nós controladores da nuvem; e analisar a presença e impacto de eventos periódicos.
%
Para auxiliar o processo de caracterização do tráfego, será implementado um sistema de monitoramento de tráfego para alguns serviços do OpenStack.

Como estudo de caso, será realizado o monitoramento da rede de controle de uma nuvem computacional baseada em OpenStack. 
%
O tráfego coletado servirá para caracterizar o comportamento de certas funcionalidades do OpenStack na rede de controle, e como elas influenciam no desempenho desta rede.
%
Em relação à este trabalho, os objetivos específicos são:

\begin{itemize}
	\item Identificar e analisar métodos mais adequados à caracterização de tráfego, sob a ótica de nuvens computacionais;
	\item Identificar e analisar abordagens para o monitoramento de tráfego que possibilitem monitorar redes de controle em nuvens computacionais baseadas em OpenStack;
	\item Especificar uma ferramenta para monitoramento do tráfego da rede de controle em nuvens computacionais baseadas em OpenStack, para auxiliar na sua caracterização;
	\item Implementar um sistema de monitoramento para redes de controle em nuvens computacionais baseadas em OpenStack;
	\item Aplicar o sistema de monitoramento implementado em uma rede de controle de uma nuvem computacional baseada em OpenStack; e
	\item Analisar e caracterizar o comportamento de funcionalidades selecionadas no tráfego da rede, levantando questões de desempenho.
\end{itemize}

Foi adotada a pesquisa aplicada como método de pesquisa neste trabalho. 
%
Dessa forma, na primeira parte (TCC-I) do trabalho foi realizado levantamento bibliográfico, a análise de trabalhos relacionados, a definição da arquitetura de funcionamento do sistema de monitoramento e a definição dos experimentos que serão realizados.
%
Na segunda parte (TCC-II) do trabalho será realizado um estudo de caso aplicando o sistema de monitoramento implementado na nuvem TCHE, localizada no LabP2D do \mbox{CCT/UDESC}, com o objetivo de coletar dados para realizar a caracterização do tráfego da rede de controle da nuvem em questão.


O trabalho organiza-se da seguinte forma.
%
O Capítulo \ref{cap2} apresenta a fundamentação teórica sobre o tema proposto, contendo uma introdução à computação em nuvem e virtualização, e conceitos básicos sobre software de gerenciamento de nuvem OpenStack.
%
Após, são apresentadas abordagens para efetuar a caracterização de tráfego.
%
Com todas as definições básicas feitas, é apresentado o problema abordado neste TCC, bem como trabalhos relacionados.
%
O Capítulo \ref{cap3} contém a proposta para realização da caracterização de tráfego.
%
Para tal, explica-se em maior detalhes os serviços do OpenStack que este trabalho tem interesse ao caracterizar o tráfego.
%
Com estas características definidas, detalha-se a proposta do sistema de monitoramento de tráfego e como será realizada a análise, e então define-se como serão realizados os experimentos na qual a proposta será aplicada.
%
Por fim, o Capítulo \ref{cap:conclusao} apresenta as considerações desta etapa (TCC-I) e estabelece as metas que serão desenvolvidas nos trabalhos futuros.


