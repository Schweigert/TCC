\chapter{Introdução}
\label{introducao}



Os avanços tecnológicos de sistemas distribuídos estão permitindo que as pessoas utilizem serviços com volumes massivos de dados para aplicações sensíveis a latência.
%
Essa situação é propícia à área de jogos massivos, e tem atraído pesquisadores para essa área de estudo~\cite{mmo_analytic,1417630,6267019,6063041}.
%
O principal objetivo destas pesquisas é reduzir a carga e o impacto da latência para o usuário final nesses serviços, aumentando a quantidade de jogadores simultâneos em um único serviço.
%
Reduzir a carga e impacto da latência em serviços para jogos massivos resulta em uma melhor experiência de jogabilidade aos usuários finais, sendo este um dos fatores críticos para o sucesso destes serviços~\cite{1417630}.



Jogos \ac{mmorpg} são utilizados como negócio viável e lucrativo, sendo que experiência de jogabilidade na qual o usuário final será submetido é um fator crítico para o sucesso.
%
Neste gênero, especificado de interpretação de papéis de forma massiva em um ambiente compartilhado, tem como sua principal característica a comunicação e representação virtual de personagens em um mundo fantasia compartilhado no qual cada jogador pode interagir com objetos ou tomar ações sobre outros jogadores em tempo real, tendo como principais objetivos a resolução de problemas conforme a sua regra de negócio de cada projeto~\cite{video_game_technologies}.



A maioria dos jogos \ac{mmorpg} disponíveis no mercado estão implementados sobre uma arquitetura que executa o serviço sobre diversos servidores~\cite{stephenclarkewillson2017}, nos quais o desempenho destes servidores influencia tanto na experiência de jogabilidade do usuário final, quanto no custo de manutenção destes serviços~\cite{1417630}.
%
Modelar um sistema de alto desempenho torna-se um trabalho essencial para a satisfação do usuário final neste cenário~\cite{1417630}.
%
As ocorrências geradas por um sistema de baixo desempenho podem acarretar em frustração do usuário com o serviço e/ou aumento dos gastos com recurso computacional para manter o serviço.
%
Uma ocorrência é qualquer tipo de mal funcionamento em uma aplicação, não necessariamente aparente ao usuário final~\cite{1417630}.
%
Evitar ou eliminar as ocorrências durante o projeto e desenvolvimento das arquiteturas do serviço é um processo crítico para o bom funcionamento desses jogos.



Algumas análises/propostas discorrem sobre a escalabilidade dos serviços baseados em microsserviços com o seu foco para desenvolvimento \textit{web}~\cite{photon_engine, mmorpg_culture}.
%
Entretanto, a estrutura de um serviço \ac{mmorpg} baseado em microsserviços é bem mais abrangente, necessitando de maior desempenho comparado a serviços \textit{web}~\cite{photon_engine, mmorpg_culture}.
%
Dessa forma, torna-se interessante a análise de serviços \ac{mmorpg} baseados em microsserviços.



Portanto o problema principal identificado é a análise de consumo de recursos para a manutenção de arquiteturas de microsserviços específicos a jogos \ac{mmorpg} utilizando como base arquiteturas encontradas na literatura.
%
Adicionalmente a este problema, será necessário o desenvolvimento de tais arquiteturas para submissão a experimentos e análise.


%ccm <-objetivos aqui

Para dar suporte a proposta deste trabalho, uma revisão da literatura foi apresentada, a fim de esclarecer os conceitos principais referentes a temática de serviços \ac{mmorpg}.
%
Ao analisar os trabalhos relacionados identificados, houve uma dificuldade para encontrar os tópicos \ac{mmorpg} e sistemas distribuídos baseado em microsserviços.
%
A relevância desta análise contribui, diretamente para que outros trabalhos científicos possam utilizar da análise de arquiteturas para jogos \ac{mmorpg} baseados em arquiteturas de microsserviços.






\section{Objetivos}

%ccm Objetivo

Os objetivos específicos deste trabalho são:


%ccm esse são as etapas, não os objetivos específicos
\begin{enumerate}
	\item \textbf{Levantamento e fichamento das referências:} Pesquisa de fontes para embasamento teórico do trabalho, com base nos objetivos específicos;
	      
	\item \textbf{Consolidação das referências:} Compreensão e seleção de artefatos literários que permitam atingir o objetivo do Trabalho de Conclusão de Curso I;
	      
	\item \textbf{Identificação e definição de arquiteturas descritas na literatura:} Enumeração e definir das arquiteturas de microsserviços descritas na literatura, bem como os seus objetivos;
	      
	\item \textbf{Especificação das arquiteturas selecionadas:} Especificar o funcionamento das arquiteturas selecionadas.
	      
	\item \textbf{Identificação e definição de experimentos aplicáveis ao teste:} Eleger e definir a simulação de clientes a ser aplicada nos testes;
	      
	\item \textbf{Especificação do experimento elegida:} Especificar os requisitos;
	      
	\item \textbf{Desenvolvimento da simulação:} Desenvolvimento da simulação para interagir com as arquiteturas de microsserviços;
	      
	\item \textbf{Desenvolvimento da arquitetura:} Desenvolvimento da arquitetura para executar os testes;
	      
	\item \textbf{Implantação das arquiteturas selecionadas na pesquisa referênciada:} Aplicação das arquiteturas descritas sobre uma nuvem computacional;
	      
	\item \textbf{Realização dos testes utilizando a simulação elegida na pesquisa referênciada:} Execução de testes da arquitetura desenvolvida sobre a nuvem computacional;
	      
	\item \textbf{Análise das arquiteturas testadas:} Analisar as métricas obtidas dos testes e descrever resultados, identificando possíveis gargalos nas arquiteturas;
	      
	\item \textbf{Otimização para melhorar as métricas obtidas:} Identificar pontos de gargalo nos microsserviços identificados e propor soluções viáveis para aumentar o desempenho desses sistemas.
	      
\end{enumerate}



\section{Metodologia}



A analise será inicialmente realizada através de um pesquisa referenciada sobre arquiteturas de microsserviços e arquitetura de serviços \ac{mmorpg}.
%
Depois, um estudo sobre a intersecção de ambos os temas.
%
Também serão identificados trabalhos relacionados na presente literatura que foquem tanto em microsserviços tal como serviços \ac{mmorpg}.
%
Em seguida será definida a proposta, junto a testes e cenários a fim de analisar tais arquiteturas.



Este trabalho de conclusão de curso possui natureza aplicada pois será necessário a implementação das arquiteturas descritas na literatura, utilizando as mesmas regras de negócio, viabilizando uma análise igualitária entre as arquiteturas propostas.
%
A análise será abordada de maneira qualitativa, pois será feito um estudo a partir de valores gerados por experimentos utilizando as ferramentas, abordando as características de tais arquiteturas.

\section{Organização do trabalho}


Este trabalho está organizado em cinco capítulos, que dão suporte a análise das arquiteturas específicas a jogos \ac{mmorpg} proposta.
%
No Capítulo~\ref{cap2} são apresentados os conceitos necessários para o entendimento desse trabalho, com a finalidade de apresentar o funcionamento básico de um cliente e serviço \ac{mmorpg}, arquitetura de um serviço baseado em microsserviços e por fim em específico algumas arquiteturas de microsserviços \ac{mmorpg} encontradas na literatura.
%
No final do Capítulo~\ref{cap2}, são abordados trabalhos relacionados encontrados na literatura, tendo como objetivo principal destacar e comparar suas características, a fim de prover fundamentos para a análise dos serviços descritos na referenciação teórica.
%
A proposta é definida no Capítulo~\ref{cap3}, a qual define o plano de extração de valores, a interpretação de tais valores e cenário a qual as arquiteturas propostas serão implantados.
%
%A conclusão definida no Capítulo~\ref{cap:conclusao} levanta algumas considerações finais do trabalho desenvolvido até o momento, bem como indica as atividades concluídas, em andamento e próximas atividades deste trabalho.
