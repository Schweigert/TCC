 
\chapter{Desenvolvimento das arquiteturas propostas}
\label{cap5}



Após o levantamento e fichamento dos requisitos para execução dos testes sobre as arquiteturas Rudy, Salz e Willson,
é notório a necessidade de exemplificar o funcionamento destas arquiteturas para jogos \ac{mmorpg}.
Nesse sentido, o capítulo atual irá levantar as devidas considerações sobre o desenvolvimento prático destas arquiteturas.



A explicação técnica dos projetos estão organizadas da seguinte forma para cada serviço desenvolvido das arquiteturas:



\begin{enumerate}
    \item Linguagem de Programação e bibliotecas: Exibe pontos a cerca das linaguagens de programação e quais módulos ou bibliotecas foram utilizados para o seu desenvolvimento.
    \item Protocolos Utilizados: Mostra quais protocolos estes serviços utilizam para troca de mensagens em seu ecossistema. Será mostrado utilizando uma tabela de adjacência, tal qual utilizado para relação entre nós em um Grafo.
    \item Processamento ativo interno: Demonstra qual o processamento ativo no serviço independente das requisições de seus microsserviços vizinhos ou cliente.
\end{enumerate}



Ao final será levantado o funcionamento do cliente das arquiteturas Rudy (Subseção \ref{rudy}), Salz (Subseção \ref{salz}) e Willson (Subseção \ref{willson}).
%
Nesta etapa será exibido o fluxo de mensagens a partir das ações realizadas pelo cliente.
%
Nesse sentido este capítulo deve abrangir todos os aspéctos técnicos da aplicação, levando em consideração o seu fluxo de funcionamento e integração entre os microsserviços no ecossistema da arquitetura.

\section{Arquitetura Rudy}
\label{sec:arc_rudy}

A arquitetura Rudy (Subseção \ref{rudy}), a primeira arquitetura a ser desenvolvida, teve o seu funcionamento reduzido aos microsserviços básicos para o funcionamento do Gerente de Mundo.
%
Nesse sentido, os microsserviços implementados para esta arquitetura foram:

\begin{enumerate}
    \item Serviço de Jogo: Ou \textit{rudygh}.
    \item Gerenciador de Consultas: Ou \textit{rudydb}.
    \item Serviço de Autenticação: Ou \textit{rudya}.
    \item Serviço Web Dinâmico: Ou \textit{rudyweb}.
\end{enumerate}



Além destes microsserviços, a arquitetura utiliza os serviços PostgreSQL e Redis, ambas de código aberto.
%
Tais serviços são utilizados respectivamente como banco de dados permanente e banco de dados em memória cache para autenticação.
%
Este método de autenticação foi replicado para as demais arquiteturas.



Em relação aos protocolos utilizados na arquitetura Rudy, temos dois serviços RPC e dois serviços HTTP.
%
A relação detalhada é exibida na Tabela~\ref{tab:protocolos_rudy}.



\begin{table}[htb!]
    \centering
    \caption{Protocolos dos microsserviços da arquitetura Rudy.}
    \label{tab:protocolos_rudy}
    \begin{tabular}{|l|l|l|l|}
    \hline
    Microsserviço & Linguagem de Programação    & Porta & Protocolo \\ \hline
    rudygh        & Golang 1.11 / RPC Nativo    & 3000  & TCP       \\ \hline
    rudydb        & Golang 1.11 / Gin Framework & 3000  & HTTP      \\ \hline
    rudya         & Golang 1.11 / RPC Nativo    & 3000  & TCP       \\ \hline
    rudyweb       & Golang 1.11 / Gin Framework & 3000  & HTTP      \\ \hline
    \end{tabular}
    
    Fonte: O próprio autor.
\end{table}

Além de quais protocolos, a Tabela~\ref{tab:protocolos_rudy} exibe qual tecnologia está utilizando para manter o serviço.
%
Tanto nos microsserviços da arquitetura Rudy quanto aos demais microsserviços implementados para este TCC foram escritos na linguagem Go.
%
Ambos os serviços compartilham dos mesmos modelos de padrão \ac{mvc} tanto para as aplicações \ac{rpc} e \ac{http}.

Para o desenvolvimento dos microsserviços da arquitetura Rudy, foram utilizados os seguintes pacotes externos da linguagem:

\begin{itemize}
    \item Gin: Framework focado em desenvolvimentos de \ac{api} para web escrito em Golang.
    \item Redis: Pacote de conexão sobre \ac{tcp} ou \ac{udp} em um serviço Redis.
    \item Protofub: Pacote que implementa serialização de dados estruturados de modo mínimo. É utilizado para minimizar custo de chamadas \ac{rpc} em Go.
    \item Gorm: Golang \ac{orm} que permite a conexão com o banco de dados Postgres sem utilizar \ac{sql}.
    \item Graphite: Pacote que permite a conexão com o servidor de logs Graphite para enviar métricas.
    \item Gorequest: Pacote para estruturar requisições Web utilizado para comunicação interna entre serviços Web utilizando assinatura \ac{jwt}.
    \item Testify: Suíte de testes para manter o funcionamento íntegro durante a implementação da aplicação. Em específico, a aplicação possúi 90\% de convergência de código, a qual garante a integridade de futuras alterações.
\end{itemize}

Todas as bibliotecas utilizadas são distribuídas com modelo \it{Open Source}, pelas licenças \it{MIT}, \it{BSD} e \it{GPLv3}.
%
Estas mesmas bibliotecas são utilizadas nas demais arquiteturas.