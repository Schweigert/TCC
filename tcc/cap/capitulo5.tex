\chapter{Implementação}
\label{cap5}

Este capítulo visa expressar o funcionamento prático dos microsserviços implementados.
%
Isto é necessário, tal que a escolha da linguagem, tecnologias utilizadas e boas práticas adotadas em seu desenvolvimento interferem a qualidade do serviço desenvolvido.

Devido a preocupação das bibliotecas disponíveis para utilização e as linguagens de programação que permitem utilizar tais tecnologias,
a Seção~\ref{sec:tecnologias} descreve a necessidade tecnológica necessária para o desenvolvimento. A partir desta demanda funcional, mostra-se qual linguagem de programação utilizar e por sua vez, o conjunto de bibliotecas ou serviços externos que derivam de tal escolha.
%
Não derivado diretamente da linguagem escolhida, porém tendo uma correlação forte desta escolha, a Seção~\ref{sec:tecnologias} também aborda o conjunto de serviços externos utilizados para facilitar o desenvolvimento e implantação dos microsserviços.

Por fim, para contextualizar o funcionamento dos microsserviços operando em rede, a Seção~\ref{sec:interconexao} visa descrever a interconexão dos microsserviços implementados.

\section{Tecnologias Utilizadas}
\label{sec:tecnologias}

\subsection{Linguagem de programação}

Golang

\subsection{Regra de Negócios}

\subsection{Bibliotecas}

Gin, \ac{rpc}, gorm, metricas, mt coisa aqui

\subsection{Serviços Externos}

Docker, Testify

TravisCI, Coveralls, Dockerhub, Github

\section{Interconexão entre os microsserviços}
\label{sec:interconexao}
