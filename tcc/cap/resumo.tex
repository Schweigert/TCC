A crescente popularização de jogos \acf{mmorpg} demanda por novas abordagens tecnológicas, a fim de suprir as necessidades dos usuários com menor custo de recursos computacionais.
%
Projetar essas arquiteturas, do ponto de vista da rede, é pertinente e impactante para o sucesso desses jogos.
%
Este trabalho realiza uma análise voltada a identificação dos recursos computacionais consumidos pelas arquiteturas de microsserviços Rudy, Salz e Willson, na qual são arquiteturas de microsserviços elaboradas para jogos \ac{mmorpg}.
%
Essa análise é executada após a realização de uma pesquisa referenciada, seguida de uma análise das principais arquiteturas e a execução de testes utilizando clientes automatizados sobre as arquiteturas implantadas em uma nuvem computacional para auxiliar na identificação de gargalos de recursos.
%
A análise concluiu que as três arquiteturas obtiveram êxito em seus papeis e suas características disponíveis na literatura puderam ser validadas.
%
Outro conclusão trata do desempenho, do ponto de vista de tempo de resposta, que identificou-se uma relação à melhor utilização da \ac{cpu}, seja pelo serviço de armazenamento ou de processamento de dados.
%
Os resultados obtidos são pertinentes para a elaboração de jogos \ac{mmorpg}, relatando características e observações acerca dos recursos consumidos, comparando as arquiteturas conforme o tempo de resposta e o comportamento dos recursos alocados.
