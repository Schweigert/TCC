A crescente popularização de jogos \acf{mmorpg} demanda por novas abordagens tecnológicas, a fim de suprir as necessidades dos usuários com menor custo de recursos computacionais.
%
Projetar essas arquiteturas, do ponto de vista da rede, é pertinente e impactante para o sucesso desses jogos.
%
O objetivo deste trabalho é realizar uma análise voltada a identificar os recursos computacionais consumidos pelas arquiteturas de microsserviços Rudy, Salz e Willson, na qual são arquiteturas de microsserviços elaboradas para jogos \ac{mmorpg}.
%
Esse objetivo será atingido após realizar uma pesquisa referenciada, seguida de uma análise das principais arquiteturas e, preferencialmente, a execução de testes utilizando clientes automatizados sobre as arquiteturas implantadas em uma nuvem computacional para auxiliar na identificação de gargalos de recursos. 
%
Espera-se que os resultados obtidos possam auxiliar provedores de serviços \ac{mmorpg} a reduzir gastos de manutenção e melhorar a qualidade de tais serviços. \\
