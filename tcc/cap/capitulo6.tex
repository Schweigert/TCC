\chapter{Implantação}
\label{cap6}

A implantação refere-se ao modo como os microsserviços foram executados, levando em conta o seu ambiente, caracteristicas da rede, serviços auxiliares e modelo de automatização utilizado.
%
Neste modelo pode-se também adicionar o modo de armazenamento e obtenção dos dados da execução dos testes.

A fim de permitir a reprodução do experimento em um ambiente com as mesmas características, a Seção~\ref{sec:ambiente} descreve o ambiente, segregando em sub-redes e suas interconexões.
%
Existe uma física e lógica entre as redes utilizadas, visto que a quantidade de recurso computacional para os testes é elevada, faz necessário segregar sub-redes para organizar conjuntos computacionais para serviço (Sessão~\ref{sec:ambiente_mic}) , clientes (Sessão~\ref{sec:ambiente_cli}) e armazenamento de métricas (Sessão~\ref{sec:ambiente_met}).

Devido ao modelo de implantação utilizando contêineres, pode-se utilizar ferramentas adequadas que auxiliam na execução dos testes.
%
Neste sentido, a Seção~\ref{sec:servicos_aux} visa descrever os serviços externos utilizados durante a execução dos testes.


\section{Ambiente}
\label{sec:ambiente}

O ambiente de testes foi dividido em dois sub-ambientes, a fim de controlar a conexão entre o cliente e a arquitetura de microsserviços fisicamente. Dessa forma, obriga-se a utilização da rede para a conexão entre ambos.

O ambiente de microsserviços foi implantado em um ambiente virtualizado sobre uma nuvem OpenStack.
%
Esta implantação é detalhada na Subseção~\ref{sec:ambiente_mic}.

O ambiente de clientes foi implantado em um laboratório de computadores, sobre o sistema de gestão de contêineres Docker Swarm.
%
Esta implantação é detalhada na Subseção~\ref{sec:ambiente_cli}.

Durante a execução dos testes, ambos os ambientes  de microsserviços e enviam dados ao serviço de métricas, a qual foi implantado sobre uma nuvem OpenStack.
%
Esta implantação é detalhada na Subseção~\ref{sec:ambiente_mic}.

\subsection{Ambiente de Microsserviços}
\label{sec:ambiente_mic}

\subsection{Ambiente de Clientes}
\label{sec:ambiente_cli}

\subsection{Ambiente de Metricas}
\label{sec:ambiente_met}

\section{Serviços Auxiliares}
\label{sec:servicos_aux}
DockerHub
