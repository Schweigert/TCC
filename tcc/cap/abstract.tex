Cloud computing management softwares are popular, having both deployment scalability and data security as their main characteristics.
%
Among all available private cloud management softwares, OpenStack stands among the most popular ones.
%
However, like in any other cloud management software, performance is considered one of the most important metrics. 
%
In this context, most researches focus only on analysing the cloud section visible to users, relinquishing the internal operations and behavior of the cloud provider.
%
OpenStack has a network group specifically for management traffic, named Management Domain, which contains all the operational and administrative tasks of the cloud.
%
However, there is a lack of information regarding how both user generated tasks (\textit{i.e., } \ac{vm} initialization) and periodic tasks (\textit{i.e.,} update active \ac{vm} list) may impact the behavior of the Management Domain in OpenStack clouds.
%
This work aims to understand the OpenStack management network behavior better, which is a network contained inside the Management Domain.
%
We will achieve it by characterizing the management traffic through an analysis approach using the data provided by a network monitoring software.
%
To reach the set goal, we start studying traffic measurement / analysis approaches, as well as OpenStack software focusing on its deployment architecture.
%
Thus, we specify, design, and deploy a traffic monitoring system, which will be responsible for generating data about the traffic in the management network.
%
Finally, the traffic analysis will focus on key OpenStack services, using the data generated by the traffic monitoring system as input.