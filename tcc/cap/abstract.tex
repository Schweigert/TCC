The popularization increase of \acf{mmorpg} demands new technological approaches in order to supply the requirements of users with lower cost of computational resources.
%
Designing these architectures, from the network point of view, is relevant and impacting to the success of these games.
%
This work analyzes the identification of computational resources consumed by the microservices architectures Rudy, Salz and Willson, in which they are microservices architectures elaborated for \ac{mmorpg} games.
%
This analysis is performed after conducting a referenced survey, followed by an analysis of key architectures and testing using automated clients on the architectures deployed in a computational cloud to assist in identifying resource bottlenecks.
%
The analysis concluded that the three architectures were successful in their roles and their characteristics available in the literature could be validated.
%
Another conclusion concerns the performance, from the point of view of response time, which identified a relation to the best utilization of \ac{cpu}, either by the storage or data processing service.
%
The obtained results are pertinent to the elaboration of \ac{mmorpg} games, reporting characteristics and observations about the consumed resources, comparing the architectures according to the response time and the allocated resources behavior.
%
