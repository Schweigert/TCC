\listofabbreviations{Lista de Abreviaturas}
\begin{acronym}[]
	\acro{amqp}[AMQP]{{\it Advanced Message Queuing Protocol}}
	\acro{api}[API]{{\it Application Programming Interface}}
  \acro{aws}[AWS]{{\it Amazon Web Services}}
	\acro{cli}[CLI]{{\it Command Line Interface}}
	\acro{crud}[CRUD]{{\it Create Read Update Delete}}
	\acro{cpu}[CPU]{{\it Central Processing Unit}}
	\acro{cs}[C/S]{{Cliente/Servidor}}
	\acro{ddos}[DDoS]{{\it Distributed Denial of Service}}
	\acro{fps}[FPS]{{\it First-person shooter}}
	\acro{http}[HTTP]{{\it Hypertext Transfer Protocol}}
	\acro{iaas}[IaaS]{{\it Infrastructure as a Service}}
	\acro{ide}[IDE]{{\it Integrated Development Environment}}
	\acro{idl}[IDL]{{\it Interface Description Language}}
  \acro{ids}[IDS]{{\it Intrusion Detection System}}
	\acro{json}[JSON]{{\it JavaScript Object Notation}}
	\acro{jwt}[JWT]{{\it JSON Web Token}}
	\acro{kvm}[KVM]{{\it Kernel-based Virtual Machine}}
	\acro{labp2d}[LabP2D]{{Laboratório de Processamento Paralelo e Distribuído}}
	\acro{lan}[LAN]{{\it Local Area Network}}
  \acro{ldap}[LDAP]{{\it Lightweight Directory Access Protocol}}
	\acro{mhz}[MHz]{{\it Mega-hertz}}
	\acro{mmo}[MMO]{{\it Massively Multiplayer Online}}
	\acro{mmofps}[MMOFPS]{{\it Massively Multiplayer Online First-Person Shooter}}
	\acro{mmorpg}[MMORPG]{{\it Massively Multiplayer Online Role-Playing Game}}
	\acro{moba}[MOBA]{{\it Multiplayer Online Battle Arena}}
	\acro{mvc}[MVC]{{\it Model-View-Controller}}
	\acro{nist}[NIST]{{\it National Institute of Standards and Technology}}
	\acro{nosql}[NoSQL]{{\it Not Only SQL}}
	\acro{npcs}[NPCs]{{\it Non-Playable Characters}}
	\acro{ntsc}[NTSC]{{\it National Television System Committee}}
	\acro{p2p}[P2P]{{\it Peer-to-Peer}}
	\acro{pvp}[PvP]{{\it Player vs Player}}
	\acro{pvnpc}[PvNPCs]{{\it Player vs \ac{npcs}}}
	\acro{paas}[PaaS]{{\it Platform as a Service}}
	\acro{pov}[POF]{{\it Point of View}}
	\acro{qos}[QoS]{{\it Quality of Service}}
	\acro{ram}[RAM]{{\it Random Access Memory}}
	\acro{rest}[REST]{{\it Representational State Transfer}}
	\acro{rpc}[RPC]{{\it Remote Procedure Call}}
	\acro{rpg}[RPG]{{\it Role-Playing Game}}
	\acro{rts}[RTS]{{\it Real-Time Strategy}}
	\acro{sdn}[SDN]{{\it Software Defined Network}}
	\acro{saas}[SaaS]{{\it Software as a Service}}
	\acro{snmp}[SNMP]{{\it Simple Network Management Protocol}}
	\acro{sql}[SQL]{{\it Structured Query Language}}
	\acro{tcp}[TCP]{{\it Transmission Control Protocol}}
	\acro{tps}[TPS]{{\it Third-person Shooter}}
	\acro{tia}[TIA]{{\it Television Interface Adapter}}
	\acro{udp}[UDP]{{\it User Datagram Protocol}}
	\acro{vlan}[VLAN]{{\it Virtual Local Area Network}}
	\acro{vm}[VM]{{\it Virtual Machine}}
	\acro{vpn}[VPN]{{\it Virtual Private Network}}
	\acro{wan}[WAN]{{\it Wide Area Network}}
	\acro{ws}[WS]{{\it Web Services}}
	\acro{xdr}[XDR]{{\it External Data Representation}}
	\acro{xml}[XML]{{\it Extensible Markup Language}}



	\acrodefplural{vpn}[VPNs]{{\it Virtual Private Networks}}
	\acrodefplural{vlan}[VLANs]{{\it Virtual Local Area Networks}}
	\acrodefplural{vm}[VMs]{{\it Virtual Machines}}
\end{acronym}

% Defining: \acro{acronym}[short name]{full name}
% Usaging:
% \ac{acronym}     -- writes the full name followed by the acronym in brackets; later calls will write only the acronym
% \acf{acronym}     -- writes the full name followed by the acronym in brackets
% \acs{acronym}     -- writes the short name only
% \acl{acronym}     -- writes the full name only
% Use p at the end of previous commands for plural form (e.g., \acp for the plural form of \ac)
% \acresetall        -- reset usage of all acronyms (i.e., \ac will print full name again)
% \acused                -- mark the acronym as used
