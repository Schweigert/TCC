\chapter{Proposta para análise de arquiteturas \ac{mmorpg}}
\label{cap3}

%ccm O quê precisa ter na contextualização:
% 1- ressaltar o problema de não poder comparar as arquiteturas de maneira equivalente, qual o problema disso? que é afetado
% 2- resolvendo o problema, o quê espera-se ganhar?

Este trabalho propõe a análise do consumo de recursos de arquiteturas de microsserviços, a fim de contribuir com a redução do consumo de recursos de arquiteturas de jogos \ac{mmorpg}.
% ccm desenvolver um pouco mais, explicar (o quê não como)

%ccm organização do capítulo

\section{Proposta}

%ccm
% objetivos
% Definição das métricas para os itens levantados nas Tabelas 2.4 e 2.5

\section{Critérios de análise}

%ccm
% Identificar como os valores obtidos pelas métricas devem ser interpretados
% pode ser individdual ou em grupos, defiir a finalidade? Custo? Desempenho?

\section {Plano de testes}
%ccm Genérico/abstratpo
% Quais experimentos se reão realizados, se não necessário explicitar os caso de uso e o quê espera-se de resultado.

Para realizar a análise proposta sobre as arquiteturas descritas na referência teórica, se faz necessário obter coletar/medir métricas de recursos consumidos por tal arquitetura.
%
Nesse sentido, faz-se necessário a descrição específica da arquitetura analisada, equivalente as arquiteturas descritas no Capítulo~\ref{cap2}.
%
Para este fim, este trabalho irá analisar o consumo de recursos como:

\begin{enumerate}
  \item{Memória: Analisar a quantia de memória utilizada referente a cada arquitetura, analisando possíveis rotinas que consomem recursos de forma inapropriada;}
  \item{Banda de rede: Analisar o comportamento da arquitetura em relação ao consumo de banda, relevando a topologia da arquitetura;}
  \item{Processamento: Analisar rotinas de processamento de baixo desempenho;}
  \item{Tempo de resposta: Analisar o tempo de resposta para que as ações sejam computadas, a partir da chamada do cliente, pelo número de conexões no serviço;}
  \item{Limite de conexões: Analisar o limite de conexões de determinada arquitetura;}
\end{enumerate}

A fim de garantir a confiabilidade dos dados coletados, cada microsserviço da arquitetura executará de forma isolada, a nível do sistema operacional.
%
Pode-se garantir este isolamento utilizando técnicas de virtualização.
%
Analisando os dados obtidos, espera-se encontrar gargalos em tais arquiteturas, provendo possíveis soluções para os gargalos encontrados.
%
Outros critérios também serão abordados na análise, tais como a estabilidade de operação, capacidade de processamento, custo de operação e limite de conexão das arquiteturas.
%

Entretanto, as arquiteturas de microsserviços para jogos \ac{mmorpg} são aplicações de alta complexidade, a qual sofrem alterações conforma as regras de negócio impostas ao jogo em específico.
%
Por este motivo, torna-se necessário descrever as suas funcionalidades específicas, a fim de suprir comfiabilidade na análise das arquiteturas, descrevendo as especificações de implementação sobre as arquiteturas Rudy, Willson e Salz.

\section{Proposta}

O atual trabalho propôe uma análise de consumo de recursos a arquiteturas de jogos \ac{mmorpg}, em específico para as arquiteturas Rudy (~\ref{rudy}), Willson (~\ref{willson}) e Salz (~\ref{salz}).

Esta análise contribuirá a melhorar técnicas de desenvolvimento de jogos massivos, reduzindo custos de manutenção de tais serviços.


\section{Critérios de análise}

A fim de guiar um padrão para todas as arquiteturas analisadas, será utilizado um padrão de regra de negócios para os serviços desenvolvidos.
%
Este padrão seguirá as seguintes regras de negócio:

\begin{enumerate}
  \item Ambiente tridimenssional, com posicionamento bidimenssional: O ambiente de jogo será computado em 3D dimenssões, porém os personagens são fixos em um eixo.
  \item Troca de mensagens locais e globais: O serviço deve permitir a troca de mensagens com os demais personagens ao seu redor, dentro do seu campo de visão, e com todos os personagens de um mundo.
  \item Movimentação no ambiente: O serviço deve permitir o personagem a movimentar-se da sua posição atual para outra posição.
  \item Realizar ações no ambiente: O serviço deve permitir o usuário a interagir com itens, \ac{npcs} e outros jogadores.
\end{enumerate}

\subsection{Ambiente Tridimenssional}
\subsection{Troca de mensagens locais e globais}
\subsection{Realizar ações no ambiente}

\section{Plano de testes}

Descrever como cenários, critérios e objetivos de testes.
