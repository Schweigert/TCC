\chapter{Proposta para análise de arquiteturas \ac{mmorpg}}
\label{cap3}

Este trabalho propõe a análise do consumo de recursos de arquiteturas de microsserviços, a fim de contribuir com a redução do consumo de recursos de arquiteturas de jogos \ac{mmorpg}.
%
Para realizar a análise proposta sobre as arquiteturas descritas na referência teórica, se faz necessário obter coletar/medir métricas de recursos consumidos por tal arquitetura.
%
Nesse sentido, faz-se necessário a descrição específica da arquitetura analisada, equivalente as arquiteturas descritas no Capítulo~\ref{cap2}. 
%
Para este fim, este trabalho irá analisar o consumo de recursos como:

\begin{enumerate}
  \item{Memória}
  \item{Banda de rede}
  \item{Processamento}
  \item{Tempo de resposta}
\end{enumerate}

A fim de garantir a confiabilidade dos dados coletados, cada microsserviço da arquitetura executará de forma isolada, a nível do sistema operacional.
%
Pode-se garantir este isolamento utilizando técnicas de virtualização.
%
Analisando os dados obtidos, espera-se encontrar gargalos em tais arquiteturas, provendo possíveis soluções para os gargalos encontrados.
%
Outros critérios também serão abordados na análise, tais como a estabilidade de operação, capacidade de processamento, custo de operação e limite de conexão das arquiteturas.
%

Entretanto, as arquiteturas de microsserviços para jogos \ac{mmorpg} são aplicações de alta complexidade, a qual sofrem alterações conforma as regras de negócio impostas ao jogo em específico.
%
Por este motivo, torna-se necessário descrever as suas funcionalidades específicas, a fim de reduzir incertezas entre a comparação das arquiteturas, e descrever as especificações de implementação de tais requisitos, sobre as arquiteturas NAME1, NAME2, NAME3.


\section{Proposta}



\section{Critérios de análise}

Descrever de forma funcional, o que a arquitetura deverá ter.

Descrever com diagramas, fluxogramas, etc.

\section{Plano de testes}

Descrever como cenários, critérios e objetivos de testes.