\chapter{Proposta para análise de arquiteturas de microsserviços \ac{mmorpg}}
\label{cap3}

Para realizar a análise de consumo de recursos de uma arquitetura de jogos \ac{mmorpg} é necessário coletar / medir os recursos utilizados para posterior análise.
%
Na fase de coleta de informação, é possível utilizar ferramentas existentes para coletar informações sobre o consumo da rede (\textit{e.g.}, Tcpdump, Wireshark, etc.) e consumo de recursos computacionais do serviço (e.g., Golang Flame, Ruby Thread profiler tool, Golang profiler, \textit{etc.})
%
Estas ferramentas contribuem com a análise, sendo de extrema importância para a coleta de dados.



Contudo, se faz necessário uma arquitetura de microsserviços para jogos \ac{mmorpg} a fim de ser utilizado para análise.
%
Esta arquitetura será descrita na Seção~\ref{sec:arquitetura_proposta}.


\section{Arquitetura proposta para análise}
\label{sec:arquitetura_proposta}

Esta seção tem como objetivo descrever a arquitetura implementada para análise, baseada na arquitetura Willson~\cite{stephenclarkewillson2017}.
%
Esta mesma arquitetura de microsserviços é utilizada no jogo NCSoft Guild Wars 2\footnote{NCSoft Guild Wars 2: \url{https://www.guildwars2.com/}}.

Em uma visão macro, o serviço terá os seguintes microsserviços:

\begin{enumerate}
  \item{\textit{Account Service}}: Responsável por criar, alterar e autenticar as contas dos jogadores e seus personagens.
  \item{\textit{Proxy Service}}: Receberá requisições e distribuirá o mesmo ao microsserviço de interesse.
  \item{\textit{Map Service}}: Gerenciará posicionamento dos personagens, consulta de caminhos e calculos de área de interesse.
  \item{\textit{Character Service}}: Armazenará as informações referentes ao inventário do personagem, habilidades, nivel e afins.
\end{enumerate}
