\chapter{Proposta para análise de arquiteturas \ac{mmorpg}}
\label{cap3}



As arquiteturas de serviços \ac{mmorpg} são desenvolvidas visando suprir as necessidades do \textit{design} do jogo desenvolvido, de forma a qual viabilize a utilização deste serviço.
%
Nesse sentido, jogos com mecânicas de \textit{design} parecidos possuem implementações parecidas para os clientes.
%
Entretanto, a arquitetura escolhida impacta no custo de operação e qualidade do serviço aos jogadores.
%
Por este motivo, diferentes arquiteturas com o mesmo \textit{design} não são comparáveis entre sí, visto que dependem das regras de negócio.



Ao desenvolver um serviço \ac{mmorpg} é necessário decidir uma arquitetura a qual reduza custos, consumo de recursos e minimize ocorrências para os jogadores a fim de viabilizar o seu comércio como produto.
%
Porém, a impossibilidade de comparação direta entre as arquiteturas de serviço \ac{mmorpg} instiga a análise das características básicas que possam influenciar o \textit{game design}, tais como consumo de recursos, tempo de resposta, latência e número máximo de clientes.
%
Sendo assim, uma analise do consumo de recursos computacionais das arquiteturas levantadas previamente na literatura tem valor científico no auxílio na escolha de implementações arquiteturas de microsserviços, em específico para serviços \ac{mmorpg}.



Nesta seção é descrito a proposta para análise de consumo de recursos computacionais em arquiteturas \ac{mmorpg}.
%
A atual seção descreve a Proposta (Subseção~\ref{sec:proposta}), trazendo os objetivos desta análise, quais recursos e métricas serão analisadas.
%
Os Critérios de Análise (Subseção~\ref{sec:criterios}) exibem como os dados obtidos devem ser interpretados, baseando-se nos objetivos da análise das arquiteturas.
%
O Plano de Testes (Subseção~\ref{sec:plano}) exibe como será realizada a coleta dos dados, descrevendo cenários, critérios e objetivos dos testes.



\section{Proposta}
\label{sec:proposta}

Tendo analisado os trabalhos relacionados e as arquiteturas específicas para jogos \ac{mmorpg}, o presente trabalho tem como objetivo analisar as arquiteturas \ac{mmorpg} visando complementar os recursos computacionais não analisados nos trabalhos relacionados.
%
Em específico, serão obtidos das arquiteturas Rudy (Subseção~\ref{rudy}), Salz (Subseção~\ref{salz}) e Willson (Subseção~\ref{willson}) os seguintes valores referente aos recursos (Tabela~\ref{tab:recursos_categoria}):

\begin{enumerate}
  \item \textbf{\ac{cpu}}: o uso de CPU, com relação a porcentagem dos núcleos utilizados;
  \item \textbf{Memória}: Quantia de memória utilizada pelos processos da máquina. O valor será com valor absoluto;
  \item \textbf{Rede}: Serão obtidos os valores de entrada e saída do serviço, utilizando valores absolutos.
\end{enumerate}

Além dos recursos computacionais, esta análise levará em conta valores referente a outras métricas.
%
As métricas, cujos os valores serão obtidos são:

\begin{enumerate}
  \item \textbf{Número máximo de jogadores simultâneos}: Descobrir o limite de conexões para as arquiteturas propostas a análise.
  \item \textbf{Tempo de resposta das requisições}: Obter tempo de resposta por categoria de requisição, conforme o número de jogadores no serviço.
\end{enumerate}

Todas estas métricas obtidas, serão obtidos com simulações, e por este motivo se faz necessário descrever o comportamento dos jogadores simulados.

Para prever o comportamento destas métricas e recursos, serão calculadas as complexidades das operações usadas no plano de teste.
%
Espera-se em situações adversas, caracterizar os comportamentos, gargalos e o custos computacionais para manutenção dos serviços em cada arquitetura.
%
Para este fim, se faz necessário a descrição dos critérios que serão utilizados durante a análise dos valores obtidos nos experimentos.

%ccm
% objetivos
% Definição das métricas para os itens levantados nas Tabelas 2.4 e 2.5

\section{Critérios de análise}
\label{sec:criterios}

A fim de realizar a análise das arquiteturas de microsserviços específicos a jogos \ac{mmorpg}, se faz necessário definir critérios para analisar os valores obtidos.
%
Os critérios de análise vão guiar os casos de teste e a análise das arquiteturas.



\begin{itemize}
  \item Tabelas de consumo de recurso por cenário;
  \item Gráficos de relação de consumo de recursos por número de conexões;
\end{itemize}




%ccm
% Identificar como os valores obtidos pelas métricas devem ser interpretados
% pode ser individdual ou em grupos, defiir a finalidade? Custo? Desempenho?

%ccm Usar os critérios das colunas das Tabelas 2.4 e 2.5 de modo a deixar clara a importância de realizar uma análise que contemple todos os itens e não apenas partes deles, como identificado nos trabalahos relacionados (Seção 2.7).

\section {Plano de testes}
\label{sec:plano}



O plano de testes definirá os cenários que serão aplicados sobre todas as arquiteturas de microsserviços para jogos \ac{mmorpg}.
%
Ele servirá para descrever formas de estressar as arquiteturas, a fim de obter os valores para análise.
%
Entretanto, antes de relatar os cenários de teste, se faz necessário descrever o ambiente onde ocorrerá tais testes.
%
A Figura~\ref{Ambiente de testes} descreve a infraestrutura utilizada para execução das camadas de aplicação utilizados nos testes.



\begin{figure}[htb!]
  \caption{Ambiente de testes definido para a coleta de dados}
  \label{Ambiente de testes}
  \includegraphics[height=6.5cm]{img/cap3/infraestrutura.png}
  \centering

  Fonte: O próprio autor.
\end{figure}



Como visível na Figura~\ref{Ambiente de testes}, o ambiente de testes planejado está segregado em 5 camadas.
%
Essas camadas tem o objetivo de diminuir o impacto de desempenho e consumo de recursos por outras ferramentas durante os testes.
%
Por este motivo, as regiões da infraestrutura planejada são:



\begin{enumerate}
  \item \textbf{Serviço de Jogo}: A camada de serviço da infraestrutura dos testes concentrará a arquitetura de microsserviços referente as arquiteturas de microsserviços analisadas.
  \item \textbf{Banco de dados do serviço de jogo}: A camada de banco de dados do serviço de jogo conterá os serviços de dados e web a fim de manter um padrão de banco de dados para ambos os serviços utilizados e auxiliar na inicialização dos testes.
  \item \textbf{Estresse}: A camada de estresse será responsável por realizar requisições ao serviço a fim de estressá-lo, simulando padrões de requisição de um padrão de um jogador.
  \item \textbf{Cliente}: A camada de cliente será composta pelos mesmos elementos da camada de estresse, porém em um ambiente controlado para que a suíte de estresse não interfira nas métricas obtidas no lado do cliente.
  \item \textbf{Dados}: A camada de dados será composta por algum banco de dados de log a fim de armazenar os dados obtidos da camada Cliente e serviço, para utilizar na análise.
\end{enumerate}



As regiões da infraestrutura utilizada no ambiente de testes deve manter um padrão a fim que não exista interferência entre os testes, além do desempenho das arquiteturas do serviço.
%
Espera-se utilizar em grande parte o mesmo sistema de cliente para ambas as arquiteturas, excluso casos onde a arquitetura necessite de alterações.
%
Nesse sentido, espera-se obter somente a interferência das arquiteturas expressa nos valores obtidos.



Para os casos de uso, serão utilizada as arquiteturas de microsserviços específicos a jogos \ac{mmorpg} obtidos da literatura.
%
São essas elas:



\begin{enumerate}
  \item \textbf{Arquitetura Rudy} (Subseção~\ref{rudy});
  \item \textbf{Arquitetura Salz} (Subseção~\ref{salz});
  \item \textbf{Arquitetura Willson} (Subseção~\ref{willson}).
\end{enumerate}



Tais arquiteturas vão impactar o serviço de jogo, banco de dados e as requisições a qual os clientes deverão realizar.
%
Espera-se obter os valores referente a diferença de consumo de recursos computacionais dentro de cenários controlados utilizando o ambiente de testes.



Com o objetivo de obter dados, se faz necessário estressar as arquiteturas em casos diferentes para garantir a confiabilidade dos dados obtidos.
%
Dessa forma, será desenvolvido três cenários distintos a fim de obter dados utilizando para um número mínimo de jogadores simulados, com um custo operacional crescente de jogadores simulados a fim de descobrir as limitações das arquiteturas e por fim um cenário utilizando um número real de jogadores de um serviço \ac{mmorpg} obtidos da literatura.



\subsection{Cenário I}



O Cenário I corresponde execução dos casos de uso utilizando o menor número de jogadores simultâneos simulados possível.
%
Em específico as arquiteturas de microsserviços para jogos \ac{mmorpg}, este número será de apenas um cliente para o serviço.



Tal cenário contribuirá para obter valores referente a execução das arquiteturas com um número mínimo de conexões, a qual deve retornar os requisitos mínimos para execução de tais arquiteturas.
%
Para alcançar tal objetivo, este cenário será executado utilizando o método descrito na Tabela~\ref{tab:cenario_1}.

\begin{table}[htb!]
\centering
\caption{Método de execução do cenário 1}
\label{tab:cenario_1}
\begin{tabular}{|l|l|}
\hline
Tempo de captura    & 30 minutos \\ \hline
Número de jogadores & 1          \\ \hline
Jogadores por tempo & $f(t) = 1$ \\ \hline
Número de execuções & 5          \\ \hline 
\end{tabular}
\end{table}



A Tabela~\ref{tab:cenario_1} descreve as características de tempo de captura de logs para cara execução, a partir do inicio dos testes, o número de jogadores inicial, a função de crescimento de jogadores pelo tempo e o número de execuções do mesmo cenário.
%
Espera-se que seguindo tais medidas, obtenha-se dados semelhantes.


\subsection{Cenário II}

O cenário II corresponde ao cenário onde o número de jogadores seguirá crescendo, linearmente, até o limite do serviço.
%
Espera-se ocupar o máximo de todos os recursos do serviço, e com isso descobrir características como o número máximo de conexões de um serviço, tipos de requisições que geram gargalos e o seu consumo limite de consumo de recursos, conforme a limitação da infraestrutura dos testes.
%
Além dos recursos ocupados, outro valor importante será o tempo de resposta, a qual espera-se estressar com um número de jogadores mais elevado.
%
Para alcançar tais objetivos, este cenário será executado utilizando o método descrito na Tabela~\ref{tab:cenario_2}.

\begin{table}[htb!]
\centering
\caption{Método de execução do cenário 2}
\label{tab:cenario_2}
\begin{tabular}{|l|l|}
\hline
Tempo de captura    &    *              \\ \hline
Número de jogadores & 50                \\ \hline
Jogadores por tempo & $f(t) = 50 + 5*t$ \\ \hline
Número de execuções & 5                 \\ \hline 
\end{tabular}
\end{table}

O método descrito na Tabela~\ref{tab:cenario_2} não leva em conta o tempo de execução, visto que o mesmo seguirá até o limite da arquitetura, a qual espera-se alcançar com a elevação gradual de jogadores por parcela de tempo.
%
Serão executados 5 testes, para obter-se dados semelhantes.

\subsection{Cenário III}

O cenário III executará utilizando um gráfico de número de jogadores simultâneos de um serviço \ac{mmorpg} real obtido da literatura.
%
Espera-se obter com este cenário métricas reais para a manutenção deste serviço em ambiente de produção.

Estes dados serão valiosos para analisar a estabilidade dos recursos consumidos, garantindo que o teste de estresse está obtendo dados corretos para a análise.
%
A fim de alcançar tais objetivos, este cenário executará utilizando o método descrito na Tabela~\ref{tab:cenario_3}.

\begin{table}[htb!]
\centering
\caption{Método de execução do cenário 3}
\label{tab:cenario_3}
\begin{tabular}{|l|l|}
\hline
Tempo de captura    &  1 ciclo de função \\ \hline
Número de jogadores & *                 \\ \hline
Jogadores por tempo & $f(x) multimodal$ \\ \hline
Número de execuções & 5                 \\ \hline 
\end{tabular}
\end{table}

Para o método descrito na Tabela~\ref{tab:cenario_3}, espera-se executar 5 vezes para cada arquitetura utilizando a função multimodal (Figura~\ref{fig:players_peer_time}) obtida da literatura.
Cada ciclo de tempo será considerado uma execução completa.


%ccm Genérico/abstratpo
% Quais experimentos serão realizados, se não necessário explicitar os caso de uso e o quê espera-se de resultado.
% 
% Descrever como cenários, critérios e objetivos de testes.
