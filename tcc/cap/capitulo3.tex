\chapter{Proposta para análise de arquiteturas \ac{mmorpg}}
\label{cap3}

As arquitetura de serviços \ac{mmorpg} são desenvolvidas visando suprir as necessidades do \textit{design} do jogo desenvolvido, de forma a qual viabilize o comércio deste serviço como produto.
%
Nesse sentido, jogos com características parecidas podem, não obrigatoriamente, obter soluções de serviço parecidas.
%
As mudanças na arquitetura de tais serviços podem aumentar o de consumo de recursos e/ou manutenção destes serviços.


Tanto para o consumo de recursos quanto ao desempenho do serviço ao usuário final existe uma preocupação com a qualidade do serviço desenvolvido, visto que tais serviços fazem parte da base do comércio a cerca de jogos \ac{mmorpg}.
%
Nesse sentido, as arquiteturas de serviços \ac{mmorpg} apresentados a partir da literatura visam atender da melhor forma possível o produto para qual eles foram implementados, e por este motivo não podem ser comparados de forma direta, já que cada produto possui um \textit{design} diferente.


A dificuldade para comparar serviços \ac{mmorpg} torna-se um problema para uma análise e/ou decisão de desenvolvimento de novas arquiteturas, visto que se faz necessário tomar decisões sem embasamento teórico dentre as arquiteturas publicadas.
%
Este problema dificulta o desenvolviemento de jogos deste gênero por empresas de pequeno porte e/ou desenvolvedores independentes, por conta do baixo orçamento para pequisa prévia, deixando estes produtos por vezes mais caros por problemas de manutenção ou com consumo de recursos elevados em tais serviços, inviabilizando o consumo de tais jogos pelos usuários finais.



Com uma análise a cerca das arquiteturas de microsserviços para jogos \ac{mmorpg}, espera-se tornar de domínio público as métricas das arquiteturas desenvolvidas, gargalos e possíveis alterações para melhorar seu desempenho.
%
Tais informações podem ser utilizadas por pesquisas futuras sobre jogos \ac{mmorpg}, microsserviços, sistemas distribuídos, segurança, complexidade de algoritmos, banco de dados e redes de computadores, também contribuindo com futuros desenvolvedores de tais arquiteturas.


A proposta (Seção~\ref{sec:proposta}) de análise dos recursos utilizados em serviços \ac{mmorpg} mostra qual será o objetivo deste trabalho e qual a seu valor científico.
%
Nesta seção são exibidos quais métricas serão utilizadas e por qual motivo será realizada a atual análise proposta.



Os critérios de análise (Seção~\ref{sec:criterios}) mostram como os valores obtidos devem ser interpretados e qual o impácto da variação dos valores.
%
Nesta seção é exibido quais situações podem gerar ocorrências, dentro dos serviços \ac{mmorpg}.
%
Esses itens serão utilizados pelo plano de testes para forçar tais situações nestes serviços.


O plano de testes (Seção~\ref{sec:plano}) do atual trabalho mostra como os valores serão obtidos, dentre eles mostrando os cenários e experimentos.

\section{Proposta}
\label{sec:proposta}

%ccm
% objetivos
% Definição das métricas para os itens levantados nas Tabelas 2.4 e 2.5

\section{Critérios de análise}
\label{sec:criterios}

%ccm
% Identificar como os valores obtidos pelas métricas devem ser interpretados
% pode ser individdual ou em grupos, defiir a finalidade? Custo? Desempenho?

\section {Plano de testes}
\label{sec:plano}

%ccm Genérico/abstratpo
% Quais experimentos se reão realizados, se não necessário explicitar os caso de uso e o quê espera-se de resultado.

Para realizar a análise proposta sobre as arquiteturas descritas na referência teórica, se faz necessário obter coletar/medir métricas de recursos consumidos por tal arquitetura.
%
Nesse sentido, faz-se necessário a descrição específica da arquitetura analisada, equivalente as arquiteturas descritas no Capítulo~\ref{cap2}.
%
Para este fim, este trabalho irá analisar o consumo de recursos como:

\begin{enumerate}
  \item{Memória: Analisar a quantia de memória utilizada referente a cada arquitetura, analisando possíveis rotinas que consomem recursos de forma inapropriada;}
  \item{Banda de rede: Analisar o comportamento da arquitetura em relação ao consumo de banda, relevando a topologia da arquitetura;}
  \item{Processamento: Analisar rotinas de processamento de baixo desempenho;}
  \item{Tempo de resposta: Analisar o tempo de resposta para que as ações sejam computadas, a partir da chamada do cliente, pelo número de conexões no serviço;}
  \item{Limite de conexões: Analisar o limite de conexões de determinada arquitetura;}
\end{enumerate}

A fim de garantir a confiabilidade dos dados coletados, cada microsserviço da arquitetura executará de forma isolada, a nível do sistema operacional.
%
Pode-se garantir este isolamento utilizando técnicas de virtualização.
%
Analisando os dados obtidos, espera-se encontrar gargalos em tais arquiteturas, provendo possíveis soluções para os gargalos encontrados.
%
Outros critérios também serão abordados na análise, tais como a estabilidade de operação, capacidade de processamento, custo de operação e limite de conexão das arquiteturas.
%

Entretanto, as arquiteturas de microsserviços para jogos \ac{mmorpg} são aplicações de alta complexidade, a qual sofrem alterações conforma as regras de negócio impostas ao jogo em específico.
%
Por este motivo, torna-se necessário descrever as suas funcionalidades específicas, a fim de suprir confiabilidade na análise das arquiteturas, descrevendo as especificações de implementação sobre as arquiteturas Rudy, Willson e Salz.

\section{Proposta}


%ccm Usar os critérios das colunas das Tabelas 2.4 e 2.5 de modo a deixar clara a importância de realizar uma análise que contemple todos os itens e não apenas partes deles, como identificado nos trabalahos relacionados (Seção 2.7).

O atual trabalho propõe uma análise de consumo de recursos a arquiteturas de jogos \ac{mmorpg}, em específico para as arquiteturas Rudy (~\ref{rudy}), Willson (~\ref{willson}) e Salz (~\ref{salz}).

Esta análise contribuirá a melhorar técnicas de desenvolvimento de jogos massivos, reduzindo custos de manutenção de tais serviços.


\section{Critérios de análise}

% A fim de guiar um padrão para todas as arquiteturas analisadas, será utilizado um padrão de regra de negócios para os serviços desenvolvidos.
% %
% Este padrão seguirá as seguintes regras de negócio:
% 
% \begin{enumerate}
%   \item Ambiente tridimensional, com posicionamento bidimensional: O ambiente de jogo será computado em 3D dimensões, porém os personagens são fixos em um eixo.
%   \item Troca de mensagens locais e globais: O serviço deve permitir a troca de mensagens com os demais personagens ao seu redor, dentro do seu campo de visão, e com todos os personagens de um mundo.
%   \item Movimentação no ambiente: O serviço deve permitir o personagem a movimentar-se da sua posição atual para outra posição.
%   \item Realizar ações no ambiente: O serviço deve permitir o usuário a interagir com itens, \ac{npcs} e outros jogadores.
% \end{enumerate}
% 
% \subsection{Ambiente Tridimenssional}
% \subsection{Troca de mensagens locais e globais}
% \subsection{Realizar ações no ambiente}
% 
\section{Plano de testes}
% 
% Descrever como cenários, critérios e objetivos de testes.
