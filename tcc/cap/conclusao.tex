\chapter{Considerações \& Próximos passos}
\label{cap:conclusao}

                % 1. Uma explicação informando de modo claro se atingiu ou não os objetivos estabelecidos (aqui pode-se ter também uma subdivisão entre objetivos gerais e objetivos específicos). Em cada caso devem ser explicados os motivos:
                %     a. Caso tenha atingido os objetivos: informar os principais fatores que contribuíram para o sucesso, descrevendo-os de forma breve porém que não deixem dúvidas;
                %     b. Caso não tenha atingido os objetivos: informar o quanto do objetivo foi atingido e citar os fatores que contribuíram para o insucesso, descrevendo-os de forma breve porém que não deixe dúvidas.
                % 2. Descrever as principais considerações e conclusões que foram obtidas em decorrência da execução do trabalho. Aqui não deve ser repetido texto já existente no trabalho mas escrever as impressões dessas considerações e como elas contribuíram para a execução e atingir o objetivo;
                % 3. Citar e descrever as principais dificuldades encontradas para execução do trabalho e projeto. Todo o trabalho desenvolvido significa uma evolução para o aluno, sendo que para chegar essa evolução o mesmo necessitou transpor uma série de obstáculos. Relatar os obstáculos e como superou (ou não superou)  ajuda a dignificar e mostrar o mérito do trabalho em si para o leitor/ avaliador. Também é uma contribuição, no sentido que uma vez expostos os problemas e soluções os leitores/avaliadores aprendem/conhecem formas de resolução ou de abordagem a tais problemas; 
                % 4. Comentar se ocorreram modificações durante a execução do trabalho no escopo definido na fase de Projeto e no que fora desenvolvido. Deve ser explicado o quê gerou essas modificações, fundamentando e justificando tais alterações. 
                % 5. Pode ser descrita a relação entre cronograma proposto e cronograma realizado no trabalho. Permitindo assim o leitor/avaliador aprender com as distorções/acertos indicados.  
                % 6. Descrever ou citar trabalhos futuros que podem ser feitos com base nesse trabalho desenvolvido. No decorrer da execução de um trabalho busca-se atingir um objetivo definido no projeto, porém vários assuntos interessantes de pesquisar são  revelados (sendo que os mesmo não são tratados/pesquisados no trabalho por não condizerem com os objetivos / escopo do trabalho). A descrição de tais assuntos/temas/pesquisas demonstra a percepção desenvolvida pelo aluno no desenvolvimento assim como a sua visão de objetividade na execução desse trabalho.

Este TCC tem como objetivo levantar questões relacionadas a complexidade de coordenação de arquiteturas de microsserviços para jogos \ac{mmorpg}.
%
Estas arquiteturas são complexas devido a quantidade que visam suprir a demanda do mercado para esse tipo de jogo.
%
Sendo assim, para entender melhor as características de tais arquiteturas, este trabalho propõe a realização de uma análise sobre arquiteturas distintas expressas na literatura.


O atual trabalho, iniciado no primeiro semestre de 2018, atingiu seus objetivos.
%
Dentre as atividades estipuladas para a primeira fase do TCC, todas foram concluídas, porém não dentro do prazo estipulado de um semestre.

As principais considerações acerca da pesquisa referenciada são relevantes as características de projeto de um serviço distribuído em microsserviços para atender jogos \ac{mmorpg}.
%
Elas podem ser enumeradas da seguinte forma:

\begin{itemize}
  \item Preocupação de consumo de recursos e escalabilidade para muitos jogadores, evitando a segregação de jogadores em seu ambiente;
  \item Preocupação dos autores das arquiteturas acerca do processamento em relação ao escalonador de processos do sistema operacional, cache de dados e/ou requisições e modelo de processamento paralelo das requisições, visando alto desempenho das arquiteturas;
  \item Preocupação dos desenvolvedores das arquiteturas com o funcionamento independente do cliente, fomentando o desenvolvimento multiplataforma do gênero.
\end{itemize}

Estas preocupações ampliam a utilidade do atual trabalho.
%
Entretanto, o conteúdo encontrado para a pesquisa referenciada dificultou o processo de desenvolvimento da referenciação teórica.
%
Em específico, as dificuldades encontradas durante a elaboração da primeira parte do atual trabalho foram:

\begin{itemize}
  \item Dificuldade de encontrar material científico ou acadêmico correlacionando arquiteturas de microsserviços e arquiteturas para jogos \ac{mmorpg}.
  \item Os autores das arquiteturas especificam com exatidão determinados pontos relevantes ao projeto, mas não existe um aprofundamento para determinados microsserviços da arquitetura. Dessa forma foi necessário realizar buscas na literatura a fim de viabilizar tais projetos, seguindo as especificações das arquiteturas.
\end{itemize}

Além das dificuldades com relação direta aos objetivos atuais deste trabalho, houve a necessidade de realizar um estudo sobre as tecnologias que permeiam o processo de implantação, desenvolvimento e testes de arquiteturas de microsserviços.
%
A resolução destas dificuldades na atual etapa facilitará o desenvolvimento do TCC II.

Para o TCC-II serão desenvolvidas as arquiteturas de microsserviços para jogos \ac{mmorpg} descritas na proposta do atual trabalho.
%
Também serão desenvolvidos testes de carga automatizados a fim de facilitar a etapa de testes das arquiteturas.
%
Por fim, os dados serão coletados por um sistema de análise de recursos desenvolvido no TCC-II, durante os testes, serão analisados a fim de gerar uma análise das arquiteturas de microsserviços para jogos \ac{mmorpg} que levantará questões de desempenho presentes nela.
%
Também serão desenvolvidos testes de carga automatizados a fim de facilitar a etapa de testes das arquiteturas.
%
Um sistema de análise de recursos, ao qual será também desenvolvido no TCC-II, coletará os dados durante os testes. Estes dados serão utilizados na análise das arquiteturas de microsserviços para jogos \ac{mmorpg}, essa análise deverá levantar questões acerca de desempenho.


\section{Cronograma}



Até o presente momento, o cronograma proposto no Plano do TCC foi seguido conforme o previsto, na ordem e nos prazos estipulados.
%
A etapas presentes nesta seção foram definidas no Plano de TCC para atingir os objetivos propostos.



\section{Etapas realizadas}



\begin{enumerate}
  \item \textbf{Levantamento e fichamento das referências:} Pesquisa de fontes para embasamento teórico do trabalho, com base nos objetivos específicos;

  \item \textbf{Consolidação das referências:} Compreensão e seleção de artefatos literários que permitam atingir o objetivo do Trabalho de Conclusão de Curso I;

  \item \textbf{Identificação e definição de arquiteturas descritas na literatura:} Enumeração e definição das arquiteturas de microsserviços descritas na literatura, bem como os seus objetivos;

  \item \textbf{Especificação das arquiteturas selecionadas:} Especificação do funcionamento das arquiteturas selecionadas.

  \item \textbf{Identificação e definição de simulações aplicáveis ao teste:} Seleção e definição da simulação de clientes a ser aplicada nos testes;

  \item \textbf{Especificação da simulação elegida:} Especificação dos requisitos necessários para a simulação dos clientes aos serviços escolhidos;

  \item \textbf{Escrita Trabalho de Conclusão de Curso I};
\end{enumerate}



\section{Etapas a realizar}



\begin{enumerate}
  \setcounter{enumi}{7}
  \item \textbf{Desenvolvimento da simulação:} Desenvolvimento da simulação para interagir com as arquiteturas de microsserviços;

  \item \textbf{Desenvolvimento da arquitetura:} Desenvolvimento da arquitetura para executar os testes;

  \item \textbf{Aplicação das arquiteturas selecionadas na pesquisa referênciada:} Aplicação das arquiteturas descritas sobre uma nuvem computacional;

  \item \textbf{Realização dos testes utilizando a simulação elegida na pesquisa referênciada:} Execução de testes da arquitetura desenvolvida sobre a nuvem computacional;

  \item \textbf{Análise das arquiteturas testadas:} Análise das métricas obtidas dos testes e descrever resultados, identificando possíveis gargalos nas arquiteturas;

  \item \textbf{Otimização para melhorar as métricas obtidas:} Identificação dos pontos de gargalo nos microsserviços selecionados e proposta de soluções viáveis para aumento do desempenho desses sistemas.

  \item \textbf{Escrita Trabalho de Conclusão de Curso II};
\end{enumerate}



\section{Execução do cronograma}

\begin{center}

\vspace{0.5cm}
{\tiny
\noindent \begin{adjustbox}{max width=\textwidth}\begin{tabular}{|c||c|c|c|c|c|c|c|c|c|c|c|c||c|c|c|c|c|c|c|c|c|c|c|c|}
  \hline
  \multirow{2}{*}{\textbf{\small{Etapas}}} & \multicolumn{12}{|c||}{\textbf{\small{2018}}} & \multicolumn{12}{|c|}{\textbf{\small{2019}}} \\
  \cline{2-25}
   & \textbf{J} & \textbf{F} & \textbf{M} & \textbf{A} & \textbf{M} & \textbf{J} & \textbf{J} & \textbf{A} & \textbf{S} & \textbf{O} & \textbf{N} & \textbf{D} & \textbf{J} & \textbf{F} & \textbf{M} & \textbf{A} & \textbf{M} & \textbf{J} & \textbf{J} & \textbf{A} & \textbf{S} & \textbf{O} & \textbf{N} & \textbf{D} \\
  \hline \hline
  \textbf{\small{ 1}} &  & x & x & x & x & x & x & \cellcolor{black} \textcolor{white}{X} &  &  &  &  &  &  &  &  &  &  &  &  &  &  &  & \\ \hline
  \textbf{\small{ 2}} &  &  & x & x & x &  &  & \cellcolor{black} \textcolor{white}{X} &  &  &  &  &  &  &  &  &  &  &  &  &  &  &  & \\ \hline
  \textbf{\small{ 3}} &  &  &  &  & x & x & x & \cellcolor{black} \textcolor{white}{X} & \cellcolor{black} \textcolor{white}{X} &  &  &  &  &  &  &  &  &  &  &  &  &  &  & \\ \hline
  \textbf{\small{ 4}} &  &  &  &  &  &  &  & \cellcolor{black} \textcolor{white}{X} & \cellcolor{black} \textcolor{white}{X} &  &  &  &  &  &  &  &  &  &  &  &  &  &  & \\ \hline
  \textbf{\small{ 5}} &  &  &  &  &  &  &  &  &  & \cellcolor{black} \textcolor{white}{X} &  &  &  &  &  &  &  &  &  &  &  &  &  & \\ \hline
  \textbf{\small{ 6}} &  &  &  &  &  &  &  &  &  & \cellcolor{black} \textcolor{white}{X} &  &  &  &  &  &  &  &  &  &  &  &  &  & \\ \hline
  \textbf{\small{ 7}} &  &  &  &  &  &  &  & \cellcolor{black} \textcolor{white}{X} & \cellcolor{black} \textcolor{white}{X} & \cellcolor{black} \textcolor{white}{X} & \cellcolor{black} \textcolor{white}{X} &  &  &  &  &  &  &  &  &  &  &  &  & \\ \hline
  \textbf{\small{ 8}} &  &  &  &  &  &  &  &  &  &   & x & \cellcolor{black} & \cellcolor{black} &  &  &  &  &  &  &  &  &  &  & \\ \hline
  \textbf{\small{ 9}} &  &  &  &  &  &  &  &  &  & x & x & \cellcolor{black} & \cellcolor{black} &  &  &  &  &  &  &  &  &  &  & \\ \hline
  \textbf{\small{10}} &  &  &  &  &  &  &  &  &  &  &  &  &  & \cellcolor{black} &  &  &  &  &  &  &  &  &  & \\ \hline
  \textbf{\small{11}} &  &  &  &  &  &  &  &  &  &  &  &  &  & \cellcolor{black} &  &  &  &  &  &  &  &  &  & \\ \hline
  \textbf{\small{12}} &  &  &  &  &  &  &  &  &  &  &  &  &  &  & \cellcolor{black} &  &  &  &  &  &  &  &  & \\ \hline
  \textbf{\small{13}} &  &  &  &  &  &  &  &  &  &  &  &  &  &  &  & \cellcolor{black} &  &  &  &  &  &  &  & \\ \hline
  \textbf{\small{14}} &  &  &  &  &  &  &  &  &  &  &  & \cellcolor{black} & \cellcolor{black} & \cellcolor{black} & \cellcolor{black} & \cellcolor{black} & \cellcolor{black} & \cellcolor{black} &  &  &  &  &  & \\ \hline
\end{tabular}
\end{adjustbox}
}
\end{center}
