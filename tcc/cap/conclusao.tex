\chapter{Considerações \& Trabalhos futuros}
\label{cap:conclusao}

% 1. Uma explicação informando de modo claro se atingiu ou não os objetivos estabelecidos (aqui pode-se ter também uma subdivisão entre objetivos gerais e objetivos específicos). Em cada caso devem ser explicados os motivos:
%     a. Caso tenha atingido os objetivos: informar os principais fatores que contribuíram para o sucesso, descrevendo-os de forma breve porém que não deixem dúvidas;
%     b. Caso não tenha atingido os objetivos: informar o quanto do objetivo foi atingido e citar os fatores que contribuíram para o insucesso, descrevendo-os de forma breve porém que não deixe dúvidas.
% 2. Descrever as principais considerações e conclusões que foram obtidas em decorrência da execução do trabalho. Aqui não deve ser repetido texto já existente no trabalho mas escrever as impressões dessas considerações e como elas contribuíram para a execução e atingir o objetivo;
% 3. Citar e descrever as principais dificuldades encontradas para execução do trabalho e projeto. Todo o trabalho desenvolvido significa uma evolução para o aluno, sendo que para chegar essa evolução o mesmo necessitou transpor uma série de obstáculos. Relatar os obstáculos e como superou (ou não superou)  ajuda a dignificar e mostrar o mérito do trabalho em si para o leitor/ avaliador. Também é uma contribuição, no sentido que uma vez expostos os problemas e soluções os leitores/avaliadores aprendem/conhecem formas de resolução ou de abordagem a tais problemas;
% 4. Comentar se ocorreram modificações durante a execução do trabalho no escopo definido na fase de Projeto e no que fora desenvolvido. Deve ser explicado o quê gerou essas modificações, fundamentando e justificando tais alterações.
% 5. Pode ser descrita a relação entre cronograma proposto e cronograma realizado no trabalho. Permitindo assim o leitor/avaliador aprender com as distorções/acertos indicados.
% 6. Descrever ou citar trabalhos futuros que podem ser feitos com base nesse trabalho desenvolvido. No decorrer da execução de um trabalho busca-se atingir um objetivo definido no projeto, porém vários assuntos interessantes de pesquisar são  revelados (sendo que os mesmo não são tratados/pesquisados no trabalho por não condizerem com os objetivos / escopo do trabalho). A descrição de tais assuntos/temas/pesquisas demonstra a percepção desenvolvida pelo aluno no desenvolvimento assim como a sua visão de objetividade na execução desse trabalho.

%ccm Incluir seção de contribuições do TCC
%ccm Incluir seção de Trabalhos futuros



Jogos \ac{mmorpg} são utilizados como negócio viável e lucrativo, sendo que, a experiência de jogabilidade na qual o usuário final será submetido é um fator crítico para o sucesso destes jogos.
%
Tais serviços são implementados sobre arquiteturas que executam o serviço sobre diversos servidores, na qual o desempenho deste serviço e o custo de sua manutenção é um fator crítico para o sucesso de um jogo deste gênero.
%
Modelar um sistema de alto desempenho para tais serviços torna-se um trabalho essencial para a satisfação do usuário final neste cenário.


O atual trabalho teve como objetivo analisar as arquiteturas de microsserviços Rudy, Salz e Willson, caracterizadas especificamente para jogos \ac{mmorpg} com o objetivo de oferecer relações e efeitos sobre as arquiteturas selecionadas.
%
Esta análise é baseada na coleta de informações do consumo de recursos para sua execução e um valor de qualidade das arquiteturas, do ponto de vista do cliente.

Nesse sentido, o atual trabalho obteve sucesso ao realizar esta análise, gerando informação sobre uma relação entre a qualidade das arquiteturas de microsserviços selecionadas para um conjunto de regras de negócio de um jogo genérico e o consumo de recursos computacionais para as respectivas execuções.
%
Ambas as arquiteturas desempenharam seus papéis sem problemas, demonstrando características únicas na qual evidenciaram-se neste trabalho.

Um objetivo específico deste trabalho era validar as características obtidas da literatura, haja vista que os autores citavam o seu comportamento sem a comprovação destes comportamentos.
%
Dessa forma, o atual trabalho concluiu com sucesso a validação destas características, tornando tais dados como verdadeiros para futuros pesquisadores.

Outro fator pertinente após a analise é com relação ao gargalo encontrado, no qual mostrou-se relacionado a forma de obter e armazenar dados.
%
Entretanto, tal informação não é uma contribuição inicial do atual trabalho visto que outros autores do levantamento teórico realizado já citavam tal característica.

O atual trabalho concluiu que o desempenho, do ponto de vista de tempo de resposta, está relacionado a melhor utilização da \ac{cpu}, seja pelos microsserviços de armazenamento de dados ou por microsserviços de processamento de dados.
%
Mostrou-se viável a aplicação de sistemas de filas ou barreiras para gerenciamento do acesso ou minimização do consumo de um serviço interno, impactando diretamente na vazão dos dados pela arquitetura.
%
Outro critério de atenção é a sincronização de dados, a qual pode consumir um valor significativo de \ac{cpu} e, caso desempenhe um papel incompatível com o necessário, pode prejudicar o processamento e vazão dos dados pela arquitetura.

Ao final deste trabalho, não foi possível analisar todos os dados obtidos.
%
A partir dos dados coletados, existem dados intermediários que não foram utilizados nesta análise.
%
Este conjunto de dados possui informações de monitoramento de todas as máquinas envolvidas nos experimentos de forma individual, quanto dos processos individuais na arquitetura do Docker Swarm e Docker Compose.
%
Pretende-se analisar esses dados, preferencialmente, durante a escrita de um artigo técnico científico previsto como trabalho futuro. 

Este trabalho teve algumas dificuldades durante a sua execução.
%
Durante o processo de busca por referências teóricas na literatura, mostrou-se uma área atacada por diversos modelos para processamento de dados e desenvolvimento web. 
%
Porém existe uma baixa frequência de publicações de artigos para jogos \ac{mmorpg}, abordando sua arquitetura em baixo nível.
%
Tal problema foi solucionado ao buscar as arquiteturas para jogos \ac{mmorpg} e encontrar modelos próximos em desenvolvimento web.

Outro problema significativo tem relação com a arquitetura e organização de código que foi necessário para a implementação das arquiteturas e clientes.
%
Por se tratar de um projeto de considerável porte, na qual teve a duração de seis meses de desenvolvimento, foram encontrados diversos problemas de engenharia de software para permitir o isolamento da regra de negócio de forma abstrata, garantindo que todas as arquiteturas estavam processando as requisições sem otimizações particulares.
%
Este problema foi superado ao utilizar técnicas de \textit{Clean Architecture} e testes automatizados utilizando uma suíte de integração contínua.
%
Estas técnicas aumentaram a produtividade para o desenvolvimento em longo prazo, permitindo a sua implementação em um menor espaço de tempo.
%

A segunda maior dificuldade apresentada no desenvolvimento deste trabalho foi encontrada durante a execução dos testes no ambiente.
%
Encontraram-se dois problemas relacionados a infraestrutura dos ambientes, especificamente no \textit{firewall} que interconecta a rede de laboratórios a nuvem LabP2D.
%
Estes problemas foram a limitação de tráfego entre as redes e o bloqueio de acesso a diversos protocolos limitando a conexão entre os clientes e o serviços.
%
O problema de limitação de tráfego foi resolvido limitando o recurso do serviço para ser estressado mais facilmente e o problema de bloqueio de acesso a protocolos foi resolvido após a realização de alguns chamados ao suporte técnico da rede.
%

A execução do atual trabalho não obteve alterações em seu escopo, entretanto foi executada em um período prolongado comparado ao estimado.
%
Este prolongamento não teve relações com os problemas identificados no desenvolvimento do atual trabalho e sim com dificuldades encontradas no processo de produção textual do atual trabalho, sendo que esta foi a maior dificuldade encontrada na realização deste trabalho.

\section{Contribuições do TCC}

Este TCC contribui com futuros jogos \ac{mmorpg} ou arquiteturas distribuídas de outras áreas, a qual beneficiam-se dos dados, características e conclusões encontradas no atual trabalho.
%
Além desta contribuição, o atual trabalho também serve como guia de engenharia de software para a implementação de protótipos funcionais, contribuindo com exemplos designados a esta área na qual obteve-se dificuldades de encontrar material científico.

O atual trabalho também, indiretamente, contribuiu com diversas ferramentas e softwares \textit{OpenSource} durante o processo de implementação das arquiteturas, pois tais ferramentas necessitaram de pequenos ajustes para funcionamento de protocolos de redes.
%
A fim de criar um serviço válido, protótipos de clientes reais com o motor gráfico Godot foram implementados para validação do protocolo de comunicação do serviço.
%
A ferramenta que mais obteve contribuições foi o motor gráfico Godot, a qual recebeu diversas alterações em sua classe de conexão \ac{tcp} para viabilizar protótipos deste trabalho.
%
Tais alterações foram disponibilizadas no repositório oficial da ferramenta, sendo essa disponibilização aplicada a versão mais recente (3.1, disponível no segundo semestre de 2019) e consequentemente utilizadas por outros desenvolvedores da comunidade para desenvolvimento de futuros jogos.




\section{Trabalhos futuros}

Este TCC permite uma sequência de trabalhos futuros.
%
Alguns possíveis trabalhos futuros podem utilizar os dados pertinentes neste trabalho, tal qual analisar o impacto de melhorias ou otimizações em tais sistemas.
%
Também pode-se abordar os temas adjacentes utilizados neste trabalho, como tecnologias, protocolos ou metodologias utilizadas no desenvolvimento, implantação e análise do atual trabalho, utilizando os resultados e ferramentas desenvolvidas em futuras pesquisas.
%
Em específico, pode-se continuar este trabalho com os seguintes sugestões de temas:

\begin{itemize}
 \item Predição do uso de recursos baseado no modelo computacional das arquiteturas Rudy, Salz e Willson;
 \item Análise do gerenciador de processos do Docker na validação de consumo de \ac{cpu} com características senoidais;
 \item Análise do impacto na troca do ambiente de implantação;
 \item Análise do impacto no tempo de resposta ao remover o sistema de monitoramento de uso de recursos;
 \item Análise do impacto no troca dos serviços de bando de dados;
 \item Análise do impacto na utilização do sistema de assinatura de dados \ac{jwt} como sistema de autorização;
 \item Análise do impacto na otimização dos protocolos de comunicação utilizados;
 \item Anaĺise do consumo de recursos pela arquitetura de armazenamento de métricas; e
 \item Análise dos dados capturados mas não analisados no atual trabalho.
\end{itemize}

%ccm escrever artigo
