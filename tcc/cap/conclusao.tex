\chapter{Considerações \& Próximos passos}
\label{cap:conclusao}

Este TCC tem como objetivo levantar questões relacionadas a complexidade de coordenação de arquiteturas de microsserviços para jogos \ac{mmorpg}.
%
Estas arquiteturas são complexas devido a quantidade de elementos em sua arquitetura a fim de suprir a demanda do mercado para estes jogos.
%
Sendo assim, para entender melhor as características de tais arquiteturas, este trabalho propõe a realização de uma análise sobre arquiteturas distintas expressas na literatura.



O levantamento bibliográfico é um dos passos iniciais para guiar este objetivo. Nessa etapa apresentou-se conceitos de jogos \ac{mmorpg}, padrões de projetos para tais jogos, protocolos utilizados e técnicas de desenvolvimento.
%
Sobre estes conceitos foi descrito o problema referenciado neste TCC, no qual aplicaram-se alguns dos conceitos descritos anteriormente.



Após, foram analisados trabalhos relacionados cujos os objetivos são de análise de arquiteturas de microsserviços ou análise de arquiteturas de jogos \ac{mmorpg}.
%
Para guiar o inicio da proposta, foi descrito as arquiteturas encontradas na literatura, na qual serão utilizadas no plano de testes do atual trabalho.
%
Na segunda parte da proposta, foi apresentado o plano de testes a fim de dar suporte a análise das arquiteturas.



Para o TCC-II será desenvolvido as arquiteturas de microsserviços para jogos \ac{mmorpg} descritas na proposta do atual trabalho.
%
Também será desenvolvido testes de carga automatizado a fim de facilitar a etapa de testes das arquiteturas.
%
Por fim, os dados coletados por um sistema de análise de recursos desenvolvido no TCC-II, durante os testes, serão analisados a fim de gerar uma análise das arquiteturas de microsserviços para jogos \ac{mmorpg} que levantará questões de desempenho presentes nela.
%
Espera-se obter características específicas diferentes de cada arquitetura durante esta análise, casos de uso onde cada arquitetura seja aplicável e solução de gargalos encontrados ns arquiteturas descritas.

\section{Cronograma}



Até o presente momento, o cronograma proposto no Plano do TCC foi seguido conforme o previsto, na ordem e nos prazos estipulados.
%
A etapas presentes nesta seção foram definidas no Plano de TCC para atingir os objetivos propostos.



\section{Etapas realizadas}



\begin{enumerate}
  \item \textbf{Levantamento e fichamento das referências:} Pesquisa de fontes para embasamento teórico do trabalho, com base nos objetivos específicos;

  \item \textbf{Consolidação das referências:} Compreensão e seleção de artefatos literários que permitam atingir o objetivo do Trabalho de Conclusão de Curso I;

  \item \textbf{Identificação e definição de arquiteturas descritas na literatura:} Enumeração e definir das arquiteturas de microsserviços descritas na literatura, bem como os seus objetivos;

  \item \textbf{Especificação das arquiteturas selecionadas:} Especificar o funcionamento das arquiteturas selecionadas.

  \item \textbf{Identificação e definição de simulações aplicáveis ao teste:} Eleger e definir a simulação a ser aplicada nos testes;

  \item \textbf{Especificação da simulação elegida:} Especificar os requisitos;

  \item \textbf{Escrita Trabalho de Conclusão de Curso I};
\end{enumerate}



\section{Etapas a realizar}



\begin{enumerate}
  \setcounter{enumi}{7}
  \item \textbf{Desenvolvimento da simulação:} Desenvolvimento da simulação para interagir com as arquiteturas de microsserviços;

  \item \textbf{Desenvolvimento da arquitetura:} Desenvolvimento da arquitetura para executar os testes;

  \item \textbf{Aplicação das arquiteturas selecionadas na pesquisa referênciada:} Aplicação das arquiteturas descritas sobre uma nuvem computacional;

  \item \textbf{Realização dos testes utilizando a simulação elegida na pesquisa referênciada:} Execução de testes da arquitetura desenvolvida sobre a nuvem computacional;

  \item \textbf{Análise das arquiteturas testadas:} Analisar as métricas obtidas dos testes e descrever resultados, identificando possíveis gargalos nas arquiteturas;

  \item \textbf{Otimização para melhorar as métricas obtidas:} Identificar pontos de gargalo nos microsserviços identificados e propor soluções viáveis para aumentar o desempenho desses sistemas.

  \item \textbf{Escrita Trabalho de Conclusão de Curso II};
\end{enumerate}



\section{Execução do cronograma}

\begin{center}

\vspace{0.5cm}
{\tiny
\noindent \begin{tabular}{|c||c|c|c|c|c|c|c|c|c|c|c|c||c|c|c|c|c|c|c|c|c|c|c|c|}
  \hline
  \multirow{2}{*}{\textbf{\small{Etapas}}} & \multicolumn{12}{|c||}{\textbf{\small{2018}}} & \multicolumn{12}{|c|}{\textbf{\small{2019}}} \\
  \cline{2-25}
   & \textbf{J} & \textbf{F} & \textbf{M} & \textbf{A} & \textbf{M} & \textbf{J} & \textbf{J} & \textbf{A} & \textbf{S} & \textbf{O} & \textbf{N} & \textbf{D} & \textbf{J} & \textbf{F} & \textbf{M} & \textbf{A} & \textbf{M} & \textbf{J} & \textbf{J} & \textbf{A} & \textbf{S} & \textbf{O} & \textbf{N} & \textbf{D} \\
  \hline \hline
  \textbf{\small{ 1}} &  &  &  &  &  &  &  & \cellcolor{black} \textcolor{white}{X} &  &  &  &  &  &  &  &  &  &  &  &  &  &  &  & \\ \hline
  \textbf{\small{ 2}} &  &  &  &  &  &  &  & \cellcolor{black} \textcolor{white}{X} &  &  &  &  &  &  &  &  &  &  &  &  &  &  &  & \\ \hline
  \textbf{\small{ 3}} &  &  &  &  &  &  &  & \cellcolor{black} \textcolor{white}{X} & \cellcolor{black} \textcolor{white}{X} &  &  &  &  &  &  &  &  &  &  &  &  &  &  & \\ \hline
  \textbf{\small{ 4}} &  &  &  &  &  &  &  & \cellcolor{black} \textcolor{white}{X} & \cellcolor{black} \textcolor{white}{X} &  &  &  &  &  &  &  &  &  &  &  &  &  &  & \\ \hline
  \textbf{\small{ 5}} &  &  &  &  &  &  &  &  &  & \cellcolor{black} \textcolor{white}{X} &  &  &  &  &  &  &  &  &  &  &  &  &  & \\ \hline
  \textbf{\small{ 6}} &  &  &  &  &  &  &  &  &  & \cellcolor{black} \textcolor{white}{X} &  &  &  &  &  &  &  &  &  &  &  &  &  & \\ \hline
  \textbf{\small{ 7}} &  &  &  &  &  &  &  & \cellcolor{black} \textcolor{white}{X} & \cellcolor{black} \textcolor{white}{X} & \cellcolor{black} \textcolor{white}{X} & \cellcolor{black} \textcolor{white}{X} &  &  &  &  &  &  &  &  &  &  &  &  & \\ \hline
  \textbf{\small{ 8}} &  &  &  &  &  &  &  &  &  &  &  & \cellcolor{black} & \cellcolor{black} &  &  &  &  &  &  &  &  &  &  & \\ \hline
  \textbf{\small{ 9}} &  &  &  &  &  &  &  &  &  &  &  & \cellcolor{black} & \cellcolor{black} &  &  &  &  &  &  &  &  &  &  & \\ \hline
  \textbf{\small{10}} &  &  &  &  &  &  &  &  &  &  &  &  &  & \cellcolor{black} &  &  &  &  &  &  &  &  &  & \\ \hline
  \textbf{\small{11}} &  &  &  &  &  &  &  &  &  &  &  &  &  & \cellcolor{black} &  &  &  &  &  &  &  &  &  & \\ \hline
  \textbf{\small{12}} &  &  &  &  &  &  &  &  &  &  &  &  &  &  & \cellcolor{black} &  &  &  &  &  &  &  &  & \\ \hline
  \textbf{\small{13}} &  &  &  &  &  &  &  &  &  &  &  &  &  &  &  & \cellcolor{black} &  &  &  &  &  &  &  & \\ \hline
  \textbf{\small{14}} &  &  &  &  &  &  &  &  &  &  &  & \cellcolor{black} & \cellcolor{black} & \cellcolor{black} & \cellcolor{black} & \cellcolor{black} & \cellcolor{black} & \cellcolor{black} &  &  &  &  &  & \\ \hline
\end{tabular}
}
\end{center}
