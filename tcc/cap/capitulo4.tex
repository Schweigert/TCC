\chapter{Proposta para análise de arquiteturas \ac{mmorpg}}
\label{cap3}



As arquiteturas de serviços \ac{mmorpg} são desenvolvidas visando suprir as necessidades do projeto do jogo desenvolvido, de forma a viabilizar a utilização deste serviço.
%
Nesse sentido, jogos com mecânicas de projeto similares possuem implementações parecidas para os clientes.
%
Entretanto, a arquitetura escolhida impacta no custo de operação e qualidade do serviço aos jogadores.
%
Por este motivo, diferentes arquiteturas com a mesma regra de negócio são comparáveis entre si, visto que visam resolver os mesmos problemas.



Ao desenvolver um serviço \ac{mmorpg} é necessário decidir uma arquitetura que possibilite reduzir custos, consumo de recursos e minimize ocorrências para os jogadores a fim de viabilizar a sua implantação como produto.
%
Porém, a impossibilidade de comparação direta entre as arquiteturas de serviço \ac{mmorpg} instiga a análise das características básicas destas arquiteturas que possam influenciar o \textit{game design}, tais como consumo de recursos e tempo de resposta comparado ao número de jogadores simultâneos.
%
Sendo assim, uma análise do consumo de recursos computacionais das arquiteturas levantadas previamente na literatura tem valor científico no auxílio da escolha de implementações de arquiteturas de microsserviços, em específico para serviços \ac{mmorpg}.



Neste capítulo é descrita a proposta para análise de consumo de recursos computacionais em arquiteturas \ac{mmorpg}.
%
Inicialmente, é descrita a proposta em alto nível (Seção~\ref{sec:proposta}), trazendo os objetivos desta análise, quais recursos e métricas serão analisadas.
%
Os Critérios de Análise (Seção~\ref{sec:criterios}) exibem como os dados obtidos devem ser interpretados, baseando-se nos objetivos da análise das arquiteturas.
%
O Plano de Testes (Seção~\ref{sec:plano}) exibe como será realizada a coleta dos dados, descrevendo o ambiente, cenário, critérios e os testes que serão realizados.

\section{Proposta}
\label{sec:proposta}

Tendo analisado os trabalhos relacionados (Seção~\ref{sec:similares}) e as arquiteturas específicas para jogos \ac{mmorpg}, o presente trabalho tem como objetivo analisar as arquiteturas \ac{mmorpg} visando complementar a análise de arquitetura e consumo de recursos computacionais não analisados nos trabalhos relacionados.
%
Em específico, serão obtidos os valores referentes aos uso dos seguintes recursos computacionais nas arquiteturas Rudy (Subseção~\ref{rudy}), Salz (Subseção~\ref{salz}) e Willson (Subseção~\ref{willson}):

\begin{enumerate}
  \item \textbf{\ac{cpu}}: o uso de CPU, com sua representação sendo em relação a porcentagem de processamento nos núcleos utilizados;
  \item \textbf{Memória}: Quantidade de memória utilizada pelos processos do serviço/arquitetura. A sua representação será como dado absoluto; e
  \item \textbf{Rede}: Banda de rede utilizada nas operações de entrada e saída para cada microsserviço, utilizando valores absolutos;
\end{enumerate}

Além dos recursos computacionais, esta análise levará em conta valores referentes a outras métricas.
%
As métricas, cujos os valores serão obtidos são:

\begin{enumerate}
  \item \textbf{Número máximo de jogadores simultâneos}: Descobrir o limite de conexões para as arquiteturas propostas a análise. Será representado como valor absoluto.
  \item \textbf{Tempo de resposta das requisições}: Descobrir o tempo de resposta por categoria de requisição, conforme o número de jogadores no serviço. Será representado como tempo decorrido da requisição, em milissegundos.
\end{enumerate}

Todos estes valores serão obtidos a partir de clientes automatizados, e por este motivo faz-se necessário descrever o comportamento dos jogadores autônomos (Seção~\ref{sec:SimulaCliente}).
%
Espera-se, em situações adversas, caracterizar os comportamentos das arquiteturas bem como gargalos e os custos de recursos computacionais para manutenção das arquiteturas de microsserviços.
%
Para este fim, faz-se necessário a descrição dos critérios que serão utilizados durante a análise dos valores obtidos nos experimentos.

\section{Critérios de análise}
\label{sec:criterios}

A fim de padronizar a análise dos dados obtidos, estes serão estabelecidos usando como base o comportamento dos valores obtidos em referenciais.
%
Neste sentido, a análise dos dados obtidos será guiada pelo esperado dos valores obtidos em um serviço padrão:

\begin{enumerate}
  \item \textbf{Consumo CPU}: Espera-se estressar com um elevado número de requisições;
  \item \textbf{Consumo Memória}: Espera-se estressar com requisições nas quais exija armazenamento em memória;
  \item \textbf{Vazão Rede - Entrada}: Espera-se estressar com requisições nas quais tenham uma carga de dados elevada;
  \item \textbf{Vazão Rede - Saída}: Espera-se estressar com respostas nas quais tenham uma carga de dados elevada;
  \item \textbf{Número de Conexões Simultâneas}: Servirá como guia de comparação com os demais valores e desempenho da arquitetura, caso exista um limite de escalabilidade;
  \item \textbf{Tempo de resposta das requisições}: Servirá como guia de qualidade da arquitetura; e
\end{enumerate}

Em um caso de uso ideal, todos os recursos não são estressáveis, com um número de conexões simultâneas elevado.
%
Porém, espera-se para este trabalho um possível conjunto de ocorrências, na qual podem ou não ocorrer.
%
Este conjunto servirá de guia / exemplo de problemas relevantes retirado dos valores obtidos.
%
Logo, a automação e cenários de execução foram elaborados para forçar tais ocorrências.


A partir dos valores obtidos, e seguindo o esperado de uma arquitetura totalmente relacionada ao número de conexões, espera-se encontrar um conjunto de eventuais problemas nas arquiteturas.
%
Um conjunto exemplo destes problemas estão listados na Tabela~\ref{tab:problemas}.
\pagebreak

\begin{table}[htb!]
  \centering
  \caption{Possíveis conjuntos para a análise.}
  \label{tab:problemas}
  \begin{tabular}{l|l|l|l|l|l|l|l}
  \hline \hline
  \multicolumn{7}{|c|}{Recursos}                                                                      & \multirow{2}{*}{Descrição} \\ \cline{1-7}
  \rotatebox[origin=c]{90}{\ac{cpu}} & \rotatebox[origin=c]{90}{Memória} & \rotatebox[origin=c]{90}{Rede Entrada} & \rotatebox[origin=c]{90}{Rede Saída} & \rotatebox[origin=c]{90}{Conexões Simultâneas} & \rotatebox[origin=c]{90}{Tempo de Resposta} & \rotatebox[origin=c]{90}{Latência} &                            \\ \hline \hline
  $\uparrow$    &              &              &              & $\downarrow$ &              &              & \thead{Rotina de processamento de requisições\\está ocupando muita \ac{cpu}} \\ \hline
                & $\uparrow$   &              &              & $\downarrow$ &              &              & \thead{O microsserviço está armazenando\\informações as quais poderiam estar alocadas em outros\\microsserviços}  \\ \hline
                &              & $\uparrow$   &              & $\downarrow$ &              &              & \thead{Uma entrada de dados elevada pode indicar\\o uso de um protocolo\\inapropriado para o serviço} \\ \hline
                &              &              & $\uparrow$   & $\downarrow$ &              &              & \thead{Caso a saída esteja muito elevada\\pode indicar uma configuração inapropriada de elementos\\que são transitados na rede ou\\uso inadequado de protocolos} \\ \hline
                &              &              &              & $\downarrow$ & $\uparrow$   &              & \thead{Pode estar relacionado\\ao desempenho de processamento,\\ modelo de paralelismo ou congestionamento de rede} \\ \hline
                &              &              &              & $\downarrow$ & $\uparrow$   & $\uparrow$   & \thead{Está relacionado com\\ congestionamento da rede} \\ \hline
  $\downarrow$  &              & $\uparrow$   & $\uparrow$   &              &              &              & \thead{Possível gargalo na rede\\ou protocolo ineficiente} \\ \hline
  $\uparrow$    & $\uparrow$   & $\downarrow$ & $\downarrow$ &              &              &              & \thead{Possível gargalo nos algoritmos\\utilizados no serviço} \\ \hline
                &              &              &              & $\downarrow$ &              &              & \thead{Bloqueio de novas conexões pelo\\sistema operacional ou\\modelo de paralelismo} \\ \hline
  $\uparrow$    & $\uparrow$   & $\uparrow$   & $\uparrow$   & $\uparrow$   & $\downarrow$ &  $\uparrow$  & \thead{Limite de processamento da arquitetura} \\ \hline
  $\downarrow$  & $\downarrow$ & $\downarrow$ & $\downarrow$ & $\uparrow$   & $\downarrow$ &  $\downarrow$& \thead{Teste ideal} \\ \hline \hline


  \end{tabular}

  Fonte: O próprio autor.
\end{table}


Não foram encontrados trabalhos para guiar a caracterização dos dados, sendo assim a caracterização foi definida genericamente.
%
Dessa forma, a linha de base para a caracterização será encontrada ao decorrer da análise, utilizando dos valores de testes com poucas conexões (nenhum cliente e um cliente), esperando que o serviço escale linearmente conforme o inicio do processo.
%
A linha de base definida será uma das contribuições para trabalhos futuros.


A Tabela~\ref{tab:problemas} relaciona os recursos conforme dois padrões:

\begin{itemize}
  \item Valores acima da média ($\uparrow$).
  \item Valores baixo da média ($\downarrow$).
\end{itemize}

Espera-se encontrar problemas mais detalhados, além de problemas padronizados de forma genérica na Tabela~\ref{tab:problemas}.
%
Pode ser possível identificar eventuais problemas conforme o tipo de requisição e projeto da arquitetura.
%
Estes problemas servirão como guias na análise final das arquiteturas.

Seguindo estes critérios de análise, torna-se necessário definir um plano de testes a fim de obter os dados conforme os cenários e casos de uso definidos no atual trabalho.
%
Tais testes servem para ocasionar situações nos serviços a fim de obter dados para posterior análise.



\section {Plano de testes}
\label{sec:plano}



O plano de testes define os cenários que serão aplicados sobre as arquiteturas de microsserviços para jogos \ac{mmorpg} selecionadas.
%
Esta seção serve para descrever formas de estressar as arquiteturas, a fim de obter valores para análise.
%
Entretanto, antes de relatar os cenários de teste, é importante descrever o ambiente no qual serão realizados os experimentos.
%
A Figura~\ref{Ambiente de testes} descreve a infraestrutura utilizada para execução das camadas de aplicação utilizadas nos testes.



\begin{figure}[htb!]
  \caption{Ambiente de testes definido para a coleta de dados.}
  \label{Ambiente de testes}
  \includegraphics[width=0.8\textwidth]{img/cap3/infraestrutura.png}
  \centering

  Fonte: O próprio autor.
\end{figure}



Como visível na Figura~\ref{Ambiente de testes}, o ambiente de testes planejado está organizado em quatro regiões.
%
Essas regiões foram isoladas com o objetivo de diminuir o impacto de desempenho e consumo de recursos por outras ferramentas durante a coleta de dados.
%
Por este motivo, as regiões da infraestrutura planejada são:



\begin{enumerate}
  \item \textbf{Serviço de Jogo}: A camada de serviço da infraestrutura dos testes concentrará a arquitetura dos microsserviços referente às arquiteturas de microsserviços analisadas.
  \item \textbf{Banco de dados do serviço de jogo}: A camada de banco de dados do serviço de jogo conterá os serviços de dados e web a fim de manter um padrão de banco de dados para ambos os serviços utilizados e auxiliar na inicialização dos testes.
  \item \textbf{Estresse}: A camada de estresse será responsável por realizar requisições ao serviço a fim de estressá-lo, automatizando um cliente com padrões de requisições tal qual um jogador.
  \item \textbf{Dados}: A camada de dados será composta por um banco de dados de \textit{log} a fim de armazenar os dados obtidos da camada Cliente e Serviço, posteriormente utilizado exclusivamente na coleta de dados.
\end{enumerate}



Tais regiões da infraestrutura utilizada no ambiente de testes devem manter um padrão ao qual garante-se a inexistência de interferência entre as regiões, focando em obter métricas válidas para posterior análise.
%
Dessa forma, espera-se dividir as aplicações conforme a sua rede, facilitando o seu monitoramento.


Para os casos de uso, serão utilizadas as arquiteturas de microsserviços específicas a jogos \ac{mmorpg} obtidos da literatura.
%
São essas elas:



\begin{enumerate}
  \item Arquitetura Rudy (Subseção~\ref{rudy}), na qual baseia-se na segregação de jogadores por canais;
  \item Arquitetura Salz (Subseção~\ref{salz}), na qual baseia-se em gerar muitos serviços escaláveis; e
  \item Arquitetura Willson (Subseção~\ref{willson}), na qual baseia-se em extrair microsserviços de regras de negócio mais custosas.
\end{enumerate}



Tais arquiteturas vão impactar o serviço de jogo, banco de dados e as requisições as quais os clientes deverão realizar.
%
Espera-se obter os valores referente a diferença de consumo de recursos computacionais dentro de cenários controlados utilizando o ambiente de testes.



Com o objetivo de obter dados, torna-se claro a necessidade de estresse das arquiteturas em múltiplos casos diversos, garantindo assim a confiabilidade dos dados obtidos.
%
Dessa forma, foi desenvolvido um cenário que comporte ambas as arquiteturas de microsserviços propostas na análise.
%
Este cenário possibilita a execução do experimento junto a simulação de clientes.



\subsection{Cenário}



O cenário reflete diretamente sobre o ambiente proposto para esta análise.
%
Este será executado sobre cinco camadas, nas quais cada camada estará isolada em sub-redes em uma nuvem de computadores.

O cenário será composto por quatro sub-redes, as quais cada uma será responsável pela operação de uma região do ambiente planejado (Figura~\ref{fig:cenario}).
%
Esta divisão física em redes facilitará a orquestração dos serviços na rede, seguindo boas práticas de implantação de microsserviços.
%
Além disso, pode-se garantir que a interface do serviço público para o cliente segue o padrão de um serviço \ac{mmorpg} real.

\begin{figure}[htb!]
  \caption{Rede de execução dos testes.}
  \label{fig:cenario}
  \includegraphics[width=\textwidth]{img/cap3/cenario.png}
  \centering

  Fonte: O próprio autor.
\end{figure}

O cenário, exibido na Figura~\ref{fig:cenario} descreve cinco sub-redes.
%
Estas redes são descritas da seguinte forma:

\begin{enumerate}
  \item \textbf{Dados}: Armazena e processa os dados obtidos da arquitetura. É usado pelas redes \textit{Cliente} e \textit{Serviço de Jogo}.
  \item \textbf{Banco de dados do serviço de jogo}: Armazena dados do jogo. É usado pela rede \textit{Serviço de Jogo}.
  \item \textbf{Serviço de Jogo}: Executa sistemas \textit{web} e \ac{rpc}, em específico da camada de serviço de jogo. Gera métricas para os serviços na rede \textit{Dados} e manipula os bancos de dados na rede \textit{Banco de dados do serviço de jogo}.
  \item \textbf{Clientes}: Executa um número de clientes definido para obter métricas de tempo de resposta entre as redes \textit{Cliente} e \textit{Serviço de Jogo}.
\end{enumerate}

Conforme a descrição de cada rede, existe uma interdependência dentre as redes.
%
Tal interdependência pode ser visualizada na Tabela~\ref{tab:interdependencia}.

\begin{table}[htb!]
\centering
\caption{Tabela de interdependência das sub-redes.}
\label{tab:interdependencia}
\begin{tabular}{l||l|l|l|l}
\hline \hline
\multicolumn{1}{c||}{\rotatebox[origin=c]{-45}{Linha depende de Coluna}}  & \rotatebox[origin=c]{90}{Dados} & \rotatebox[origin=c]{90}{Banco de dados do serviço de jogo} & \rotatebox[origin=c]{90}{Serviço de Jogo} & \rotatebox[origin=c]{90}{Estresse} \\ \hline \hline
Dados                             & --    & Não                               & Não             & Não     \\ \hline
Banco de dados do serviço de jogo & Não   & --                                & Não             & Não      \\ \hline
Serviço de Jogo                   & Sim   & Sim                               & --              & Não      \\ \hline
Clientes                          & Não   & Não                               & Sim             & --       \\ \hline \hline
\end{tabular}

Fonte: O próprio autor.
\end{table}

A Tabela~\ref{tab:interdependencia} refere-se a dependência das redes, conforme os serviços necessários para o seu funcionamento.
%
Portanto, espera-se bloquear o tráfego de pacotes entre redes que são independentes.

A implantação do serviço de jogo utilizará métodos de implantação de microsserviços, com ferramentas para gerenciamento de nós em uma rede de contêineres.
%
Por este motivo, todas as máquinas virtuais da rede \textit{Serviço de Jogo} terão os mesmos recursos, facilitando o comportamento do gerenciador ao escalar os serviços.

A implantação dos bancos de dados do serviço de jogo devem ser implantadas diretamente no sistema operacional da máquina virtual.
%
Seguindo assim uma boa prática de não armazenar dados dentro de contêineres.
%
Por sua vez, estas máquinas virtuais serão focadas em armazenamento.

A estrutura de implantação da rede \textit{Clientes} será dada por um gerenciador de nós em uma rede de contêineres.
%
Esta rede terá um gestor de contêineres para aumentar a demanda conforme o caso de teste.

O limite de recursos do cenário é definido na Tabela~\ref{tab:limite_recursos}, sendo definida a faixa de endereços de cada rede a qual foi projetada.
%
Estes valores podem ser alterados conforme a necessidade dos testes, tendo seus recursos reduzidos ou incrementados.

\begin{table}[htb!]
\centering
\begin{adjustbox}{max width=\textwidth}
\caption{Limite de recursos por instância de cada rede.}
\label{tab:limite_recursos}
\begin{tabular}{l|l|l|l|l|l}
\hline \hline
Nome da Rede                      & Rede                     & Instâncias & Armazenamento / Ins. & N. Núcleos & Memória \\ \hline \hline
Dados                             & 10.0.*.* / 255.255.0.0   & 1          & 25GB                & 4          & 8GB     \\ \hline
Banco de dados do serviço de jogo & 10.51.*.* / 255.255.0.0  & 1          & 25GB                & 4          & 8GB     \\ \hline
Serviço de Jogo                   & 10.52.*.* / 255.255.0.0  & 4         & 25GB                 & 4          & 8GB     \\ \hline
Clientes                          & 10.101.*.* / 255.255.0.0 & 20          & 1TB                 & 8          & 16GB     \\ \hline \hline
\end{tabular}
\end{adjustbox}

Fonte: O próprio autor.
\end{table}

A partir da Tabela~\ref{tab:limite_recursos}, espera-se definir um limite de recursos máximos utilizados pelo Serviço de Jogo.
%
A única rede que pode variar suas características conforme o teste será a rede Clientes, visto que a demanda para estressar o serviço pode mudar conforme as características do teste e arquitetura utilizada.

A partir deste cenário é definido qual o comportamento dos clientes na execução dos testes automatizados.
%
Nesse sentido, uma definição das características mínimas de regra de negócio, padrão de comportamentos e interface esperada para o serviço e cliente é necessária.



\subsection{Automatização de Clientes}
\label{sec:SimulaCliente}



Utilizando a automatização de clientes objetiva-se estressar as arquiteturas utilizando um ataque com \textit{bots}.
%
Neste cenário, para padronizar a coleta de dados, todos os \textit{bots} terão a mesma rotina evitando assim um comportamento aleatório, a qual pode descaracterizar os dados obtidos para análise.



Porém, como requisito para estipular as requisições, faz-se obrigatório realizar um levantamento de requisitos no qual tanto o serviço quanto o cliente devem implementar.
%
Nesse sentido, a Tabela~\ref{tab:requisitos_funcionais} relaciona a funcionalidade com o impacto de implementação de tal funcionalidade.



\begin{table}[htb!]
\centering
\begin{adjustbox}{max width=\textwidth}
\caption{Requisitos das funcionalidades e respectivo impacto de implementação.}
\label{tab:requisitos_funcionais}
\begin{tabular}{l|l|l}
\hline \hline
Requisito                                                       & Descrição                                                                                                                                                                                  & Implementação                                                                                                                                                                             \\ \hline \hline
Identificação                                                   & \begin{tabular}[c]{@{}l@{}}Gera uma numeração única (token) com base\\ em uma tupla de dados.\end{tabular}                                                                                 & \begin{tabular}[c]{@{}l@{}}Será implementado utilizando algoritmo de \textit{hash},\\ de forma a garantir que este \textit{token} seja único e diferente a\\ cada implementação.\end{tabular}         \\ \hline
Autenticação                                                    & \begin{tabular}[c]{@{}l@{}}Recebe o \textit{token} e garante que não existe nenhuma\\ conexão utilizando o mesmo \textit{token}.\end{tabular}                                                                & \begin{tabular}[c]{@{}l@{}}Será implementado usando um serviço de chave valor,\\ como o Redis.\end{tabular}                                                                               \\ \hline
\begin{tabular}[c]{@{}l@{}}Seleção de\\ Personagem\end{tabular} & \begin{tabular}[c]{@{}l@{}}Uma conexão deve requerer o controle de um\\ personagem.\end{tabular}                                                                                           & \begin{tabular}[c]{@{}l@{}}Será implementado utilizando uma árvore de cena interna\\ ao serviço, onde o tipo do nó será Personagem e o seu nome\\ será o nome do personagem.\end{tabular} \\ \hline
Envio de Mensagem                                               & \begin{tabular}[c]{@{}l@{}}Será possível enviar mensagens e receber. Elas serão\\ baseadas na região. Será mantido uma distância fixa\\ para todos os casos de uso.\end{tabular}           & \begin{tabular}[c]{@{}l@{}}Deve existir uma estrutura de busca interna ao serviço para\\ consultar personagens de uma região em relação a um\\ usuário.\end{tabular}                      \\ \hline
Movimentação                                                    & \begin{tabular}[c]{@{}l@{}}Será possível movimentar o personagem para as\\ células adjacentes. Isso indica que o\\ posicionamento do personagem será baseado\\ em uma matriz.\end{tabular} & \begin{tabular}[c]{@{}l@{}}Esta ação deve comunicar a atualização para todos os demais\\ jogadores da região de interesse.\end{tabular}                                                   \\ \hline \hline
\end{tabular}
\end{adjustbox}

Fonte: O próprio autor.
\end{table}

A Tabela~\ref{tab:requisitos_funcionais} relata uma lista de funcionalidades mínimas que serão executadas na simulação.
%
A partir destas funcionalidades, pode-se definir quais requisições estarão disponíveis na \ac{api} para o cliente requisitar ao serviço.
%
A lista de comandos públicos é exibida na Tabela~\ref{tab:api_publica}.


\begin{table}[htb!]
\centering
\begin{adjustbox}{max width=\textwidth}
\caption{Requisitos mínimos funcionais para a implementação da simulação descrita.}
\label{tab:api_publica}
\begin{tabular}{l|l|l|l|l}
\hline \hline
Nome                  & Argumentos            & Retorno & Protocolo & Direção              \\ \hline \hline
\textit{Auth}                  & \textit{Username, Password}    & \ac{json} & \textit{Web}       & Cliente para Serviço \\ \hline
\textit{CreateAccount}         & \textit{Username, Password}    & \ac{json} & \textit{Web}       & Cliente para Serviço \\ \hline
\textit{UpdateAccount}         & \textit{Username, Password}    & \ac{json} & \textit{Web}       & Cliente para Serviço \\ \hline
\textit{CreateCharacter}       & \textit{Token, Character Name} & \ac{json} & \textit{Web}       & Cliente para Serviço \\ \hline
\textit{DeleteCharacter}       & \textit{Token, Character ID}   & \ac{json} & \textit{Web}       & Cliente para Serviço \\ \hline
\textit{SelectCharacter}       & \textit{Token, Character ID}   & \ac{json} & \ac{rpc}           & Cliente para Serviço \\ \hline
\textit{WalkTo}                & \textit{Token, PosX, PosY}     &           & \ac{rpc}           & Cliente para Serviço \\ \hline
\textit{ConsumeItem}           & \textit{Token, Item}           &           & \ac{rpc}           & Cliente para Serviço \\ \hline
\textit{AtackHere}             & \textit{Token}                 &           & \ac{rpc}           & Cliente para Serviço \\ \hline
\textit{SendMessage}           & \textit{Token, Message}        &           & \ac{rpc}           & Cliente para Serviço \\ \hline
\textit{UpdateMapEstate}       & \textit{NPC, Action, MoreData} &           & \ac{rpc}           & Serviço para Cliente \\ \hline
\textit{UpdateCharacterEstate} & \textit{NPC, Action, MoreData} &           & \ac{rpc}           & Serviço para Cliente \\ \hline
\textit{ReceiveMessage}        & \textit{NPC, Message}          &           & \ac{rpc}           & Serviço para Cliente \\ \hline
\textit{ReBind}                & \textit{IP, Port}              &           & \ac{rpc}           & Serviço para Cliente \\ \hline \hline
\end{tabular}
\end{adjustbox}
\end{table}

A Tabela~\ref{tab:api_publica} descreve todos os comandos que estarão disponíveis na rede.
%
Além destes, serão implementados outros comandos para \ac{api} privada do serviço.
%
Porém, os demais comandos da \ac{api} privada não serão utilizados pelo cliente, sendo necessariamente um requisito para o funcionamento do serviço com determinada arquitetura.


O mundo de jogo será baseado em matrizes.
%
Cada mapa pode ser visualizado como uma matriz de 100x100 unidades.
%
Dessa forma, temos um ambiente com valor exato, o qual facilita cálculos internos do serviço e cliente, assim promovendo um ambiente vasto porém sem utilizar recursos de forma exagerada para os testes.
%
Os personagens podem locomover-se para as células adjacentes a sua localização.
%
O raio de interesse é de quatro células.
%
Um exemplo de estado do ambiente do jogo pode ser visualizado na Figura~\ref{fig:roi}.

\begin{figure}[htb!]
  \caption{Área de interesse do ambiente do jogo com raio de quatro células.}
  \label{fig:roi}
  \includegraphics[height=8.0cm]{img/cap3/roi.png}
  \centering

  Fonte: O próprio autor.
\end{figure}

A partir da área de interesse do jogo, como o da Figura~\ref{fig:roi}, o \textit{bot} poderá decidir suas ações baseado em um autômato.
%
Caso ele alcance os extremos do mapa, ele será movimentado para outro mapa.
%
Nos casos de arquiteturas com múltiplos gerenciadores de mundo, será utilizado o comando \textit{ReBind} para realizar a conexão com o microsserviço correto após o translado do personagem.



Todas as ações do \textit{bot} no mapa são baseadas em um autômato.
%
Sendo assim, espera-se obter um padrão de movimentação a fim de evitar ciclos que um jogador comum dificilmente realizará (\textit{e.g.,} Andar para frente e para trás, ficar equipando itens em ciclo, ficar consumindo itens até acabar, \textit{etc.}).
%
A movimentação do personagem seguirá o autômato descrito na Figura~\ref{fig:movimentacao}.


\begin{figure}[htb!]
  \caption{Autômato de movimentação dos personagens automatizados.}
  \label{fig:movimentacao}
  \includegraphics[height=3.5cm]{img/cap3/movimentacao.png}
  \centering

  Fonte: O próprio autor.
\end{figure}

A Figura~\ref{fig:movimentacao} descreve o comportamento de um \textit{bot} no ambiente do jogo automatizado.
%
Ele seguirá um padrão de procurar outros jogadores para interagir pelo \textit{chat}.
%
Este conjunto de ações simula o comportamento de um jogador interagindo com outros jogadores em um jogo \ac{mmorpg}.
%
As características específicas de cada estado são definidas da seguinte forma:

\begin{itemize}
  \item Autenticar: Realizará autenticação com o serviço \textit{web} ou \textit{rpc} apropriado a arquitetura. Neste passo o \textit{bot} receberá as informações do seu personagem; e
  \item Movimentar: O personagem movimentará em busca de outro jogadores; e
  \item Comunicar: O personagem enviará uma mensagem aleatória no \textit{chat}.
\end{itemize}

Esta sequência de ações visa forçar aos \textit{bots} a exploração do cenário e interagir com outros jogadores.
%
A cada mudança de estado, o jogador irá anunciar no chat a sua troca de ação.
%
Isso contribuirá com o monitoramento do comportamento dos personagens tanto quanto usará a funcionalidade de chat.
%
Para simular a ação de resposta de mensagens de um jogador, cada \textit{bot} que receber uma mensagem terá uma porcentagem fixa de $25\%$ de probabilidade de responder uma mensagem aleatória.
%
Um \textit{bot} não poderá responder a uma mensagem na qual ele mesmo emitiu na região.

O ambiente final do jogo automatizado possui três mapas no eixo horizontal e no eixo vertical, visto que esta é a combinação mínima para que o serviço tenha um mapa central com bordas para efetuar transições de novos personagens.

Após definido as características dos clientes automatizados, pode-se definir os testes que serão executados sobre as arquiteturas de microsserviços específicos para jogos \ac{mmorpg}.
%
Tais testes servem de guia para obter as métricas para a análise das arquiteturas de microsserviços especificadas.

\subsection{Testes}
\label{sec:cap4_testes}

Os testes que serão executados no atual trabalho esperam obter informações a cerca das características de consumo de recursos ao  executar os microsserviços, analisando conforme o crescimento de jogadores simultâneos.
%
Nesse sentido, foram definidos os seguintes testes:

\begin{enumerate}
  \item Executar 3 vezes as arquiteturas com zero jogadores simultâneos, por 5 minutos;
  \item Executar 3 vezes as arquiteturas com um jogador apenas; e
  \item Executar 3 vezes as arquiteturas com zero jogadores simultâneos, aumentando a cada 30 segundos o número jogadores (um por vez) ao serviço, até o serviço obter algum erro interno de microsserviço e terminar a sua execução por erro interno ou chegar a 100 jogadores simultâneos.
\end{enumerate}

Os dados são capturados ao decorrer do tempo, podendo assim relacionar os recursos consumidos com o número de conexões simultâneas no serviço e com os demais recursos.
%
Esta estrutura auxiliará a visualização e interpretação dos dados para posterior análise.

Todos os testes serão executados utilizando a arquitetura Rudy, Salz e Willson.
%
Eles serão executados sequencialmente, aplicando sobre o cenário cada arquitetura e seus devidos clientes sobre o cenário, a fim de obter as métricas para posterior análise.
%
A partir desta sequência de testes, são obtidos valores suficientes para a análise das arquiteturas de microsserviços específicas a jogos \ac{mmorpg}.
%
Estes serão utilizados para análise conforme os critérios definidos na Seção~\ref{sec:criterios}, buscando possíveis gargalos e problemas de escalabilidade do sistema.


\section{Considerações parciais}


Este capítulo definiu os objetivos da análise de microsserviços específicos a jogos \ac{mmorpg}.
%
Para alcançar tais objetivos, se fez necessário definir quais métricas serão obtidas e como estes dados serão analisados.
%
Por fim, definiu-se qual o ambiente, cenário e testes que serão executados para obter tais dados.

Dessa forma, foi definido que será obtido os valores de uso de \ac{cpu}, memória, vazão de rede (tanto para entrada quanto saída), número de conexões simultâneas, tempo de resposta das requisições e latência entre cliente e serviço.
%
Estes valores serão analisados conforme possíveis problemas conhecidos, definidos na seção de critérios (Seção~\ref{sec:criterios}).

A fim de obter tais valores para análise, o plano de testes definiu um ambiente baseado em camadas conforme a demanda das arquiteturas submetidas a análise.
%
Foram projetadas cinco sub-redes no cenário de testes baseado nas regiões definidas no ambiente de testes.
%
A partir destas sub-redes, foi estipulado valores de recursos computacionais limites e números de instâncias máximas.
%
Entretanto, não foi possível definir o limite de instâncias para a sub-rede de estresse, visto que não foi encontrado estudos anteriores para ter uma linha base de consumo de recursos.
%
Dessa forma, o atual trabalho contribuirá conjuntamente com futuros trabalhos guiando o consumo de recursos para as arquiteturas Rudy, Salz e Willson.
%
Por fim, foram definidas as regras de negócio básicas para os clientes simulados, baseados em um autômato.

Os testes que serão executados sobre as redes estão baseados na finalidade de obtenção de valores de recursos mínimos para execução dos serviços e crescimento de valores de recursos conforme um número de clientes simultâneos.
