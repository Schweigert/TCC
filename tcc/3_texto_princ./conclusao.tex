\chapter{Considera\c{c}\~oes finais}

Neste trabalho apresentou-se a fundamenta��o te�rica referente a programa��o linear e a programa��o linear inteira,
com a finalidade de definir o problema de aloca��o de tripula��o, do ingl�s \textit{crew scheduling problem} (CSP).
Apresentou-se tamb�m o precedimento de gera��o de colunas, utilizado para resolver problemas com um grande n�mero
de vari�veis, que � o caso do CSP. Modelou-se os dois componentes da gera��o de colunas para o CSP: O problema
mestre e o subproblema. Por fim apresentou-se um conjunto de m�todos para resolver o CSP juntamente com a proposta
deste trabalho.

Durante o desenvolvimento deste trabalho pode-se identificar os principais conceitos necess�rios para compreender
o processo de solu��o do CSP. Pode-se ainda identificar um problema de interesse pr�tico e utilizado em larga
escala no setor de transporte terrestre e a�reo.

O trabalho desenvolvido neste curso faz parte de um projeto de Inicia��o Cient�fica (IC) e a continua��o de outro
projeto de IC, na qual foi desenvolvido e publicado um artigo no Simp�sio Brasileiro de Pesquisa Operacional (SBPO).
Este projeto de IC atualmente conta com dois bolsistas e um professor, onde atualmente todos pesquisam na area
de pesquisa operacional, em espec�fico o CSP e gera��o de colunas.
